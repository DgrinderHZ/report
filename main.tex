%%%%%%%%%%%%%%%%%%%%%%%%%%%%%%%%%%%%%%%%%
% Masters/Doctoral Thesis 
% LaTeX Template
% Version 2.5 (27/8/17)
%
% This template was downloaded from:
% http://www.LaTeXTemplates.com
%
% Version 2.x major modifications by:
% Vel (vel@latextemplates.com)
%
% This template is based on a template by:
% Steve Gunn (http://users.ecs.soton.ac.uk/srg/softwaretools/document/templates/)
% Sunil Patel (http://www.sunilpatel.co.uk/thesis-template/)
%
% Template license:
% CC BY-NC-SA 3.0 (http://creativecommons.org/licenses/by-nc-sa/3.0/)
%
%%%%%%%%%%%%%%%%%%%%%%%%%%%%%%%%%%%%%%%%%

%----------------------------------------------------------------------------------------
%	PACKAGES AND OTHER DOCUMENT CONFIGURATIONS
%----------------------------------------------------------------------------------------

\documentclass[
openany,
11pt, % The default document font size, options: 10pt, 11pt, 12pt
%oneside, % Two side (alternating margins) for binding by default, uncomment to switch to one side
french, % ngerman for German
singlespacing, % Single line spacing, alternatives: onehalfspacing or doublespacing
%draft, % Uncomment to enable draft mode (no pictures, no links, overfull hboxes indicated)
%nolistspacing, % If the document is onehalfspacing or doublespacing, uncomment this to set spacing in lists to single
%liststotoc, % Uncomment to add the list of figures/tables/etc to the table of contents
%toctotoc, % Uncomment to add the main table of contents to the table of contents
%parskip, % Uncomment to add space between paragraphs
%nohyperref, % Uncomment to not load the hyperref package
headsepline, % Uncomment to get a line under the header
%chapterinoneline, % Uncomment to place the chapter title next to the number on one line
%consistentlayout, % Uncomment to change the layout of the declaration, abstract and acknowledgements pages to match the default layout
]{MastersDoctoralThesis} % The class file specifying the document structure

\usepackage[utf8]{inputenc} % Required for inputting international characters
\usepackage[T1]{fontenc} % Output font encoding for international characters

\usepackage{mathpazo} % Use the Palatino font by default

\usepackage{float} % figure position
%\usepackage[thinlines]{easytable}
%\setlength\parskip{1\baselineskip}

\usepackage[backend=bibtex,style=authoryear,natbib=true]{biblatex} % Use the bibtex backend with the authoryear citation style (which resembles APA)

\addbibresource{example.bib} % The filename of the bibliography


\usepackage[autostyle=true]{csquotes} % Required to generate language-dependent quotes in the bibliography
\usepackage{hyperref}
\hypersetup{
	colorlinks=false    
}
\usepackage{makecell}
\usepackage{amsmath,amsfonts,amssymb,amsthm,epsfig,epstopdf,titling,url,array}

\renewcommand\theadalign{bc}
\renewcommand\theadfont{\bfseries}
\renewcommand\theadgape{\Gape[2pt]}
\renewcommand\cellgape{\Gape[2pt]}
\setcounter{secnumdepth}{3}
%----------------------------------------------------------------------------------------
%	MARGIN SETTINGS
%----------------------------------------------------------------------------------------

\geometry{
	paper=a4paper, % Change to letterpaper for US letter
	inner=2.5cm, % Inner margin
	outer=3.8cm, % Outer margin
	bindingoffset=.5cm, % Binding offset
	top=1.5cm, % Top margin
	bottom=1.5cm, % Bottom margin
	%showframe, % Uncomment to show how the type block is set on the page
}

%%----------------------------------------------------------------------------------------
%%	THESIS INFORMATION
%%----------------------------------------------------------------------------------------
%
%\thesistitle{Thesis Title} % Your thesis title, this is used in the title and abstract, print it elsewhere with \ttitle
%\supervisor{Dr. James \textsc{Smith}} % Your supervisor's name, this is used in the title page, print it elsewhere with \supname
%\examiner{} % Your examiner's name, this is not currently used anywhere in the template, print it elsewhere with \examname
%\degree{Doctor of Philosophy} % Your degree name, this is used in the title page and abstract, print it elsewhere with \degreename
%\author{John \textsc{Smith}} % Your name, this is used in the title page and abstract, print it elsewhere with \authorname
%\addresses{} % Your address, this is not currently used anywhere in the template, print it elsewhere with \addressname
%
%\subject{Biological Sciences} % Your subject area, this is not currently used anywhere in the template, print it elsewhere with \subjectname
%\keywords{} % Keywords for your thesis, this is not currently used anywhere in the template, print it elsewhere with \keywordnames
%\university{\href{http://www.university.com}{University Name}} % Your university's name and URL, this is used in the title page and abstract, print it elsewhere with \univname
%\department{\href{http://department.university.com}{Department or School Name}} % Your department's name and URL, this is used in the title page and abstract, print it elsewhere with \deptname
%\group{\href{http://researchgroup.university.com}{Research Group Name}} % Your research group's name and URL, this is used in the title page, print it elsewhere with \groupname
%\faculty{\href{http://faculty.university.com}{Faculty Name}} % Your faculty's name and URL, this is used in the title page and abstract, print it elsewhere with \facname
%
%\AtBeginDocument{
%\hypersetup{pdftitle=\ttitle} % Set the PDF's title to your title
%\hypersetup{pdfauthor=\authorname} % Set the PDF's author to your name
%\hypersetup{pdfkeywords=\keywordnames} % Set the PDF's keywords to your keywords
%}

\begin{document}
\DeclareDocumentEnvironment{abstract}{ O{} }{
	\checktoopen
	\tttypeout{\abstractname}
	#1%added to be able to have abstract more than one page long
	\thispagestyle{plain}
	\begin{center}
		{\huge\textit{\abstractname} \par}
	\end{center}
}

\frontmatter % Use roman page numbering style (i, ii, iii, iv...) for the pre-content pages

\pagestyle{plain} % Default to the plain heading style until the thesis style is called for the body content

%----------------------------------------------------------------------------------------
%	TITLE PAGE
%----------------------------------------------------------------------------------------

\begin{titlepage}
\begin{center}

%\includegraphics{Logo} % University/department logo - uncomment to place it
\begin{figure}[H]
	\centering
	\includegraphics[width=0.5\textwidth]{logo}
\end{figure}
{\scshape\LARGE Université Moulay Ismail \\
	Faculté des Sciences et Techniques d’Errachidia\\
	Département d’Informatique \par}\vspace{1.5cm} % University name

\textbf{MEMOIRE}\\

Présenté pour l’obtention du diplôme de:\\

\textsc{\Large Master en Sciences et Techniques \\ Système d’information Décisionnels et Imagerie (SIDI)}\\[0.5cm] % Thesis type

\HRule \\[0.4cm] % Horizontal line
{\huge \bfseries Les descripteurs de la forme et de la texture pour l’indexation et la recherche d’image par le contenu : Accélération par M-tree \par}\vspace{0.4cm} % Thesis title
\HRule \\[1.5cm] % Horizontal line
 
\begin{minipage}[t]{0.4\textwidth}
\begin{flushleft} \large
\emph{Présenté par:}\\
%\href{http://www.johnsmith.com}{} % Author name - remove the \href bracket to remove the link
Hassan ZEKKOURI
\end{flushleft}
\end{minipage}
\begin{minipage}[t]{0.4\textwidth}
\begin{flushright} \large
\emph{Encadré par:} \\
%\href{http://www.jamessmith.com}{} % Supervisor name - remove the \href bracket to remove the link  
 Brahim AKSASSE
 

\end{flushright}
\end{minipage}\\[1cm]

\begin{center} \large
	\emph{Les membres de jury:}\\
	
	Mohammed OUANAN (Président)\\
	Hassan SILKAN (Examinateur)\\
	Mohamed OUHDA (Examinateur)\\
	\small
	Brahim AKSASSE (Encadrant)
\end{center}
 
%\vfill

%\large 

%\large \textit{A thesis submitted in fulfillment of the requirements\\ for the %degree of \degreename}\\[0.3cm] % University requirement text
%\textit{in the}\\[0.4cm]
%\groupname\\\deptname\\[2cm] % Research group name and department name
 
\vfill

{\large 17/10/2020}\\[4cm] % Date
%\includegraphics{Logo} % University/department logo - uncomment to place it
 
\vfill
\end{center}
\end{titlepage}


%----------------------------------------------------------------------------------------
%	DECLARATION PAGE
%----------------------------------------------------------------------------------------

%\begin{declaration}
%\addchaptertocentry{\authorshipname} % Add the declaration to the table of contents
%\noindent I, \authorname, declare that this thesis titled, \enquote{\ttitle} and the work presented in it are my own. I confirm that:
%
%\begin{itemize} 
%\item This work was done wholly or mainly while in candidature for a research degree at this University.
%\item Where any part of this thesis has previously been submitted for a degree or any other qualification at this University or any other institution, this has been clearly stated.
%\item Where I have consulted the published work of others, this is always clearly attributed.
%\item Where I have quoted from the work of others, the source is always given. With the exception of such quotations, this thesis is entirely my own work.
%\item I have acknowledged all main sources of help.
%\item Where the thesis is based on work done by myself jointly with others, I have made clear exactly what was done by others and what I have contributed myself.\\
%\end{itemize}
% 
%\noindent Signed:\\
%\rule[0.5em]{25em}{0.5pt} % This prints a line for the signature
% 
%\noindent Date:\\
%\rule[0.5em]{25em}{0.5pt} % This prints a line to write the date
%\end{declaration}
%
%\cleardoublepage
%
%----------------------------------------------------------------------------------------
%	QUOTATION PAGE
%----------------------------------------------------------------------------------------

%\vspace*{0.2\textheight}
%
%\noindent\enquote{\itshape Thanks to my solid academic training, today I can write hundreds of words on virtually any topic without possessing a shred of information, which is how I got a good job in journalism.}\bigbreak
%
%\hfill Dave Barry




%----------------------------------------------------------------------------------------
%	ACKNOWLEDGEMENTS
%----------------------------------------------------------------------------------------

\begin{Large}
	
	
	\dedicatory{
		\begin{Huge}
			\textbf{REMERCIEMENTS}\\
			
		\end{Huge}
		
		Je remercie avant tout, Allah de m'a\\
		
		prodiguée la force morale et physique et m'a\\
		
		permis d’achever ce travail.\\
		
		Je tiens tout d’abord à remercier M. Brahim Aksasse pour  m'avoir encadré tout au long de ce présent projet, pour sa disponibilité, ses critiques constructives, et ses
		suggestions pertinentes.\\
		
		Je remercie vivement Mohammed Ouanan d'avoir accepté de juger notre travail.\\
		
		Je remercie vivement Hassan Silkan  d'avoir accepté de juger notre travail.\\
		
		Je remercie vivement Mohamed Ouhda d'avoir accepté de juger notre travail.\\
		
		Je remercie  toutes les personnes qui m'ont\\
		
		aidé et entouré durant cette thèse de près ou de loin.\\
		
		Je remercie  tous les enseignants de la faculté des sciences et techniques d'Errachidia que je respecte beaucoup.\\
		
		Je n'oublie jamais les auteurs à la fin de ce document. Grâce
		à leurs excellents références et articles, nous retirons les
		connaissances dans le domaine de recherche d'images basée sur le contenu.\\
		
		
		Enfin, je remercie toutes ma familles et mes
		amis.
		
		
	} 
	
	
\end{Large}


%----------------------------------------------------------------------------------------
%	DEDICATION
%----------------------------------------------------------------------------------------

\dedicatory{
	\begin{Huge}
		\centering
		\textbf{Dédicaces}\\
		
	\end{Huge}
	Tous les mots ne sauraient exprimer la gratitude,\\
	L'amour, le respect, la reconnaissance…\\
	Aussi, c’est tout simplement que\\
	\textbf{Je dédie cette thèse …} \\
	
	A la mémoire de mon père. Que Dieu ait vos âmes dans sa sainte
	miséricorde.\\
	
	A ma mère, aucune dédicace ne saurait exprimer mon respect, mon amour éternel et ma considération pour les sacrifices que vous avez consenti pour mon instruction et mon bien être.
	Puisse Dieu, le Très Haut, vous accorder santé, bonheur et longue vie et faire en sorte que jamais je ne vous déçoive. \\
	
	A mes chers et adorable frères et sœurs.\\
	
	A tous mes professeurs de département d’informatiques pour leurs
	conseils.\\
	
	A Mes amis et amies de par le monde qui n'ont cessé de m'encourager.\\
	
	A toutes les personnes qui ont participé à
	l’élaboration de ce travail à tous ceux que j’ai
	omis de citer.\\
	
	Que ce travail soit l’accomplissement de vos vœux
	tant allégués, et le fuit de votre soutien infaillible,
	Merci d’être toujours là pour moi.
	

} 



%----------------------------------------------------------------------------------------
%	ABSTRACT PAGE
%----------------------------------------------------------------------------------------

\begin{abstract}
	\addchaptertocentry{\abstractname} % Add the abstract to the table of contents
	L'image numérique joue un rôle important dans de nombreuses activités humaines et a remplacé quasiment l'images traditionnelles. De ce fait, la quantité de bases d’images numériques a explosé et l’accès à ces images devient de plus en plus difficile et coûteux. En réponse à ces problèmes, plusieurs outils ont émergé avant le développement des systèmes de recherche d’images basés sur le contenu. L’objectif principal de cette mémoire est d’étudier le domaine de recherche d’images par le contenu et de réaliser un système de recherche d’images à partir de ces connaissances.\\
	
	\textbf{Mots clés:} image numérique, base d’images, CBIR, indexation d’image,
	descripteur, recherche d’image par le contenu, couleur, texture, forme.\\

\textbf{---------------------------------------------- Abstract ----------------------------------------------}\\

	The digital image plays an important role in many human activities and has largely replaced traditional images. As a result, the quantity of digital image databases has exploded and access to these images is becoming increasingly difficult and expensive. In response to these problems, several tools have emerged ahead of the development of content-based image retrieval systems. The main objective of this thesis is to study content-based image retreival and to build an image search engine based on this knowledge.\\
	
	\textbf{Key words:} digital image, image database, CBIR, image indexing, descriptor,
	image search engine, color, texture, shape.
\end{abstract}

%----------------------------------------------------------------------------------------
%	LIST OF CONTENTS/FIGURES/TABLES PAGES
%----------------------------------------------------------------------------------------

{
	\hypersetup{
		colorlinks=false,
		urlcolor=black,
		linkcolor=black}
	 
	\tableofcontents % Prints the main table of contents
	\let\cleardoublepage\clearpage
	\listoffigures % Prints the list of figures
	\let\cleardoublepage\clearpage
	\listoftables % Prints the list of tables

	
}



%----------------------------------------------------------------------------------------
%	ABBREVIATIONS
%----------------------------------------------------------------------------------------

\begin{abbreviations}{ll} % Include a list of abbreviations (a table of two columns)

\textbf{CBIR} & \textbf{C}onten \textbf{B}ased \textbf{I}mage \textbf{R}etrieval\\
\textbf{TBIR} & \textbf{T}ext \textbf{B}ased \textbf{I}mage \textbf{R}etrieval\\
RIC & (Recherche d'iformation par le contenu)\\

\textbf{CIE} & \textbf{C}ommission  \textbf{I}nternationale de l'\textbf{E}clairage\\
\textbf{QBIC} & IBM's \textbf{Q}uery \textbf{B}y \textbf{I}mage \textbf{C}ontent\\

 \textbf{DWT} & \textbf{D}iscrete \textbf{W}avelet \textbf{T}ransform\\
 
 \textbf{DF} & \textbf{D}escripteurs de \textbf{F}ourier \\
 
 \textbf{ART} & \textbf{A}ngular \textbf{R}adial \textbf{T}ransform\\
 
\textbf{SVAD} & \textbf{S}omme des \textbf{V}aleurs \textbf{A}bsolue des \textbf{D}ifférences \\

\textbf{MAM} & \textbf{M}éthodes d’\textbf{A}ccès \textbf{B}étrique \\

\textbf{MT} & \textbf{M}-\textbf{T}ree

\end{abbreviations}
\let\cleardoublepage\clearpage
%----------------------------------------------------------------------------------------
%	PHYSICAL CONSTANTS/OTHER DEFINITIONS
%----------------------------------------------------------------------------------------

%\begin{constants}{lr@{${}={}$}l} % The list of physical constants is a three column table
%
%% The \SI{}{} command is provided by the siunitx package, see its documentation for instructions on how to use it
%
%Speed of Light & $c_{0}$ & \SI{2.99792458e8}{\meter\per\second} (exact)\\
%%Constant Name & $Symbol$ & $Constant Value$ with units\\
%
%\end{constants}
%
%----------------------------------------------------------------------------------------
%	SYMBOLS
%----------------------------------------------------------------------------------------

%\begin{symbols}{lll} % Include a list of Symbols (a three column table)
%
%$a$ & distance & \si{\meter} \\
%$P$ & power & \si{\watt} (\si{\joule\per\second}) \\
%%Symbol & Name & Unit \\
%
%\addlinespace % Gap to separate the Roman symbols from the Greek
%
%$\omega$ & angular frequency & \si{\radian} \\
%
%\end{symbols}


%----------------------------------------------------------------------------------------
%	THESIS CONTENT - CHAPTERS
%----------------------------------------------------------------------------------------

\mainmatter % Begin numeric (1,2,3...) page numbering

\pagestyle{thesis} % Return the page headers back to the "thesis" style

% Include the chapters of the thesis as separate files from the Chapters folder
% Uncomment the lines as you write the chapters


\chapter*{Introduction générale}
La recherche d'information peut être considérer parmi les activités quotidiennes de tout être humain d'aujourd'hui. Cette recherche n'a pas commencé avec le développement de l’Internet mais plutôt avant. Avec la révolution numérique de ces dernières décennies et l’avancement continu des capacités de calcul et de stockage des machines, l’information numérique est devenue le cœur de tous les secteurs d’activités. Ces progrès se sont accompagnés d’une expansion rapide des bases d'images, ce qui a compliqué les méthodes de recherche traditionnelle.\\

L'image numérique devient alors une nécessité grâce à son pouvoir expressif et la facilité de son transfert, de ce fait, la quantité d'images sur le web augmente rapidement, ce qui donne naissance à un nouveau besoin que l’on ne connaissait pas auparavant à savoir des outils capables d’aider l’utilisateur dans l’indexation et la recherche d’images.\\

Les premières solutions utilisées pour la recherche d’image sont basées sur la recherche textuelle par des annotations. Cette technique consiste à ajouter à chaque image un ensemble de mots-clés qui vont décrire son contenu visuel. De là, un système de gestion de base sera capable de trouver les images similaires.\\

Aujourd’hui, les moteurs de recherche d’images avancés sont classés en deux catégories selon leur principe de fonctionnement; la première catégorie exploite les caractéristiques visuelles des images et la deuxième catégorie utilise les concepts sémantiques associés avec les images.\\

Nous nous intéressons dans notre travail à la première classe, connue sous le nom de la recherche d’images par le contenu « Content Based Image Retrieval ou CBIR».
Dans ce mode, l’image est représentée par un ensemble de descripteurs
numériques dans une base d'index qui essayent de décrire des attributs visuels différents, généralement de bas niveaux comme la couleur, la forme et la texture. L’utilisateur est appelé à donner une image exemple ou requête qui ressemble à ce qu’il cherche. La recherche consiste ici de mesurer la similarité de manière automatique entre la requête et les images de la base d'index selon certains critères préétablis.\\

Ce mémoire se compose de quatre chapitres qui nous permettront de
présenter les différents aspects de notre travail. Le premier chapitre présente une études bibliographique des généralités sur le traitement d’image numérique et le principe d'un systèmes de recherche d’image par contenu (CBIR). Le deuxième chapitre donne une vue plus détaillée sur les systèmes de recherche d’images par le contenu surtout la partie de descripteur et celle de mesure de similarité. Le troisième chapitre a pour but d’exposer la partie d'accélération de recherche à l'aide de la structure d'index M-Tree. Le dernier chapitre est dédié pour la partie expérimentations et les résultats obtenus par notre système.
% Chapter Template

\chapter{Étude bibliographiques} % Main chapter title

\label{ChapterX} % Change X to a consecutive number; for referencing this chapter elsewhere, use \ref{ChapterX}

%----------------------------------------------------------------------------------------
%	SECTION 1
%----------------------------------------------------------------------------------------

\section{Introduction}
Le domaine de l’image numérique est un domaine en pleine expansion. Depuis quelques années, avec l’explosion d’Internet et aussi le développement à grande échelle de la photographie
numérique de haute qualité que nous observons ces dernières années, ce
qui a conduit à un développement constant des bases de données d’images. Le contenu de ces images peut être décrit à deux niveaux différents: 
\begin{itemize}
	\item Au niveau numérique, une image contient des pixels colorés dont on peut extraire des descripteurs de couleurs, des textures et des formes. 
	\item Au niveau sémantique, une image peut être interprétée et peut avoir au moins un sens. 
\end{itemize}

Malheureusement, dans les systèmes d'information d'aujourd'hui, les images sont décrites numériquement alors que les utilisateurs s'intéressent à leur contenu sémantique, il est actuellement difficile de trouver des correspondances entre le niveau numérique et le niveau sémantique. Alors, l’utilisation et la
gestion de ces bases d’images d’une manière efficace est très problématique, elle nécessite de nouvelles techniques de manipulation et de traitement des données. Dans ce contexte, la recherche d’images par le contenu s’intéresse à découvrir des connaissances implicitement contenues dans un ensemble de données en s’appuyant sur différentes approches qui peuvent être mises en œuvre
indépendamment ou couplées. Ces techniques visent à explorer et à décrire le contenu des données, et à en extraire l’information la plus importante et la plus significativement pertinente.
\\

Bien que plusieurs années de travaux scientifiques en recherche d’images par le contenu, ont d’ores et déjà permis la réalisation et le développement des systèmes performants, permettant plusieurs formes et techniques d’indexation par analyse du contenu, mais ce domaine demeure encore aujourd’hui un sujet très ouvert et actif dans la communauté internationale depuis plus d’une dizaine d'années.
\\

L'objectif de ce chapitre est de présenter le principe général des systèmes CBIR. Puis nous nous intéresserons à l'étude des descripteurs et les mesures de similarités les plus utilisés dans l’objectif d’avoir une idée sur quelques approches existantes dans la littérature. 


%-----------------------------------
%	SECTION 1
%-----------------------------------

\section{Représentation machine du contenu visuel des images}

Avant d’entrer dans le sujet proprement dit, il est important de comprendre la nature des objets que nous allons manipuler. Intéressons-nous à la notion d’image, qu’est-ce qu’une image ?

Une des plus anciennes définitions de l'image est celle donnée par Platon [Platon] : « J'appelle image d'abord les ombres ensuite les reflets qu'on voit dans les eaux, ou à la surface des corps opaques, polis et brillants et toutes les représentations de ce genre ». En informatique, une image est une représentation numérique en mémoire d’un sujet imprimé sur une rétine artificielle (matricielle comme le capteur d’un appareil photographique numérique
ou la scène virtuelle d’une image de synthèse ou bien linéaire comme le capteur optique du télécopieur, du photocopieur ou du scanner). \\

On distingue deux types d’images à la composition et au comportement différent: images matricielles et les images vectorielles.

\subsection{Images matricielles (ou images bitmap)}
Les images matricielles sont des images numériques qui stockent les informations sous la forme d’une matrice, des points à plusieurs dimensions, chaque dimension représentant une dimension spatiale (hauteur, largeur), ou autre (niveau de résolution). Dans le cas des images à deux dimensions, les points sont appelés pixels. Elles peuvent être regroupées en plusieurs catégories: images à niveaux de gris, image couleur, image stéréoscopiques et image multicomposante.


\begin{figure}[H]
	\centering
	\includegraphics[width=0.6\textwidth]{Figures/imgnum} 
	\caption{Image numérique dont une portion est très agrandie: notion de pixel.}
\end{figure}


Un pixel $(i, j)$, ( i est l’indice de la ligne et j est l’indice de la colonne), possède une valeur $I(i, j)$ qui peut être un scalaire représentant la valeur du niveau de gris du pixel (dans le cas des images noir et blanc ou des images en niveaux de gris), ou un vecteur représentant les trois canaux de la couleur du pixel (dans le cas des images couleurs).

\begin{figure}[H]
	\centering
	\includegraphics[width=0.6\textwidth]{Figures/pixel} 
	\caption{Image numérique au niveaux de gris: notion de pixel.}
\end{figure}

%-----------------------------------
%	SUBSECTION 1
%-----------------------------------



\subsubsection{Image à niveaux de gris}

Une image numérique à niveaux de gris est une matrice $N \times M$ de pixels. Nous notons $N$ le nombre de lignes et $M$ le nombre de colonnes de l'image. Les images à niveaux de gris sont composées de pixels de valeurs scalaires représentant la luminosité/intensité. En général ces valeurs des pixels sont  codés sur $n$ bits, ce qui lui confère des valeurs entières comprises entre 0 (noir) et $2^n-1$ (blanc), 0 et 255 si $n = 8$. Dans ce cas, le pixel est codé sur un octet (nous disposerons ainsi de $2^8=256$ couleurs). La valeur 255 correspond au blanc, et la valeur 0 correspond au noir. Les valeurs intermédiaires correspondent à des niveaux de gris allant du noir au blanc.\\

La figure ci-dessous montre un sous-tableau de $5 \times 5$ pixels extrait d’une image. Nous pouvons voir les valeurs qui composent le tableau et les niveaux de gris qui permettent d’afficher l’image.

\begin{figure}[H]
	\centering
	\includegraphics[width=0.4\textwidth]{Figures/gray} 
	\caption{Image à niveaux de gris.}
\end{figure}

%-----------------------------------
%	SUBSECTION 1
%-----------------------------------

\subsubsection{Image couleur}
Une image couleur est composée de pixels dont les valeurs sont en général multicomposantes. \\

En effet, nous pouvons citer parmi les formats les plus utilisés pour représenter la valeur du pixel, le triplet (R,V, B) ou (R,G,B) où R, G et B sont respectivement les valeurs des composantes rouge, verte et bleu du pixel. Chaque composante du triplet est représentée par un entier variant entre 0
(absence de la composante) et 255 (intensité maximale). 

\begin{figure}[H]
	\centering
	\includegraphics[width=0.6\textwidth]{Figures/grayvscol} 
	\caption{Image à niveaux de gris VS Image couleur.}
\end{figure}

Le triplet (0, 0, 0) correspond au noir, (255, 0, 0) au rouge, (255, 255, 0) au jaune et (255, 255, 255) au blanc. 
Dans ce cas, le pixel est codé sur trois octets. Une image couleur RVB ou (RGB) possède trois composantes tandis qu'une image en niveaux de gris n’en possède qu’une seule. En d'autre termes, Une image couleur correspond à la synthèse additive de 3 images, rouge, vert et bleu. Chaque pixel est donc codé sur $3 \times n = 24 $ bits.

\begin{figure}[H]
	\centering
	\includegraphics[width=0.4\textwidth]{Figures/rgb} 
	\caption{Image couleur dans l’espace RGB.}
\end{figure}

\subsection{Images vectorielles}

Les images vectorielles, contrairement aux images matricielles, contiennent les primitives de dessin (formes, position, couleurs...) des objets géométriques qu’elles représentent (segments de droite, polygones, arcs de cercles...). Ces images sont essentiellement utilisées pour réaliser des schémas ou des plans. Leur codage dépend directement du logiciel qui a permis de les créer. Ces images présentent deux avantages: elles occupent peu de place en mémoire et peuvent être facilement redimensionnées sans perte d'information.

\begin{figure}[H]
	\centering
	\includegraphics[width=0.4\textwidth]{Figures/vecteur} 
	\caption{Une image vectorielle: redimensionnable sans perte de qualité, contrairement à une image matricielle.}
\end{figure}



%-----------------------------------
%	SECTION 2
%-----------------------------------

\section{Systèmes de recherche d’image par contenu (CBIR)}
Les chercheurs dans le domaine de la vision par ordinateur se posent le problème de l'indexation automatique des images par leur contenu, qui permet la recherche dimages par le
contenu (CBIR).\\

CBIR (Content Based Image Retrieval) est un système qui utilise des contenus visuels pour récupérer des images à partir d'une base de données d'images. Ce système est devenu indispensable parce qu'il peut effectivement surmonter les problèmes d'un TBIR (Text Based Image Retrieval). Dans le CBIR, le contenu visuel est extrait par plusieurs techniques:
histogramme, segmentation... Il est également décrit par le vecteur de caractéristiques multidimensionnel. La performance du système de recherche d'images basées sur le contenu est principalement influencée par la qualité et la pertinence du vecteur de caractéristiques et la mesures de similarité. \\

Le CBIR diffère de la recherche d’information textuelle essentiellement par le fait que les bases de données d’images sont non-structurées, les images numériques n’étant que des matrices
d’intensités de pixels, sans signification inhérente les unes par rapport aux autres. Donc avant même de commencer à faire des hypothèses sur le contenu de l’image, une des questions clé dans tout type de traitement d’image est l’extraction de l’information utile à partir de ces matrices de pixels \textbf{[Quel08]}.\\

Le principe général de la recherche d'image par le contenu comporte deux étapes:
\begin{itemize}
	\item  Lors d'une première phase hors ligne (étape d'indexation), on calcule les descripteurs des images et on les stocke dans une base de données appelé base des indix.
	\item La seconde phase, dite de recherche se déroule en ligne. L'utilisateur soumet une image comme requête. Le système calcule le descripteur selon le même mode que lors de la première phase d'indexation. Ainsi, ce descripteur est comparé à l'ensemble des descripteurs préalablement stockés dans la base d'index pour en ramener les images les plus semblables à la requête.
\end{itemize}

\begin{figure}[H]
	    \label{fig:cbir_principe}
		\centering
		\includegraphics[width=0.65\textwidth]{Figures/cbir_principe} % Include the image .png
		\caption{ Schéma général d’un CBIR.}
\end{figure}

Lors de la phase d'indexation, le calcul de descripteur consiste en l'extraction de caractéristiques visuelles des images telles que:
\begin{itemize}
	\item la texture (filtre de Gabor, transformée en ondelettes discrète…)
	\item la couleur (la segmentation, les points d’intérêts, les régions d’intérêts,
	histogramme de couleurs, histogrammes dans l'espace RGB ou dans TSV, …),
	\item les formes (descripteurs de Fourier,Zernik, ART…),
	\item une combinaison de plusieurs de ces caractéristiques.
\end{itemize}

Ces caractéristiques sont dites de bas-niveau, car elles sont très proches du contenu signal (pixel), et ne véhiculent pas de sémantique particulière sur l'image.\\

Une fois ces caractéristiques extraites, la comparaison consiste généralement à définir diverses distances entre ces caractéristiques, et de définir une mesure de similarité globale
entre deux images. Au moyen de cette mesure de similarité et d'une image requête, on peut alors calculer l'ensemble des mesures de similarités entre cette image requête et l'ensemble des images de la base d'images. On peut alors ordonner les images de la base suivant leur score, et présenter le résultat à l'utilisateur, les images de plus grand score étant considérées comme les plus similaires.\\

Ce genre de système ne nécessite pas forcément une image requête pour rechercher d'autres images. Par exemple, il est possible de rechercher des images plutôt bleues, ou alors de dessiner une forme et demander de chercher toutes les images qui possèdent un objet de forme similaire.



Une des étapes essentielles en analyse d'images par le contenu est l'extraction d'une description de bas niveau de l’image. Cette description est une représentation numérique (analyse statistique, analyse quantitative, ...) des caractéristiques visuelles de l'image, généralement sous la forme d’un vecteur ou d’un ensemble de vecteurs.
Cette description doit avoir deux caractéristiques importantes :
\begin{itemize}
	\item elle doit conserver suffisamment d’information pour être discriminante, c’est-à-dire différencier deux images avec un contenu visuel différent;
	\item elle doit être la plus invariante possible (aux bruits; aux variations d’échelle; aux variations de contraste; aux déformations; etc) pour pouvoir généraliser le contenu visuel d’une image à une autre image qui n’est pas identique.
\end{itemize}


Nous présentons dans la suite les différentes familles de descripteurs.


\subsection{Descripteur globaux et descripteurs locaux}
On peut résumer l’ensemble des informations visuelles de l’image en un unique descripteur
global, ou plusieurs descripteurs locaux caractérisant chacun une partie de l’image. Les
techniques modernes en imagerie tendent à privilégier les descripteurs locaux aux globaux car
les descripteurs locaux sont plus efficaces et ils permettent une recherche plus fine et
absorbent mieux certaines variations.

\subsubsection{Descripteurs globaux}
Dans le cas de descripteurs globaux, un seule descripteur décrit la totalité de l’image, cela
les rend robustes au bruit qui peut affecter le signal, les histogrammes de couleur et des
niveaux de gris en sont des exemples classiques [Stricker 94]. De nombreux descripteurs
globaux sont également basés sur la texture et la couleur. La couleur aussi fait partie des
premières primitives visuelles utilisées. Le plus simple des descripteurs globaux basés sur la
couleur consiste à construire l’histogramme des couleurs.
L’inconvénient de ces descripteurs est qu’ils ne permettent pas de distinguer des parties de
l’image, ils ne distinguent pas, par exemple, les objets dans l’image, sauf dans le cas où
l’image ne contient qu’un seule objet sur un fond uni.

\subsubsection{Descripteurs locaux}
Les descripteurs locaux s’associent à une partie/région de l’image qu’on commence par
détecter avant de calculer le descripteur, cette partie peut concerner un objet par exemple, la
détection se fait indépendamment de la position dans l’image, ce qui assure l’invariance par
translation, rotation, etc .

Les descripteurs locaux sont de nos jours les caractéristiques visuelles les plus couramment
utilisées en analyse d’image par le contenu. Ils permettent de décrire l’ensemble des
informations visuelles d’une région de l’image en un descripteur (vecteur). La région
d’extraction est appelée région d’intérêt et elle est généralement centrée sur un point d’intérêt
Les descripteurs locaux ont une très bonne capacité de discrimination pour déterminer si
deux régions sont similaires ou dissimilaires. Cela les rends particulièrement utiles en vision
par ordinateur, notamment pour de la mise en correspondance de points d’intérêts. Ils sont
utilisés en : odométrie visuelle [Nistér 04] ; reconstruction 3D [Mouragnon 06] ; détection
d’objets [Lowe 99] ; recherche d’images par le contenu [Perronnin 10], [Negrel 14] ; etc.
Dans le chapitre suivant, nous présentons un descripteur local basé sur la segmentation par
k-means.

\subsection{Combinaison des descripteurs}

Les attributs: couleur, texture, forme décrivent les images par leur contenu visuel. La
combinaison de ces attributs peut être mieux caractérisée le contenu. Il est donc intéressant de
combiner ces différents attributs pour une recherche plus efficace et plus discriminante. Les
problèmes qui se posent lors de la combinaison de ces différents attributs pour la recherche et
l’indexation sont au moins de trois ordres :
\begin{itemize}
	\item L'espace de description : Le choix de l'espace de description consiste à rechercher les
	attributs visuels significatifs de la base de données d'images, l'ensemble de ces
	attributs étant représenté par un nuage de points dans un espace dimensionnel haut,
	alors les vecteurs contiennent plusieurs attributs, un problème qui se pose est celui de
	la dimension de l'espace de description. Ce problème est connu dans la communauté
	des bases de données par la malédiction de la dimension, lorsque le nombre de
	dimensions augmente, le volume de l'espace croît rapidement si bien que les données
	se retrouvent « isolées » et deviennent éparses.
	\item La mesure de la similarité : Il s'agit d'une étape essentielle dans tout système de
	recherche. Dans le cas où les images sont décrites par différents attributs, une solution
	classique pour mesurer la similarité est de calculer séparément les mesures de
	similarité pour chaque attribut et de déduire ensuite une mesure composite de la
	similarité globale entre les images. Cela suppose évidemment que les différents
	attributs sont indexés séparément (avec des structures d'index séparées). Dans la base de données, il y a peu de méthodes qui utilisent plusieurs index pour structurer les
	données. Une autre difficulté liée à la similitude est de déterminer comment combiner
	plusieurs mesures souvent définies sur des domaines différents, avec des dynamiques
	différentes, des degrés d'importance différents, surtout pour l'utilisateur, mais aussi de
	natures différentes.
	\item Structuration : la phase de construction d'une structure d'index est une étape utile dans
	le cas où les données sont volumineuses et appartiennent à un grand espace de
	description. Il s'agit de structurer les nuages de points relatifs aux descripteurs des
	images et de les stocker efficacement dans une machine. Cette tâche de structuration
	peut s'avérer difficile dans le cas où les données à structurer sont de nature hétérogène.
	La difficulté réside dans le choix de la distance à utiliser pour structurer (mise en place
	d'un index) et dans la standardisation des différents types de données.
\end{itemize}


\subsubsection{Espace des couleurs}
Une couleur est généralement représentée par trois composantes. Ces composantes définissent un espace des couleurs. On peut citer l'espace RVB, l'espace CIE (Commission Internationale de l'Eclairage) XYZ ou Yxy, ou encore l'espace Lab. Selon l'espace de couleurs choisi pour représenter une image couleur, le nuage des couleurs (c'est à dire l'ensemble des couleurs de l'image) n'aura pas la même répartition dans l'espace 3D.

\begin{figure}[H]
	\label{fig:espaceRVB}
	\centering
	\includegraphics[width=0.65\textwidth]{Figures/espaceRVB} % Include the image .png
	
	\caption{Espace de couleur RVB.}
	
\end{figure}

Les espaces de couleurs classiques, tels que le RVB, CIE XYZ, ...etc, sont issus d'une approche purement physique, sans la prise en compte de données psychophysiques. Dans d'autre espaces de couleur, tels que l'espace Lab, l'approche physique est corrigée selon des données de la vision humaine.\\

De nombreuses méthodes de descriptions d'images proposent de caractériser la couleur dans certains espaces couleurs pour profiter des propriétés de ces derniers. La figure 1.6
montre les principaux espaces couleurs utilisés en indexation d’images.

\begin{figure}[H]
	\label{fig:espaceCouleur}
	\centering
	\includegraphics[width=0.45\textwidth]{Figures/espaceCouleur} % Include the image .png
	\caption{Les principaux espaces couleurs utilisés en indexation d’images.}
	
\end{figure}

On peut principalement citer :
\begin{itemize}
	\item \textbf{RGB} (Rouge, Vert, Bleu): est le plus utilisé car la plupart des images originelles sont codées dans cet espace couleur, ce qui ne nécessite pas de transformation inter
	espace couleur, donc facilement applicable.
	
	\item \textbf{HSV}: chaque composante représente respectivement la teinte, la saturation et la luminance.
	
	\item \textbf{YCbCr}: est utilisé dans les normes MPEG 1, 2 et 4, ses composantes sont décorrélées et de faibles dynamiques, ce qui permet de bons taux de compression.
	
	\item \textbf{L*a*b*} ou \textbf{CIE Luv}: sont des espaces couleurs perceptuellement uniformes. Ces espaces ont été créés dans le but de rendre plus homogène l’espace des couleurs et de
	permettre de mesurer uniformément les distances entre couleurs en tout point de l'espace. Deux couleurs proches dans ces espaces couleurs sont proches perceptuellement. Ces espaces sont grandement utilisés dans les systèmes de comparaison d'images.
\end{itemize}
%----------------------------------------------------------------------------------------
%	SECTION 2
%----------------------------------------------------------------------------------------


\section{Conclusion}
Ce chapitre a fait l’état de l’art sans exhaustivité des différents descripteurs des attributs
visuels pouvant être utilisés pour la recherche d’images par le contenu ainsi que les approches
correspondantes. Aussi, nous avons dressé une liste des types de descripteurs et les mesures
de similarités avec leurs avantages et leurs inconvénients. Le chapitre suivant se focalisera sur
notre solution détaillée, le schéma de CBIR, les techniques d’extraction de descripteur. Le
choix d’un meilleur descripteur et d’une mesure de similarité promet une bonne pertinence
d’un système CBIR.
% Chapter Template

\chapter{Descripteurs d'Images \&  Mesures de Similarité} % Main chapter title

\label{ChapterX} % Change X to a consecutive number; for referencing this chapter elsewhere, use \ref{ChapterX}

%----------------------------------------------------------------------------------------
%	SECTION 1
%----------------------------------------------------------------------------------------

\section{Introduction}
Aujourd’hui avec le développement des systèmes multimédias et le recul de l’écrit, nous utilisons de plus en plus le contenu visuel comme support de communication dans différents domaines. En effet l’image et la vidéo numérique sont partie intégrante de tels systèmes par la
densité et la richesse de leur contenu. La même image peut présenter plusieurs significations à différents niveaux : analyse, description, reconnaissance et interprétation.

La recherche d’information couvre le traitement de documents numériques impliquant la structure, l’analyse, le stockage et l’organisation des données. Dans le passé, le terme recherche d’information était lié au concept de l’information textuelle. Actuellement « RI »est associé à tout type d’information, textuelle, visuelle ou autre. Cependant dû aux limitations des méthodes textuelles, le développement des méthodes basées sur le contenu visuel est devenu primordial. Ceci explique l’activité de recherche intense consacrée au système CBIR ces dernières années. Le « RIC » est souvent confronté au problème de pertinence de la
recherche, et au temps de recherche. L’objectif de n’importe quel système CBIR est de satisfaire la requête d’un utilisateur par la
pertinence des résultats. Comme l’accès à un document via sa pure sémantique est impossible, les systèmes CBIR traditionnels s’appuient sur un paradigme de représentation de bas niveau du contenu de l’image, par la couleur, la texture, la forme, etc.…, et d’autres par une
combinaison de celles-ci. La recherche d’images se fait ainsi par comparaison des descripteurs.

L’analyse et la représentation du contenu des données sources mises sous forme de vecteur caractéristique. L’information obtenue dans cette étape est une sorte de résumé des images de la base (segmentation en régions, couleur, texture, relations spatiales,…). La transformation
est généralement gourmande en temps de calcul.

Dans la suite de ce chapitre, nous présentons les différents attributs utilisés dans les systèmes de recherche d’image par contenu et ensuite les mesures de similarité entre les images après la définition de leurs descripteurs.




%----------------------------------------------------------------------------------------
%	SECTION 2
%----------------------------------------------------------------------------------------

\section{Descripteurs d’image}

%----------------------------------------------------------------------------------------
%	SUBSECTION 1
%----------------------------------------------------------------------------------------
\subsection{Descripteur de la couleur}
La couleur est l’information visuelle la plus utilisée dans les systèmes de recherche par le contenu. Ces valeurs tridimensionnelles font que son potentiel discriminatoire soit supérieur à la valeur en niveaux de gris des images. Mais la caractérisation de la couleur dans une image est une opération extrêmement complexe.

En effet, cette donnée varie considérablement avec l’orientation des surfaces, le point de vue de la caméra et l’illumination (positions et longueur d’onde des sources lumineuses). En outre, la perception de la couleur par l'être humain est un processus complexe et subjectif.

Chaque pixel d'une image numérique est constitué d'un élément de couleur. Dans une image en niveaux de gris, cette couleur varie généralement de 0 à 255, où 0 est noir, 255 est blanc, et les valeurs entre les deux sont les suivantes différentes nuances de gris, du noir au blanc. Dans une image couleur, disons avec une résolution couleur 24 bits (ce qui signifie que 24 bits sont utilisés pour les informations de couleur pour chaque pixel), une partie (souvent paire) des 24 bits est affectée à chacun des trois composants de couleur dans l’image. L'espace colorimétrique le plus couramment utilisé est RVB signifie Rouge-Vert-Bleu (RGB: ReedGreen-Blue). L'espace de couleur est défini comme un modèle pour représenter la couleur en termes de valeurs d'intensité RVB, et dans cet exemple l'image couleur 24 bits, 8 bits sont utilisées pour représenter chacun des composants. Dans cet exemple, les composantes de couleur permettent de représenter $(2 ^ 8) ^ 3$ couleurs différentes, ce qui est difficile de les
distinguer par l'oeil humain.

\begin{figure}[H]
	\label{fig:imageRVB}
	\centering
	\includegraphics[width=0.65\textwidth]{Figures/imageRVB} % Include the image .png
	
	\caption{Image couleur dans l’espace RVB.}
	
\end{figure}

\begin{figure}[H]
	\label{fig:tableRVB}
	\centering
	\includegraphics[width=0.65\textwidth]{Figures/tableRVB} % Include the image .png
	
	\caption{Représentation des pixels en fonction de nombre des bits.}
	
\end{figure}

La couleur est devenue un attribut largement utilisé dans les systèmes opérationnels de recherche d'images par le contenu. Elle facilite l'identification et l'extraction d'un objet dans une scène [Zavi01]. Il semble que son efficacité à ce stade soit liée au fait que l'être humain peut distinguer des milliers de couleurs et seulement 24 niveaux de gris [Gonz02].

\subsubsection{L'histogramme}
La plus grande majorité des systèmes de recherche d'images par le contenu se base sur la description des couleurs composant les images. De nombreux travaux ont vu le jour quant à l'utilisation de la couleur pour la recherche d’images par le contenu. L’approche la plus courante dans la littérature est d'indexer la couleur par l'utilisation d'histogrammes de couleurs [Swain91] L'histogramme des couleurs exprime la distribution statistique de celles-ci dans l'image.

Ce type d'histogramme est calculé typiquement sur un espace caractéristique quantifié. Chaque valeur de caractère dans l'histogramme représente donc un rang de couleur dans la
palette. 

L'histogramme a été introduit pour la première fois en RIC par [Swain 91], depuis il est très utilisé à cause de sa simplicité de calcul, son invariante aux changements d'échelle et aux transformations géométriques.\\

\begin{figure}[H]
	\label{fig:hist}
	\centering
	\includegraphics[width=0.65\textwidth]{Figures/hist} % Include the image .png
	\caption{Exemple d’histogramme d’une image couleur.}
\end{figure}

Les inconvénients majeurs de l'histogramme sont:
\begin{itemize}
	\item La perte de toute information spatiale dont la texture ou la forme. Par exemple un histogramme d'un tapis rouge peut être très proche de celui d'une porte rouge ou d'une voiture rouge.
	
	\item Ils sont de grandes tailles, donc par conséquent il est difficile de créer une indexation rapide et efficace en les utilisant tels qu'ils sont. 
	
	\item Ils sont sensibles à de petits changements de luminosité, ce qui est problématique pour comparer des images similaires, mais acquises dans des conditions différentes. 
	
	\item Ils sont inutilisables pour la comparaison partielle des images (objet particulier dans une image), puisque calculés globalement sur toute l’image.
	
\end{itemize}
   Des méthodes alternatives ont été proposées pour augmenter l'efficacité de l'histogramme. Nous citons quelques-unes : 
\begin{itemize}
	\item les moments de la couleur [Stricker 94], [Ravishankar 99],
	\item les constantes de couleur [Flickner 95], [Worring 01],
	\item la signature couleur [Kender 98], 
	\item les blobs [Chang 97] et le vecteur cohérent de couleur [Pass 97].\\
\end{itemize}

Une nouvelle technique à base d'histogrammes locaux de couleurs a été proposée par Wang [Wang 01]. Cette technique est insensible à la rotation. Son principe consiste à diviser une image en un ensemble de blocs égaux et calcule leur histogramme de couleurs. Ainsi elle utilise un graphe biparti pour calculer la distance ayant le coût minimal entre deux images. Enfin chaque bloc de l'image requête est comparé à tous les blocs des images de la base afin de retrouver les images similaires.\\

De point de vue de la recherche CBIR, Alaoui [Alao11] a présenté deux nouveaux descripteurs : le premier descripteur est l’entropie de distribution de couleur « CDE » qui introduit l'entropie pour décrire l'information spatiale des couleurs. Le deuxième descripteur est l’entropie hybride de couleur « CHE » qui introduit une description spatiale à caractère multi-résolution d'images couleurs. Les résultats expérimentaux montrent qu'un système d'indexation CBIR basé sur les descripteurs « CDE » et « CHE » est plus performant que les systèmes CBIR traditionnels basée sur les histogrammes locaux.\\

\subsubsection{Les moments de couleur}
Les moments de couleur on été utilisés dans plusieurs systèmes de recherche d’images par le contenu tel que QBIC, mathématiquement les trois premiers moments sont définis par :
\begin{figure}[H]
	\label{fig:moments}
	\centering
	\includegraphics[width=0.65\textwidth]{Figures/moments} % Include the image .png
	\caption{Moments}
	
\end{figure}

Les moments de couleur est une représentation compacte comparée aux autres descripteurs de couleur. Car seulement 9 valeurs (3 pour chaque composante chromatique) sont utilisées pour représenter le contenu d’une image. Pour cette raison ils peuvent diminuer le pouvoir de discrimination (description). 

\subsubsection{Conclusion}
L'information relative aux couleurs est particulièrement importante dans la caractérisation d’une image. Plusieurs études ont été menées pour trouver un critère de choix des descripteurs de couleurs pour l'indexation des images, mais aucune n'a abouti. Ceci peut s’expliquer par le manque de subjectivité de cette information, les descripteurs couleur ne suffisent pas à indexer efficacement une image, ni à la chercher.

Dans plusieurs domaines d’application, l’utilisation de descripteurs résumant l’information globale d’images couleurs, tels que le descripteur des histogrammes locaux de couleurs des images entières, n’offre pas toujours des résultats satisfaisants. Globalement, l’histogramme couleur reste le descripteur le plus utilisé. Bien qu’il ne contienne qu’une information partielle en raison de l’absence d’indication sur les caractéristiques spatiales, l’histogramme garde un fort pouvoir de description, ce qui explique sa grande utilisation.

\begin{figure}[H]
	\label{fig:samehist}
	\centering
	\includegraphics[width=0.65\textwidth]{Figures/sameHist} % Include the image .png
	\caption{Deux images différentes de même histogramme.}
	
\end{figure}

%----------------------------------------------------------------------------------------
%	SUBSECTION 2
%----------------------------------------------------------------------------------------
\subsection{Descripteur de la texture}
 
Dans [Tuceryan 98], l’auteur présente 4 familles d'outils de caractérisation de la texture. On distingue parmi elles les méthodes statistiques, les méthodes géométriques, les méthodes à base de modèles probabilistes et les méthodes fréquentielles. En utilisant des filtres prédéfinis, la méthode de [Laws 80] utilise des convolutions spatiales pour construire 25 versions d’une image texturée, chaque version décrit une caractéristique précise de l’image.\\


Au même titre que la couleur, l'identification de la texture est une partie importante du système visuel humain, c'est un arrangement spatial de pixels, habituellement dans des modèles visuels homogènes que ni la couleur ni l'intensité moyenne ne décrivent suffisamment. La modélisation et la description des textures sont un problème difficile. La texture est une caractéristique intuitive facile à reconnaître mais difficile à définir.

D’après [7], une définition formelle de la texture est quasiment impossible.

\begin{description}
	\item[Définition:] 	
	D’une manière générale, la texture dans une image peut être définie comme un arrangement spatial de couleurs ou d'intensités dans une région de cette image [Linda 01]. Elles peuvent consister en un placement structuré d’éléments mais peuvent aussi n’avoir aucun élément répétitif.
\end{description}


\begin{figure}[H]
	\label{fig:textures}
	\centering
	\includegraphics[width=0.65\textwidth]{Figures/textures} % Include the image .png
	
	\caption{Des textures différentes.}
	
\end{figure}

De nombreuses définitions ont été proposées, mais aucune ne convient parfaitement aux différents types de textures rencontrées. Dans une définition couramment citée [36], la texture est présentée comme une structure disposant de certaines propriétés spatiales homogènes et invariantes par translation. Cette définition stipule que la texture donne la même impression à l'observateur quelle que soit la position spatiale de la fenêtre à travers laquelle il observe cette texture. Par contre l’échelle d’observation doit être précisée. On peut le faire par exemple en précisant la taille de la fenêtre d’observation.\\

La notion de texture est liée à trois concepts principaux:
\begin{enumerate}
	\item un certain ordre local qui se répète dans une région de taille assez grande,
	
	\item cet ordre est défini par un arrangement structuré de ses constituants élémentaires,
	
	\item ces constituants élémentaires représentent des entités uniformes qui se caractérisent par des dimensions semblables dans toute la région considérée.\\ 
	
\end{enumerate}

Il existe un grand nombre de textures. On peut les séparer en deux classes:
\begin{itemize}
	\item Les textures
	structurées (macrotextures): constituée par la répétition d’une primitive à intervalle régulier. On peut différencier dans cette classe les textures parfaitement périodiques (carrelage, damier, ...etc.), les textures dont la primitives subit des déformations ou des changements d'orientation (mur de briques, grains de café, ...etc.).
	
	\item Les textures aléatoires (microtextures): se distinguent en général par un aspect plus fin (sable, herbe, ...etc.). Contrairement aux textures de type structurel, les textures aléatoires ne comportent ni primitive isolable, ni fréquence de répétition. On ne peut donc pas extraire de ces textures une primitive qui se répète dans l’image mais plutôt un vecteur de paramètres statistiques homogènes à chaque texture.
\end{itemize}

Dans tous les cas, ces objectifs nécessitent l'extraction d'un ou de plusieurs paramètres caractéristiques de cette texture. Nous désignerons ces paramètres sous le terme d’attributs texturaux (textural features) et l’ensemble qu’ils constituent sous le terme de descripteur de texture.

Certains de ces paramètres correspondent à une propriété visuelle de la texture (comme la directionnalité ou la rugosité). D'autres correspondent à des propriétés purement mathématiques auxquelles il est difficile d'associer une qualification perceptive.

Un recensement ainsi qu'une classification des termes de description des textures employés par les principaux auteurs pourront être trouvés dans [38] et [39].

Il a été prouvé dans la littérature que l'analyse de descripteurs de texture est efficace pour de nombreuses applications, y compris la vision artificielle, la segmentation d'images médicales, la cartographie urbaine, l'analyse d'images satellitaires, la télé-détection.\\

La texture représente également un descripteur bas niveau efficace utilisé dans le cadre de l'indexation et la recherche par le contenu. Les attributs texturaux peuvent être obtenus à partir d’un ensemble assez vaste de différentes théories mathématiques, nous citons notamment :

\subsubsection{Les matrices de co-occurrences (Mesures de Haralick)}

 En 1973, Haralick a proposé une méthode en se basant sur la matrice de co-occurrence de niveaux de gris [Haralick 73], elle est probablement une méthode classiques et la plus célèbre pour l'analyser de la texture.
 
 La texture d’une image peut être interprétée comme la régularité d’apparition de couples de niveaux de gris selon une distance (pas $k=1, 2, 3, ...etc$) donnée dans l’image. La matrice de co-occurrences contient les fréquences spatiales relatives d’apparition des niveaux
 de gris selon quatre directions: $\theta = 0◦$,  $\theta =  \frac{\pi}{4} = 45◦$,  $\theta =  \frac{\pi}{2} = 90◦$,  $\theta =  3 \times \frac{\pi}{4} = 135◦$. Une matrice de co-occurrences est définie au moyen d’une relation entre deux pixels .\\

\begin{description}
	\item[Définition] La matrice de coocurrence $P_{d,\theta}=(P_{d,\theta}(i, j))_{1\leq i,j \leq Ng}$ est une matrice carrée de taille $Ng \times Ng$. Avec:
	\begin{itemize}
		\item Ng étant le nombre de niveaux de gris de l'image (256x256). Les indices de la matrice de co-occurrences sont
		donc les niveaux de gris de la texture étudiée.
		\item On réduit souvent a des tailles 8x8, 16x16 ou 32x32.
		\item L’élément $P_{d,\theta}(i, j)$de la matrice de cooccurrence définit la fréquence d'apparition des couples de niveaux de gris $i$ et $j$ pour les couples de pixels séparés par une distance $d$ dans la direction $\theta$.
	\end{itemize} 
Pour obtenir de véritables fréquences relatives, il faut normaliser les éléments de la matrice en les divisant par
le nombre total de paires de points élémentaires séparés par la distance d dans la direction dans toute l’image
 
\end{description}

Alors on peut calculer plusieurs matrices, pour chaque distance (pas) et direction, mais avec un temps de calcul de ces matrices est assez long.

Soit l’image I définie par:

\begin{figure}[H]
	\label{fig:cooc}
	\centering
	\includegraphics[width=0.65\textwidth]{Figures/cooc} % Include the image .png
	\caption{Classification des méthodes d’extraction de forme.}
\end{figure}

On parcours l'image et pour chaque couple de pixels formé avec la distance et la direction donn´ees, on incrémente la matrice des cooccurrences de 1. Alors la matrice de cooccurrence (non normalis´ee) est:

\begin{figure}[H]
	\label{fig:cooc_nn}
	\centering
	\includegraphics[width=0.8\textwidth]{Figures/cooc_non_norm} % Include the image .png
	\caption{Classification des méthodes d’extraction de forme.}
\end{figure}

Le 2 de la matrice de cooccurrence (ligne 1 et colonne 4) signifie que l’on trouve deux fois un pixel de valeur 1 de distance 1 et de direction
0 d’un pixel de valeur 4.

A partir de cette matrice de cooccurrence, il est possible de définir plusieurs descripteurs (Mesures de Haralick), tels que ceux répertoriés dans cette table:

\begin{table}[H]
	\centering
	\begin{tabular}{|{c}|l{c}|l|}
		
		\hline
		\textbf{ Mesure} & \textbf{  Formulation} \\
		\hline
		Uniformité/Energie & $E = \sum_{i=0}^{n}\sum_{j=0}^{n} P_{d,\theta}(i, j) $     \\
		\hline
		Entropie & $En = -\sum_{i=0}^{n}\sum_{j=0}^{n} P_{d,\theta}(i, j) \log_2 (P_{d,\theta}(i, j)) $    \\
		\hline
		Contraste & $ C = \sum_{i=0}^{n}\sum_{j=0}^{n} P_{d,\theta}(i, j) (i-j)^2 $ \\
		\hline
	Moment inverse de différence & $ M = \frac{1}{1+(i-j)^2} \sum_{i=0}^{n}\sum_{j=0}^{n} P_{d,\theta}(i, j)  $
	
	\end{tabular}
\caption{Mesure de Haralick}
\end{table}
 
 
\subsubsection{Transformée en ondelettes}
 Le terme " ondelette " en anglais (wavelet) a été utilisé pour la première fois en 1984 par J. Morlet et A. Grossmann pour résoudre des problèmes de traitement des signaux pour la prospection pétrolière.
 
 Une ondelette est une fonction à la base de la décomposition en ondelettes, décomposition similaire à la transformée de Fourier à court terme. La transformée en ondelettes consiste à décomposer un signal par une cascade de filtres pour générer une famille d’ondelettes appelées les ondelettes filles, obtenues par la translation et la dilatation d’une fonction mère. Chaque ondelette à une certaine fréquence pendant un temps limité, de la même façon que les notes de musique. La formule suivante présente une ondelette fille :
\begin{center}
	\begin{equation}
		$ \psi_{a,b}(t) = \frac{1}{\sqrt{a}} \psi(\frac{t-b}{a}) $
	\end{equation}
\end{center}
  

Le paramètre a est le facteur d’échelle, il détermine la dilatation de l'atome de base (ondelette fille). Le paramètre b est le facteur de translation, il permet de translater l'atome de base à gauche ou à droite. Le paramètre $ \frac{1}{\sqrt{a}} $ est un facteur de normalisation à travers les différentes échelles.\\

La transformée en ondelettes est un outil mathématique récent qui décompose un signal en fréquences en conservant une localisation spatiale. Le signal de départ est projeté sur un ensemble de fonctions de base qui varient en fréquence et en espace. Ces fonctions de base s’adaptent aux fréquences du signal à analyser. Cette transformation permet donc d’avoir une localisation en temps et en fréquence du signal analysé. La transformation en ondelettes continue est définie par :
		 \begin{center}
		 	\begin{equation}
		 	$ \tilde{f}(a,b) = \underset{-\infty }{\overset{+\infty }{\int }}f\left(t\right) \psi_{a,b}^*(t) dt   $
		 	\end{equation}
		 \end{center}
Dans cette équation, $ \psi_{a,b}^*(t) $ est la fonction de base d’ondelette. L’étoile «*» représente le complexe conjugué.

À partir de la transformée en ondelettes on peut extraire des attributs de différents types et à différents niveaux de résolution. L'image d'approximation donne des informations sur les régions qui composent l'image, d'une résolution fine à une résolution grossière. Les images de
détails donnent des informations horizontales, verticales et diagonales sur l'image [13]. Donc les ondelettes permettent de caractériser la texture en décrivant les primitives et les règles d'arrangement qui les relient.


\subsubsection{Les filtres de Gabor}

Les filtres de Gabor sont largement utilisés en indexation, pour la description de la texture.
Introduit par Gabor [Gabo46], ces filtres ont été largement utilisés [Chaw16] à la fois comme fonctions de décomposition en ondelettes et comme outils d'analyse texturale [Anda10]. Ces filtres peuvent prendre en compte l'orientation, l'échelle et la localisation des frontières, qui peuvent être
utilisés pour caractériser la texture. Les filtres de Gabor sont des filtres passe bande, leur forme générale résulte de la multiplication d’une fonction de forme d'enveloppe gaussienne avec une fonction sinusoïdale complexe.

Les filtres de Gabor sont largement utilisés aujourd’hui pour modéliser la réponse du système visuel humain. En effet, ce dernier décompose les images texturées en un nombre important d'images filtrées dont chacune contient les variations d'intensité à travers une bande de fréquence et une orientation bien déterminées [Partio 02]. En effet, Marcelaje [Marcelaje 80] a montré que les cellules du cortex humain pouvaient être modélisées par des fonctions de Gabor à une dimension. Cette décomposition a été utilisée par Manjunath [Manjunathi 96] pour des indexations par les textures.

Les filtres de Gabor permettent une bonne résolution spatiale à haute fréquence et une bonne résolution harmonique sans grande précision spatiale à basse fréquence [Merc01]. Sommairement, les paramètres de texture sont déterminés en calculant la moyenne et l’écart type des niveaux de gris de l’image filtrée par le filtre de Gabor. En fait, ce n’est pas une seule valeur de moyenne et d’écart type qui sera calculée, mais plutôt un ensemble de valeurs égal au nombre d’échelles multiplié par le nombre
d’orientations utilisées. Nous aurons donc ce qui est parfois appelé la banque de filtre de Gabor. Mathématiquement, toutes les valeurs des moyennes et d’écarts type calculées seront regroupées dans un seul vecteur descripteur. \\

Un filtre de Gabor 2D est le produit d'une gaussienne elliptique dans toute rotation et un exponentiel complexe représentant une onde plane sinusoïdale. Nous rappelons que, dans le domaine spatial, la fonction de Gabor bidimensionnelle est une somme de deux fonctions sinusoïdales, l'une paire et réelle, l'autre impaire et imaginaire, modulée par une enveloppe gaussienne. Une fonction Gabor 2D $g(x, y)$ et sa transformée de Fourier $G(u, v)$ s'écrivent comme suit [Manj96]:


 
\begin{center}
	 \begin{equation}
	
	$ g(x, y) = \frac{1}{2\pi \sigma_x \sigma_y} \exp\left[-\frac{1}{2} (\frac{x^2}{\sigma_x^2} + \frac{y^2}{\sigma_y^2}) + 2\pi j W_x\right]$
	\end{equation}
\end{center}

  \begin{center}
 	\begin{equation}
 	
 	$ g_{mn}(x, y) = a^{-m} \exp\left[-\frac{1}{2} (\frac{(u-W)^2}{\sigma_u^2} + \frac{v^2}{\sigma_v^2})\right] $
 	\end{equation}
 \end{center}

 avec : $\sigma_u = \frac{1}{2\pi \sigma_x } $ et  $ \sigma_v = \frac{1}{2\pi  \sigma_y} $. \\
 
 Un ensemble de fonctions similaires peut être générées à partir de la dilatation et la rotation de la fonction Gabor $g(x, y)$ :
 
 \begin{center}
 	\begin{equation}
 	
 	$ G(u, v) = a^{-m} g(x', y')
 	\end{equation}
 \end{center}
 
 où : $a \ge 1 $, $x' = a^{-m} ( x \cos(\theta) + y \sin(\theta) )$,  $y' = a^{-m} ( -x \cos(\theta) + y \sin(\theta) )$ , $\theta = \frac{n\pi}{K}$ et $m,n$ sont des entiers et indiquent l'échelle et l’orientation des ondelettes respectivement avec $m = 0,1,..., M-1 $,  $n=0,1,..., N-1$, et $K$ est le nombre d'orientations. Le facteur d’échelle $ a^{-m} $ vise à assurer que l'énergie est indépendante de $m$. Les paramètres $W$ et $\theta$ représentent la fréquence et l'orientation du signal sinusoïdal et constituent le paramètre de l’espace du filtre de Gabor.\\

\begin{figure}[H]
	\label{fig:gaborFig}
	\centering
	\includegraphics[width=0.65\textwidth]{Figures/gaborFig} % Include the image .png
	
	\caption{Contour à mi-niveau de la réponse en fréquence d’un filtre de Gabor 2D de fréquence centrale $W_0$ et d’orientation $\theta$.}
	
\end{figure}

L'utilisation des filtres de Gabor permet d'extraire de l'image considérée des informations pertinentes, à la fois en espace et en fréquence. Ils peuvent capturer le spectre de fréquence d'une image, en amplitude zt en phase. Ces filtres sont toujours utilisés par bancs dans lesquels chacun des filtres est réglé à une orientation et une fréquence précise. Les recherches conduites dans la littérature montrent que les fonctions de Gabor simulent de manière convenable le système visuel humain en reconnaissance des textures; le système visuel étant considéré comme un ensemble de canaux de filtrage dans le domaine fréquentiel.
 \begin{figure}[H]
 	\label{fig:gabor49}
 	\centering
 	\includegraphics[width=0.65\textwidth]{Figures/gabor49} % Include the image .png
 	
 	\caption{Exemple du module des filtres de Gabor dans le domaine spatial.}
 	
 \end{figure}
 
 
\subsubsection{Conclusion}

Les attributs texturaux sont des attributs très importants pour la description de l'image et la
reconnaissance des objets, cependant elles ne suffisent pas pour une bonne représentation du
contenu de l'image, un autre attribut essentiel est la forme.
Dans la suite nous allons introduire cet attribut et les différentes approches utilisées pour
l'extraire.


Une grande partie de la recherche sur la description des textures s'est concentrée sur les techniques multi-échelles, notamment la transformée en ondelettes [Laws 80], [Mallat 99], [Strang 97]. Cette description a été avancée grâce à l'exploitation de modèles statistiques multi-échelles markoviens. Un framework multirésolution est le cadre le plus naturel pour la description de la texture. Une technique multi-échelles commune est la Transformée en Ondelettes Discrètes (DWT).

Les textures se produisent dans les régions d'une image, et doivent être décrites principalement en analysant l'information sur l'intensité à la résolution appropriée. Toutes les textures n'apparaissent pas sur les mêmes gammes régionales, c'est pourquoi une approche qui varie la taille de la région d'intérêt est souhaitable. Les techniques multi-échelles fournissent une grande capacité de la description de la texture.

Les structures markoviennes ont également été utilisées avec succès pour l'identification et la modélisation des textures [Choi 01], [Crousse 98].

\subsection{Descripteur de forme}
Si l'être humain est particulièrement sensible à l'attribut de couleur pour distinguer les
objets, pour certains types d'ambiguïté cela n'est pas suffisant et l'on a besoin de l'attribut de forme. Au même titre que pour la texture, l’information de forme également appelée en anglais « Shape » est complémentaire de celle de la couleur. La forme est généralement une description très riche d’un objet. Elle permet de détecter un objet sur une image plutôt que l’image elle-même.\\

La forme est un descripteur très important dans l'indexation des images. Elle est utilisée pour décrire la structure géométrique générique du contenu visuel. Généralement, les descripteurs de formes se divisent en deux catégories: les méthodes régions s'appuient sur la forme entière et les autres sur le contour. On peut ensuite distinguer deux familles pour chacune de ces catégories, une famille qui considère les objets comme une seule partie, et celle qui décrit les objets en les considérant comme un assemblage de sous parties. Une segmentation des objets est généralement utilisée avant l’extraction des attributs.

\begin{figure}[H]
	\label{fig:forme}
	\centering
	\includegraphics[width=0.65\textwidth]{Figures/forme} % Include the image .png
	
	\caption{Classification des méthodes d'extraction de forme.}
	
\end{figure}

Nous présentons dans ce qui suit quelques méthodes de description de la forme :

\subsection{Descripteurs de Fourier}
Les descripteurs de Fourier (DFs) font partie des descripteurs les plus populaires pour les applications de reconnaissance de forme et de recherche d’images. Ils ont souvent été utilisés par leur
simplicité et leurs bonnes performances en terme de reconnaissance [Zhan05] et facilitent l’étape
d’appariement. De plus, ils permettent de décrire la forme de l’objet à différents niveaux de détails.
Les DFs sont calculés à partir du contour des objets. Leur principe est de représenter le contour de
l’objet par un signal 1D, puis de le décomposer en séries de Fourier. Les DFs sont généralement
connus comme une famille de descripteurs car ils dépendent de la façon dont sont représentés les
objets sous forme de signaux.

\subsubsection{Moments géométriques}
Les moments géométriques [Sonk99] permettent de décrire une forme à l’aide de propriétés
statistiques. Ils représentent les propriétés spatiales de la distribution des pixels dans l’image.
La formule générale des moments géométriques est donnée par la relation suivante :

\begin{equation}
	$m_{p,q} = \sum_{x=0}^{M-1}\sum_{y=0}^{N-1} x^p y^q f(x, y) $   
	
\end{equation}

avec $p+q$ est l'ordre du moment. Le moment d’ordre $0 (m_{0,0})$  représente l’aire de la forme de l'objet, $f(x, y)$ : luminance du pixel (x, y), c’est-à-dire noir ou blanc en binaire.

Les moments géométriques sont très robustes et peu sensibles au bruit, préservation de l’information, facilement calculés et implémentés. Par contre, cette approche a des inconvénients comme la difficulté à corréler les moments avec la forme en elle-même, très sensible aux déformations et le temps de calcul de ces moments est très long.

\subsubsection{Moments orthogonaux}
Par opposition aux moments géométriques qui sont définis par rapport à une base quelconque $(x^p, y^q)$, les moments orthogonaux, comme leur nom l’indique, sont définis dans une base orthogonale, ce qui évite la redondance des informations portées par chacun des moments. Les deux types de moments orthogonaux les plus utilisés sont : les moments de Legendre et les moments de Zernik.

\subsubsection{Descripteurs de Fourier}
Les descripteurs de Fourier (DFs) font partie des descripteurs les plus populaires pour les applications de reconnaissance de forme et de recherche d’images. Ils ont souvent été utilisés par leur simplicité et leurs bonnes performances en terme de reconnaissance [Zhan05] et facilitent l'étape d'appariement. De plus, ils permettent de décrire la forme de l’objet à différents niveaux de détails.

Les DFs sont calculés à partir du contour des objets. Leur principe est de représenter le contour de l'objet par un signal 1D, puis de le décomposer en séries de Fourier. Les DFs sont généralement connus comme une famille de descripteurs car ils dépendent de la façon dont sont représentés les objets sous forme de signaux.

\subsubsection{Transformation ART}
Angular Radial Transform (ART) est un descripteur de forme robuste au changement d’échelle, à la translation et à la rotation et utilisé dans plusieurs travaux [AitZ14], [Chaw16]. Ce descripteur consiste à projeter l’objet à étudier sur une série de fonctions de base [Kim99]. Il a été
adopté par la norme MPEG-7. Il permet de décrire la forme d’une région de manière compacte et efficace, à partir de la distribution 2D des pixels de cette région. Formellement, ART est une transformation complexe unitaire définie sur un disque unité, qui consiste en une projection de
l’image sur des fonctions de bases, dans un repère polaire. De plus ART est une méthode basée sur les régions et traite la forme dans son ensemble et utilise l’information de tous les pixels de la région. Cette méthode mesure la distribution des pixels de la région et est peu affectée par le bruit et par les variations de formes.

Pour ces différentes raisons, nous nous sommes principalement intéressés aux méthodes d’analyses de formes basées régions, et plus précisément à la transformation ART dans notre travail.

%
% Les descripteurs basés sur le contour incluent les descripteurs de Fourier [Rui 98], [Zhang 05], et les chaînes de Freeman, qui ont été largement utilisés.
% 
%  Les descripteurs basés sur la région prennent en compte tous les pixels de la forme et les méthodes les plus courantes sont basées, sur la théorie des moments [Prokop 92], [Hu 62], [Jiang 91], [Taubin 92].
%   
%   Les moments invariants offrent une description robuste aux transformations affines, propriété appréciable pour les systèmes CBIR.
%
%Quelle que soit la méthode utilisée, il demeure important que certaines propriétés d'invariance soient vérifiées, notamment pour la translation, le changement d’échelle et la rotation, du fait que l’être humain corrige instinctivement les effets de ces transformations lors de la recherche d’un objet. Dans certaines applications, la robustesse à l’occultation peut également être souhaitée.
%
%Pour rechercher des formes similaires, des mesures de similarité entre des descripteurs sont définies pour déterminer la distance entre eux. Cependant, la comparaison n’est pas toujours simple et elle exige parfois des transformations. De plus, la dimension du descripteur est souvent élevée, ce qui augmente la complexité quand on cherche des objets similaires dans une grande collection.






\section{Mesure de similarité entre attributs}
La mesure de similarité est une étape fondamentale dans la recherche
d’images par le contenu. Elle compte sur la mesure de la ressemblance
visuelle entre une image requête et les images de la base. 

Les performances d'un CBIR dépendent largement de la mesure de similarité utilisée pour la comparaison des descripteurs des images. La mesure de similarité est aussi dépend des critères de la recherche mais également de la représentation des caractéristiques.

La mesure de similarité quantifiée la proximité des images dans l'espace des caractéristiques. Elle est souvent métrique, les images sont considérées ressemblantes si la distance est faible. La complexité de calcul d'une distance doit être raisonnable par ce que dans un système CBIR cette tâche s'exécute en temps réel ou pseudo réel. D'autres paramètres entrent en jeu telle la dimension de l'espace caractéristique, la taille de la base... La méthode naïve de recherche calcule la distance entre la requête et toutes les images de la base puis les ordonne selon leur score. Ceci par conséquent rend le temps de réponse proportionnel au
nombre d'images (O(N)). Les méthodes d'indexation du contenu permettent par ailleurs de réduire cette complexité comparée à la recherche séquentielle. Pour résumer, la mesure de similarité vérifie généralement les propriétés :

\begin{itemize}
	\item  \textbf{La perception} : Une faible distance dans l'espace de caractéristique indique deux images semblables.
	
	\item \textbf{Le calcul} : La mesure de distance se calcule rapidement pour une faible latence.
	
	\item \textbf{La scalabilité }: Le calcul de distance ne doit pas être affecté par une modification de taille de la base.
	
	\item \textbf{La robustesse} : La mesure devra être robuste aux changements des conditions d'acquisition d'image.
\end{itemize}


En mathématiques, on appelle distance sur un ensemble $E$ une application $d$ définie sur le produit $E^2 = E \times E$ et à valeurs dans l'ensemble $R^+$ des réels positifs:

\begin{figure}[H]
	\centering
	\includegraphics[width=0.2\textwidth]{Figures/distanceApp.png} % Include the image .png
\end{figure}

vérifiant les propriétés suivantes :

\begin{figure}[H]
	\centering
	\includegraphics[width=0.8\textwidth]{Figures/distanceProp.png} % Include the image .png
\end{figure}

Un ensemble muni d'une distance est un espace métrique.\\


Nous citons ci-dessous les distances les plus utilisées pour comparer des images considérées comme vecteurs ou comme distributions statistiques.

\subsection{Distances de Minkowski}
La distance de Minkowski est une famille de distances vectorielles. Soit $I_1$ , $I_2$ deux vecteurs de caractéristiques, elle s'exprime par :

\begin{equation}
       $d_p(I_1, I_2) = \sum_{i=1}^{N} \sqrt[p]{\left|{I}_{1}(i)-{I}_{2}(i)\right|^p} $ 
\end{equation}

Où $p\geq 1$ est le facteur de Minkowski et $N$ la dimension de l’espace caractéristique. Les métriques de Minkowski sont simples d’utilisation, rapides à calculer, simples à implémenter, et représentent un bon compromis entre efficacité et performance, par contre leur calcul est réalisé en considérant que chaque composante du vecteur apporte la même contribution à la distance. Pour cette famille de distances, plus le paramètre $p$ augmente, plus la distance $d_p$ aura tendance à favoriser les grandes différences entre coordonnées.

On distingue:
\begin{table}[H]
	\begin{tabular}{|c|c|c|}
		\hline
		\textbf{Distance} & \textbf{Caractéristiques}\\
		\hline
		\makecell{Manhatttan : \\ $d_1(I_1, I_2) = \sum_{i=1}^{N} \left|{I}_{1}(i)-{I}_{2}(i)\right| $ } & \makecell{Cette distance est plus convenable pour mesurer \\ la similarité entre les données multivariées, elle est\\ moins sensible au bruit coloré que \\la distance euclidienne. }   \\
		\hline
		
		
		\makecell{Euclidienne :\\ $d_2(I_1, I_2) = \sum_{i=1}^{N} \sqrt{\left|{I}_{1}(i)-{I}_{2}(i)\right|^2} $}  & \makecell{C’est la distance la plus fréquemment utilisée grâce \\à ses propriétés géométriques intéressantes. Cette distance \\à tendance à donner plus d’importance aux plus grandes \\différences sur une variable simple. Le contour de la\\ même distance (Euclidienne) à partir d’un point donnée\\ à une forme sphérique (cercle en deux dimensions).}   \\
		\hline
		
		\makecell{Chebychev :\\ $d_{\infty}(I_1, I_2) = \max_{i=1}^N  \left|{I}_{1}(i)-{I}_{2}(i)\right|} $}  & \makecell{La distance de Chebychev est adaptée aux données\\ de grande dimension, elle est souvent employée dans \\les applications où la vitesse d’exécution est importante.\\ Cette distance examine la différence absolue entre les différents\\ pairs des vecteurs, elle est considérée comme une approximation\\ de la distance Euclidienne mais avec moins de calcul.} \\   
		\hline
		
	
	\end{tabular}
	\caption{Les distances de Minkowski}
\end{table}

\subsection{Distance quadratique}
La distance de Minkowski traite les éléments du vecteur de
caractéristique d’une manière équitable. La distance quadratique en
revanche favorise les éléments les plus ressemblants. Les propriétés de cette distance la rendraient proche de la perception humaine de la couleur, ce qui en fait une métrique attractive pour les systèmes de Recherche d’images couleur par le contenu [16]. Sa formule générale est donnée par :

\begin{figure}[H]
	\centering
	\begin{equation}
	$d_q(I_1, I_2) = \sqrt{(I_1 - I_2)^T A (I_1 - I_2)} $
	\end{equation}
\end{figure}

Ou $A = (a_{ij}) $est la matrice de similarité. $a_{ij}$ représente la distance entre deux éléments des vecteurs $I_1$ et $I_2$, elle est définie par :

\begin{figure}[H]
	\centering
	\begin{equation}
	$a_{ij} = 1 - \frac{a_{ij}}{\max a_{ij}}$
	\end{equation}
\end{figure}

\subsection{Distance de Swain}
Cette mesure est l'une des premières distances utilisée dans la recherche d'image par le contenu, si les images d’une base des images sont indexées par des histogrammes, les distances géométriques s’appliquent. Cependant, il est possible de définir des mesures de similarité propres à cette représentation. Ainsi, l’intersection d’histogrammes est l’une des premières distances utilisée dans les systèmes CBIR [Swain91].Elle permet de comparer deux histogrammes normalisés $H_1$ et $H_2$ . La distance de Swain s’exprime ainsi :

\begin{figure}[H]
	\centering
	\begin{equation}
	$d_2(H_1, H_2) = 1- \frac{\sum_{i=1}^{N} \min(H_1(i),  H_2(i))}{\sum_{i=1}^{N}  H_2(i)}  $
	\end{equation}
\end{figure}

Il existe d’autres distances dans la littérature qui ne sont pas abordées dans ce mémoire.
\subsection{Conclusion}
Ce chapitre a fait l’état de l’art sans exhaustivité des différents descripteurs des attributs visuels pouvant être utilisés pour la recherche d’images par le contenu ainsi que les approches
correspondantes. Aussi, nous avons dressé une liste des types de descripteurs et les mesures de similarités avec leurs avantages et leurs inconvénients. Le chapitre suivant se focalisera sur notre solution détaillée, le schéma de CBIR, les techniques d’extraction de descripteur. Le choix d’un meilleur descripteur et d’une mesure de similarité promet une bonne pertinence d’un système CBIR. 
% Chapter Template

\chapter{Apport de la structure d’index M-Tree dans un CBIR} % Main chapter title

\label{Chapter3} % Change X to a consecutive number; for referencing this chapter elsewhere, use \ref{ChapterX}

%----------------------------------------------------------------------------------------
%	SECTION 1
%----------------------------------------------------------------------------------------

\section{Introduction}
Dans les systèmes de recherche d'images par le contenu le temps d'exécution d'une requête est d'une importance primordiale. Pour répondre aux besoins d'un CBIR d'être robuste en temps du calcul lors de l'étape d'indexation et l'étapes de recherche de nombreuse méthodes d'indexation ont vu le jour. Parmi ces méthode ou structures on trouve la structure M-tree, une structure recherche par similarité dans les espaces métriques, que nous avons choisi pour la partie d'accélération de recherche dans notre sujet. \\
 
La discussion du sujet de recherche par similarité dans les espaces métriques soulève plusieurs questions. Tout d’abord, les premières questions qui se posent sont des questions sur les espaces métriques comme : quelles sont les propriétés et les caractéristiques des espaces
métriques ? Ensuite, quelles sont les fonctions de similarité qui répondent aux critères des distances métriques? Quels sont les types des requêtes souvent utilisées pour chercher dans les espaces métriques?\\

Enfin, la discussion de recherche par similarité dans les espaces métriques conduit forcément à poser des questions sur les méthodes métriques d’indexation et de recherche, quelles sont les méthodes principales de recherche par similarité dans les espaces métriques
qui sont proposées dans la littérature, et quels sont les avantages et les inconvénients de chaque méthode?

Ce chapitre vise à discuter ce sujet afin de surligner les déférents concepts y liés.
\section{L’espace métrique}
Etant donné un ensemble E des objets, chaque fonction $d: E\times E\rightarrow R^+$ qui satisfait les trois propriétés suivantes est appelée une distance:

\begin{equation}
	\begin{array}{cc}
	(p1) \textbf{La positivité} : & \forall (x,y)\in E^2, d(x,y) \ge 0 \\
	(p2) \textbf{La symétrie} : & \forall (x,y)\in E^2, d(x,y) = d(y,x) \\
	(p3) \textbf{Réflexivité} : & \forall (x,y)\in E^2, d(x,y) = 0 \Rightarrow x = y
	\end{array}
\end{equation}

Une distance métrique est une distance qui satisfait les trois propriétés ci-dessus plus une quatrième propriété. Cette propriété est connue sous le nom « l’inégalité triangulaire », elle est définie formellement ci-dessous:

\begin{equation}
\begin{array}{cc}
(p4)\textbf{ L’inégalité triangulaire :}  & \forall x,y,z\in E, d(x,y)+d(y,z) \geq d(x,z)
\end{array}
\end{equation}

\begin{description}
	\item[Définition 1:] Un espace Métrique $ (E,d) $ est un ensemble $ E $ muni d’une distance métrique $ d $.
\end{description}

Dans le paradigme de recherche par similarité, la distance entre les objets représente le degré de ressemblance entre ses objets, c.à.d si la distance entre deux objets est petite, cela signifie que ces deux objets sont fortement similaires, et par contre, si la distance est assez grande, les objets sont assez dissimilaires.

\subsection{Les requêtes de recherche par similarité dans les espaces métriques}
La recherche d’informations dans une base de données peut se faire par plusieurs manières. Dans la suite de cette partie nous présentons deux types intéressants de requêtes largement utilisées dans la recherche par similarité dans les espaces métriques, il s’agit de la requête intervalle et la requête des k plus proches voisins.

\subsubsection{La requête intervalle (Range Query R(q,r))}
Ce type consiste à sélectionner dans la base tous les objets dont la distance par rapport à l’objet de la requête $ q $ est inférieure au rayon de recherche $ r $. Formellement, on peut définir la requête intervalle de la manière suivante.\\

\textbf{Définition 2:} Soit $ (E,d ) $ un espace métrique, $ q $ un objet requête et $ r $ est un réel  positif. La requête par intervalle $ R(q,r) $ retourne l’ensemble défini par :
\begin{equation}
	R = \left\{ o\in E / d(o, q) \leq r \right\}
\end{equation}

\begin{figure}[H]
	\centering
	\includegraphics[width=0.65\textwidth]{Figures/rangeQ} % Include the image .png
	\caption{Exemple de la requête intervalle.}
\end{figure}
La Figure 3.1 représente un exemple d’une requête par intervalle dans un espace de
deux dimensions, l’exécution de la requête $ R(q,r) $ retourne le sous-ensemble qui contient les objets $ {O_3,O_4,O_5,O_6,O_7} $.

\subsubsection{La requête des k plus proches voisins kpp (k-NN R(q,k))}
Cette requête consiste à retourner les $ k $ objets les plus similaires à l’objet de la requête $ q $. Formellement, la recherche des $ k $ plus proches voisins est définie comme suite :

\textbf{Définition 3:} Soit $ (E,d ) $ un espace métrique, $ q $ un objet requête, et $ k \in \mathbb{N} $. Alors, la requête  $ R(q,k) $ des $ k $ plus proches voisins $ k $-ppv retourne un sous-ensemble R des objets telle que

\begin{equation}
R = \left\{ o_1,...,o_k \in E / \forall o \in E, d(o, q) \ge \max_{1\leq i \leq k}\left\{d(o_i, q)\right\}  \right\}
\end{equation}

\begin{figure}[H]
	\centering
	\includegraphics[width=0.65\textwidth]{Figures/knnQ} % Include the image .png
	\caption{Exemple de la requête des k plus proches voisins.}
\end{figure}
La Figure 3.2 représente un exemple d’une requête des k plus proches voisins, l’exécution d’une requête de trois plus proches voisins de la requête q retourne les trois objets $ O_4 $, $ O_5 $ et $ O_6 $.

Nous avons effectuer une expérience pour comparer entre la performance de recherche linéaire et de recherche des k-NN de M-Tree. Nous avons effectuer la recherche dans une liste triée allant de 0 à 2001 avec des valeurs différentes à chaque test; n = 10, 100, 200, ..., 2000. la recherche linéaire retourne une seul valeur contre k=6 valeurs retourné par k-NN. Par ce fait, nous observons l'intérêts des structures d'index dans la figure 3.3 qui montre que M-tree accélère la recherche.
\begin{figure}[H]
	\centering
	\includegraphics[width=0.65\textwidth]{Figures/mtreeVSlinear.png} % Include the image .png
	\caption{Comapraisons entre la recherche linéaire et la recherche des k-NN M-tree.
	}
\end{figure}


\subsection{Les fonctions métriques}
Les fonctions métriques sont des fonctions qui permettent de mesurer la distance entre les objets d’un espace métrique. Le choix de la distance métrique convenable de comparaison se fait selon le type des données traitées, en effet, chaque type d’espace métrique à des distances métriques appropriées qui sont spécifiées par les experts. Généralement, les distances métriques sont classifiées en deux catégories [Chávez01]: les distances discrètes et les distances continues. Le choix de distance influence directement sur l’efficacité de la structure utilisée, et le choix de type de distance adéquate, pour construire une telle structure, dépend fortement de la base traitée [Hanif18]. Dans ce qui suit, un aperçu général sur quelques exemples de fonctions métriques continues et des fonctions métriques discrètes.

\subsubsection{Distances Continues}
Les distances continues sont des fonctions métriques qui peuvent retourner un ensemble infini des valeurs. Généralement, elles sont calculées en se basant sur les coefficients des objets. Dans ce qui suit  on donne l'exemples le plus connue des distances continues, La famille des distances de Minkowski Lp est un ensemble des distances continues, elles sont définies formellement par:
\begin{equation}
L_p(I_1, I_2) = \sqrt[p]{\sum_{i=1}^{N}  \left|{I}_{1}(i)-{I}_{2}(i)\right|^p} 
\end{equation}

\begin{itemize}
	\item Manhatttan :  $  L_1(I_1, I_2) = \sum_{i=1}^{N} \left|{I}_{1}(i)-{I}_{2}(i)\right|  $ 
	
	\item Euclidienne : $ L_2(I_1, I_2) =  \sqrt{\sum_{i=1}^{N} \left|{I}_{1}(i)-{I}_{2}(i)\right|^2} $
	
	\item Chebychev : 
	$L_{\infty}(I_1, I_2)=\max_{i=1}^N \left|{I}_{1}(i)-{I}_{2}(i)\right|$  
\end{itemize}

\begin{figure}[H]
	\centering
	\includegraphics[width=0.5\textwidth]{Figures/mink} % Include the image .png
	\caption{Exemples des distances de la famille Lp:\\ (a) $ L_1 $ (b) $ L_2 $ (c) 	$L_{\infty}$}
\end{figure}

\subsubsection{Distances Discrètes}
Les distances discrètes sont des distances métriques qui ne retournent qu’un nombre limité des valeurs. Généralement, elles sont calculées en se basant sur la comparaison des coefficients des objets pour retourner le nombre des objets qui sont identiques ou le nombre des coefficients qui se diffèrent. Dans la suite de cette section, on présente deux exemples des distances discrètes.

\paragraph{La distance Edit:}
La fonction edit [Levenstein65] est une métrique pour mesurer la distance entre les chaines de caractère. Elle représente le nombre minimal des opérations suffisantes pour convertir une chaine en une autre. Par exemple, on considère deux chaines de caractère $x=x_1x_2...x_n$ et $y=y_1y_2...y_m$, la distance $edit(x,y)$ entre l’objet $x$ et $y$ c’est le nombre des opérations pour convertir la chaine $x$ à $y$. Les opérations qui peuvent être utilisées sur une chaine de caractère sont définies comme suit :
\begin{itemize}
	\item Insérer : pour insérer un caractère dans une position donnée de la chaine
	\item Supprimer : pour supprimer un caractère de la chaine
	\item Remplacer : pour remplacer un caractère d’une position donnée dans la chaine par un autre caractère
\end{itemize}
Edit distance est une distance discrète souvent utilisée pour mesurer la similarité des documents texte et pour manipuler des données scientifiques comme les ADN et les protéines.

\paragraph{La distance de Hamming:}
La distance de Hamming est une distance discrète proposée en 1950 [Hamming50], elle sert à calculer la distance entre les vecteurs. La distance Hamming entre les vecteurs A et B représente le nombre des coefficients qui se diffèrent, si la distance est égale à 0, ça signifie que les vecteurs A et B sont identiques. Formellement, on peut définir la distance de Hamming comme suite:

\begin{equation}
d_h((x_1,...,x_n), (y_1,...,y_n)) = card(\left\{ i: x_i \neq y_i, 1\leq i \leq n\right\})
\end{equation}
Par exemple, la distance de Hamming entre les deux vecteurs $ (1,2,1,3) $ et $ (0,2,3,1) $, et la distance de Hamming entre les deux chaines de caractères « BFEC » et « BFAC » sont calculées comme suit :

\begin{tabular}{cc}
	 $ d_h ((1,2,1,3), (0,2,3,1))= Card({1,3,4}) = 3 $ &  $d_h(BFEC , BFAC)= Card({3}) = 1 $ \\
	 \\
\end{tabular}

L’utilisation de la distance de Hamming est très large [Deza09], elle peut être utilisée pour mesurer de similarité de déférent type des donnés comme images, DNA, Protéine et Textes.

\section{Les données métriques}
La plupart des données acquises aujourd’hui peuvent être modélisées  des objets dans un espace métrique muni d’une distance métrique. Dans cette section, nous présentons des exemples des types de bases de données qui peuvent utiliser des distances métriques pour faire la recherche par similarité.

\subsection{Les images 2D}
L’extraction automatique des caractéristiques d'images directement à partir de leur contenu numérique est une solution parmi les solutions possibles pour gérer efficacement les bases de données d'images. Chaque image est représentée par un ou plusieurs vecteurs qui représentent leurs contenus visuels. Par conséquence, la similarité entre deux images peut être mesurée par la comparaison entre leurs descripteurs. La génération d’une base de descripteurs des images représente une phase nécessaire dans les systèmes de recherche d’image par contenu. La description de contenu d’image est souvent faite par des descripteurs de bas niveau, appelés aussi vecteurs caractéristiques, tels que la couleur, la texture et la forme.

\begin{figure}[H]
	\centering
	\includegraphics[width=0.55\textwidth]{Figures/horsligne.png} % Include the image .png
	\caption{Exemple de la représentation vectorielle d'une base d'image et d'une image requête.}
\end{figure}
La Figure 3.5 représente un exemple de l’extraction du contenu visuel des images sous forme des vecteurs ; A partir d’une base d’image on peut obtenir une base des vecteurs caractéristiques (indexes) qu’on peut comparer par des distances métriques.

Les distances de Monkowski sont beaucoup utilisées pour mesurer la similarité entre les descripteurs, notamment la distance Euclidienne. 

%\subsection{Séquences d’ADN}
%L'ADN contient les instructions pour toutes les fonctions d’une cellule d’un organisme, deux séquences d’ADN similaires de deux cellules signifient que les deux cellules ont des fonctions similaires. La compréhension de la relation entre une requête d’ADN avec des séquences génomiques, déjà traitées, analysées et stockées dans une base de données, sert à aider les biologistes de bien estimer les fonctionnements de la séquence requête avant de la traiter. La structure d’ADN consiste d’une chaine linéaire des quatre nucléotides qui sont souvent représentés par quatre Alphabets A, C, G et T. La Figure II-6 montre un exemple d’une
%représentation d’un gène humain [WilliamsZobel02].
%
%\begin{figure}[H]
%	\centering
%	\includegraphics[width=0.55\textwidth]{Figures/ADN} % Include the image .png
%	\caption{Une partie de gène de croissance épidermique humain.}
%\end{figure}
%
%Les séquences génomiques peuvent se modéliser comme un espace métrique muni d’une distance métrique comme la distance Edit et la distance de Hamming.

\section{Les méthodes d’accès Métriques}
L'objectif principal des méthodes d’accès métrique (MAM) est la minimisation du temps de recherche. Le calcul de distance entre les objets métriques est souvent coûteux et consomme beaucoup du temps. Par exemple, le calcul des distances de Minkowski consomme un temps de complexité linéaire, le calcul de la distance Edit entre deux chaines de taille m et n est d’ordre $ O(n\times m) $. De même, la complexité de la distance de Hamming est d’ordre $ O(n\times m) $,certaines distances usuelles peuvent atteindre une complexité d’ordre $ O(n^4)  $ comme la distance Edit-tree [Zezula06]. Donc, la complexité de comparaison d’objets dans une base est un critère crucial qui influence directement sur la performance de la structure métrique utilisée. La mesure de performance se calcule en fonction du nombre de distances calculées lors du parcours de structure pour répondre à une requête parce que la complexité des autres parties des algorithmes de recherche est négligeable devant la complexité de calcul de distance [Chávez01].\\

Le but de MAMs vise à minimiser le nombre de distances calculées lors du stockage et lors de la recherche par l’exploitation des distances pré-calculées et en évitant le parcours des régions inutiles sans faire des calculs supplémentaires. La Figure 3.6 illustre un exemple de cette approche dans les bases de données d’images [Chávez01].\\


La technique la plus simple pour éviter le calcul des distances est l’exploitation directe de l’inégalité triangulaire. Supposons que, les distances entre objet pivot $ p $ des objets $ o_i $ sont pré-calculées, la distance entre une requête $ q $ et les objets $ o_i $ peut être estimée sans besoin du calcul direct des distances, il suffit de calculer la distance $ d(q,p) $ entre la requête et le pivot, ensuite, toutes les distances $ d(q,o_i) $ sont majorées par $ d(q,p)+d(p,o_i) $, et avec un simple effort de calcul nous pouvons déduire que la distance $ d(q,o_i) $ est minorée par $ |d(q,p)-d(p,o_i)| $. En effet, si $ q $ présente un objet requête, $ p $ un objet pivot et $ o $ un objet de la base. Si la distance $ d(q,p) $ entre la requête et le pivot est déjà calculée et la distance entre l’objet $ o $ et le pivot est pré-calculée. On peut estimer la distance entre la requête et l’objet par la formule suivante 
\begin{equation}
    d(q,p)+d(p,o) \geq d(q,o) \geq |d(q,p)-d(p,o)|
\end{equation}

\begin{figure}[H]
	\centering
	\includegraphics[width=0.6\textwidth]{Figures/similarity.png} % Include the image .png
	\caption{Modèle d'indexation métrique et la recherche par similarité.}
\end{figure}

Lors de la recherche par une requête intervalle $ R(q,r) $, les bornes supérieures et inférieures peuvent être utilisées pour accélérer la recherche. Trois situations sont possibles : si la borne supérieure est inférieure à $  r ~~ (d(q,p)+d(p,o) \leq r ) $ alors il est sûr que l’objet $ o $ est pertinent car il vérifie $ d(q,o) \leq r $, si la borne inferieure dépasse le rayon $ r ( |d(q,p)-d(p,o)| > r )  $ alors il est sûr que \begin{displaymath}
	d(q,o) > r
\end{displaymath} et par conséquence l’objet $ o $ ne peut être parmi les résultats de la requête. La dernière situation est le pire des cas, il correspond au cas où la borne supérieure est supérieure à $ r $ et la borne inférieure est inférieure à  $ r $  $  (d(q,p)+d(p,o) \geq r\geq |d(q,p)-d(p,o)|) $, il est indispensable de calculer la distance $ d(q,o) $ pour déterminer si l’objet $ o_i $ est pertinent ou non.

\begin{figure}[H]
	\centering
	\includegraphics[width=0.6\textwidth]{Figures/pivot.png} % Include the image .png
	\caption{Filtrage par un pivot: (a) l'objet $  o_i $ fait partie du résultat (b) l'objet $ o_i $ sera éliminé (c) le calcul de la
		distance $ d(q,o_i) $ est obligatoire pour savoir si $ o_i $ est pertinent ou pas.}
\end{figure}

Les lignes solides représentent les distances connues, tandis que les lignes coupées représentent les distances non connues.\\

Pour une structure métrique qui utilise un groupe de pivots $ (p_1,p_2,…, p_n) $, les distances entre les objets $ o_i $ et les pivots sont calculées et stockées au préalable. Ainsi, lors d’une recherche par requête intervalle $ R(q,r) $, chaque objet de la base qui satisfait la condition  \begin{displaymath}
	\max_{i=1,..,n}{|d(q,pi)-d(o,pi)|} > r
\end{displaymath}  est certainement non pertinent et donc il sera écarté de la recherche sans aucun calcul de distance. Comme nous avons vu précédemment, le filtrage par n pivots est vu comme un « mapping » de l’espace métrique $ (E,d) $ vers un espace vectoriel $ (R^n, L_\infty) $ en lequel chaque élément $ o \in E $ est représenté par le vecteur $ (d(o,p_1), d(o,p_2),..., d(o,p_n)) $, tout objet $ o $ qui satisfait la condition $ L_\infty((d(o,p_1),..., d(o,p_n)),(d(q,p_1),..., d(q,p_n))) > r  $ sera ignoré durant la résolution d’une requête de rayon $ r $.

\begin{figure}[H]
	\centering
	\includegraphics[width=0.6\textwidth]{Figures/npivot.png} % Include the image .png
	\caption{(a) exemple de filtrage par un pivot et (b) par deux pivots.}
\end{figure}

L’utilisation directe des pivots nécessite le stockage des distances entres les pivots et les objets dans la mémoire. Par conséquence, cette technique est coûteuse en termes de la mémoire de stockage. Pour cette raison, d'autres techniques ont été développées pour traiter ce problème comme la technique d’utilisation de double pivot et la technique de regroupement des objets dans des Clusters [Hanif18].


\begin{figure}[H]
	\centering
	\includegraphics[width=0.6\textwidth]{Figures/2pivot.png} % Include the image .png
	\caption{L'utilisation de deux pivots pour partitionner les données, (a) représente le cas où l'objet et la requête sont dans la même partions, (b) représente le cas où l’objet est la requête situées dans des partitions différent.}
\end{figure}

La première technique est l’utilisation du double pivot qui sert à regrouper, en chaque niveau de la structure, les objets d’une manière récursive à partir de deux pivots $ p_1 $ et $ p_2 $. Les objets qui sont plus proche au pivot $ p_1 $ que le pivot $ p_2 $ seront regroupés dans la région $ S(p_1) $ du pivot $ p_1 $ et les autres objets seront regroupés ainsi dans la deuxième région $ S(p_2 $) du pivot $ p_1 $.\\

La deuxième technique sert à réduire le calcul des distances lors de stockage par l’utilisation des rayons de couverture d’un pivot au lieu de stocker les distances entre le pivot $ p $ et les objets d'une région, seulement deux valeurs $ r_{max} $ et $ r_{min} $, qui vérifie   \begin{displaymath}
	r_{min}  \le d(p,o) \le r_{max}
\end{displaymath}  sont mémorisées (voir la Figure 3.10 qui suit comme un exemple). \\
En remplaçant la distance $ d(p,o) $ par ses bornes et en utilisant la formule  \begin{displaymath}
	|d(q,p)-d(p,o)| \le d(q,o) \le d(q,p)+d(p,o)
\end{displaymath}  on peut facilement montrer que la distance $ d(q,o) $ est majorée par la valeur $ d(q,p) + r_{max} $ et qu’il est minorée par la valeur
$ \max \left\{d(q,p) - r_{max}, r_{min} - d(q,p), 0\right\} $,
avec:
\begin{center}
	$  ( \max \left\{ d(q,p) - r_{max} , r_{min} - d(q,p), 0\right\} \le d(q,o) \le d(q,p) + r_{max} ) $.
\end{center}
 Donc il est possible d’éviter le parcours des Clusters inutile lors de la recherche en se basant sur le pivot et les rayons de couvertures sans besoin de calcul de la distance $ d(q,o) $.
\begin{figure}[H]
	\centering
	\includegraphics[width=0.25\textwidth]{Figures/radius.png} % Include the image .png
	\caption{Utilisation un pivot est de rayons de couvertures.}
\end{figure}

Dans la littérature, l’indexation de base de données par les méthodes d’accès métrique est conçue pour bénéficier d’une ou de plusieurs techniques discutées ci-dessus. On distinct entre deux approches : l’approche basée sur le partitionnement et l’approche basée sur les pivots (matrice des distances). Les méthodes basées sur les pivots construisent une matrice de distances entre les objets d’une région et les pivots afin de les utiliser pour filtrer les objets non pertinents lors de la recherche, tandis que, celle basées sur le partitionnement regroupent des objets similaires dans des régions, ce qui permet, lors de la recherche, d’éviter l’accès à des régions inutiles sans faire de calculs supplémentaires. Les taxonomie classiques souvent classifient les méthodes basées sur le partitionnent en deux catégories; le partitionnement par hyperplan et le partitionnement par boule [Chávez01]. Mais, plusieurs méthodes récentes ont dépassé cette classification soit par la proposition d’autres types de partitionnement ou par la combinaison de plusieurs types de partitionnement. \\

Plusieurs Méthodes d’accès métrique proposées en littérature sont basées sur l’idée de partitionnement des données. Elles peuvent se classifier sous quatre catégories; Les méthodes de partitionnement par boule, Les méthodes de partitionnement par hyperplan, Les méthodes de partitionnement par « Cut région » et les méthodes qui combinent plusieurs types de partitionnement [Hanif18].
\begin{figure}[H]
	\centering
	\includegraphics[width=0.8\textwidth]{Figures/class.png} % Include the image .png
	\caption{ Classification des méthodes d’accès métriques.}
\end{figure}
On s'intéresse principalement aux méthodes de partitionnement par boule.
Pour plus d'information sur les méthodes d'indexation, leurs avantages et inconvénients, le lecteur peut se référer à [Hanif18]. Ici on présente les avantages et inconvénients de la méthodes M-tree que nous avons choisi pour notre système:
\begin{table}[H]
	\centering
	\caption{Les avantages de M-tree.}
	\begin{tabular}{|c|c|c|}
		\hline
		\textbf{Avantages} & \textbf{Inconvenients}\\
		\hline
		\makecell{- Coût de recherche\\
			- Dynamique\\
			- Résistante à l’augmentation\\
			 de la dimension} 
		& \makecell{- Coût de recherche
			- Dynamique\\
			- Résistante à l’augmentation\\
			 de la dimension}   \\
		\hline
	
	\end{tabular}
	
\end{table}
Puisque notre objectif est de rechercher les images par similarité on a fait ce choix de faite que M-tree est efficace dans un tel système qui ne nécessite pas plusieurs opérations d'insertion et de suppression dans l'étape de la recherche.
\section{La méthode partitionnement par boule M-tree}
Les méthodes de partitionnement en boule servent à partitionner l'espace des données sous formes des boules, elles utilisent des pivots et des rayons de couverture pour diviser les données. Le nombre des pivots utilisés, le choix du type de couverture des pivots font la différence entre les méthodes de partitionnement par boule, dans cette section nous présentons la méthodes de partitionnement par boules M-tree.

\subsection{La structuration de M-tree}
La méthode M-tree [CiPaZe97] est une structure efficace dans les environnements dynamiques caractérisés par un fréquent besoin de supprimer et d'ajouter des éléments. Le processus de construction de l'arbre M-tree consiste à partitionner l'espace métrique par des sphères région en se basant sur un pivot et un rayon de couverture. \\

M-Tree organise les objets en nœuds de taille fixe, qui correspondent aux régions de l'espace métrique. Les nœuds de l'arbre peuvent stocker jusqu'à \textbf{$ M $} entrées (Les résultats présentées dans le dernier chapitre ont été obtenues en fixant $ M = 8 $), c'est la capacité des nœuds. Les feuilles d l’arbre différent dans leur structure avec les nœuds internes:
\begin{itemize}
	\item Pour chaque objet indexé, une entrée au format 
	\begin{figure}[H]
		\centering
		\includegraphics[width=0.6\textwidth]{entryleaf.png} % Include the image .png
	\end{figure} 
	est stockée dans un nœud feuille. Avec:
	\begin{itemize}
		\item \textbf{$ oid(O_j) $} est l'identifiant de l'objet qui réside dans un fichier de données séparé,
		\item \textbf{$ O_j $} sont les valeurs des caractéristiques de l'objet 
		\item et \textbf{$ d(O_j, P(O_j)) $} est la distance entre entre l’objet \textbf{$ O_j $} et l’objet  \textbf{$ P(O_j) $}, parent d'\textbf{$ O_j $}
	\end{itemize}

	\item Pour les nœuds internes, l'entrée stocké est au format
	\begin{figure}[H]
		\centering
		\includegraphics[width=0.7\textwidth]{entryInternal.png} % Include the image .png
	\end{figure} 
	Avec,
	\begin{itemize}
		\item \textbf{$ O_r $} est une valeur de caractéristique, également appelée un objet de routage,
		\item \textbf{$ r(O_r) $}$>0$ est un rayon de couverture, 
		\item \textbf{$ ptr(T( O_r)) $} est un pointeur à la racine du sous-arbre \textbf{$ T(O_r) $} - l'arbre couvrant d'\textbf{$ O_r $}, 
		\item et \textbf{$ d(O_r, P(O_r)) $} est la distance entre \textbf{$ O_r $} et \textbf{$ P(O_r) $}, l'objet parent d'\textbf{$ O_r $}\\
		
	\end{itemize}
\end{itemize}


La sémantique du rayon de recouvrement est exprimé par ce qui suit:
\begin{description}
	\item[Proprièté:] Le rayon de couverture d'un pivot (objet de routage), $ O_r $, satisfait l'inégalité $ d(O_j, O_r) \le r(O_r) $ pour chaque objet $ O_j $ stocké dans l'arbre de couverture de $ O_r $.
\end{description}

Un pivote définit donc une région dans l'espace métrique $ E $, centrée sur $ O_r $ et de rayon $ r(O_r) $, et $ O_r $ est le parent de chaque objet $ O_j $ stocké dans le nœud référencé par $ ptr(T(O_r)) $, c'est-à-dire $ O_r \equiv P(O_j)$ (voir figure 3.12 ).  Cela implique que le M-tree organise l'espace métrique en un ensemble de régions, qui peuvent se chevaucher, auxquelles le même principe est appliqué de manière récursive.

\begin{figure}[H]
	\centering
	\includegraphics[width=0.55\textwidth]{Figures/mtree.png} % Include the image .png
	\caption{ Un pivot, $ O_r $, avec un rayon de couverture, $  r(O_r) $, qui référence le pivot suivant (sous arbre), $ T(O_r) $.}
\end{figure}

Les feuilles de l’arbre contiennent les identifiants des objets et leurs distances aux pivots associés, les nœuds internes mémorisent plusieurs paramètres comme le pointeur vers le pivot suivant, le rayon de couverture des régions de pivot concernés et un pointeur qui pointe vers le pivot fils. Chaque sous arbre contient les objets qui sont à une distance de pivot $ p $ inferieur au rayon de couverture.\\

\subsection{Comment M-tree grandit?}

Comme tout autre arbre dynamique équilibré (ou balancé), Le processus de construction de M-tree (MT), qui se fait de bas en haut (un mode ascendant buttom-up), parcourt l'arbre à partir de la racine pour insérer le nouvel objet dans le sous arbre qui n'impose pas un changement de rayon de couverture. Si plusieurs sous arbre sont qualifiés, l'objet s'inséra dans le sous arbre de pivot le plus proche à l'objet inséré, sinon l'algorithme cherche le sous arbre qui peut avoir le minimum prolongement pour couvrir le nouvel objet.\\

M-tree est constitué d’un ensemble de nœuds de taille fixe $ M $, si l'insertion d'un objet produit un dépassement de capacité d'un nœud $ N $, le processus de construction fait appel à la fonction Split qui sert à créer un nouveau nœud $ N' $ de même niveau que le nœud saturé $ N  $ et d’appliquer un algorithme de redistribution de $ M+1 $ données entre l’ancien nœud et le nouveau nœud créer. Lorsque la racine se divise, une nouvelle racine est créée et l'arbre $ MT $ grandit d'un niveau. La méthode Split est décrite de manière concise comme suit :\\

\begin{figure}[H]
	\centering
	\includegraphics[width=0.95\textwidth]{Figures/split.png} % Include the image .png
\end{figure}

La méthode \textbf{Promote} choisit, en fonction de certains critères spécifiques, deux pivots, $ O_{p1} $ et $  O_{p2} $, à insérer dans le nœud parent, $ N_p $. La partition divise les $ (M + 1) $ entrées du nœud surchargé $ Ne $ en deux sous-ensembles disjoints, $ N_1 $ et $ N_2 $, qui sont ensuite stockés dans les nœuds $ N $ et $ N' $, respectivement. Une mise en œuvre spécifique de \textbf{Promote} et \textbf{Partition} définit ce que nous appelons une stratégie de fractionnement. Contrairement à d'autres modèles d'arbres métriques (statiques), chacun d'entre eux repose sur un critère spécifique d'organisation des objets, M-tree offre la possibilité de mettre en œuvre des stratégies de fractionnement alternatives, qui peuvent être ajustées en fonction des besoins de l'application \textbf{(voir section 3.5.5).}\\

Toute stratégie de fractionnement doit respecter la sémantique du rayon de couverture. Si le nœud du fractionnement, $ N $, est une feuille, cela est garanti par le paramétrage :

\begin{equation}
	r(O_{p1}) = \max \left\{d(O_j, O_{p1}) | O_j \in N_1 \right\}
\end{equation}

En général, le rayon de couverture d'un pivot pointant vers une feuille est égal à la distance maximale entre le pivot lui-même et les objets dans la feuille.\\

Lorsque la division implique un nœud interne, $ N $, chaque entrée $ O_j $ dans $ N_1 $ a un rayon de couverture non nul, $ r(O_j) $. En définissant:

\begin{equation}
r(O_{p1}) = \max \left\{d(O_j, O_{p1}) + r(O_j) | O_j \in N_1 \right\}
\end{equation}

Il est garanti, par la propriété d'inégalité triangulaire, qu'aucun objet dans $ T(O_{p1}) $ ne peut être distant de $  O_{p1} $ plus que $ r(O_{p1}) $. La figure 3.13 le montre dans le cas $ E = ( \mathbb{R^2}, L_2) $, c'est-à-dire le plan réel avec la distance euclidienne.
\begin{figure}[H]
	\centering
	\includegraphics[width=.6 \textwidth]{Figures/splitexep.png} % Include the image .png
	\caption{Fractionnement d'un noeud interne dans l'espace  $ ( \mathbb{R^2}, L_2) $}
\end{figure} 

\subsection{Les requêtes de similarité}
Les algorithmes M-tree pour le traitement des requêtes de similarité visent à réduire le nombre de nœuds accessibles ainsi que le nombre de  de distances calculées. Cela est particulièrement important lorsque la recherche s'avère être orienté CPU, ce qui peut être le cas pour les fonctions de calcul de distance intensives. À cette fin, toutes les distances (pré-calculées) stockées dans les nœuds des M-trees, c'est-à-dire $ d(O_r, P(O_r)) $ et $ r(O_r) $, sont utilisées.

\subsubsection{Les requêtes intervalle (Range Queries)}
La requête $ range(Q, r(Q)) $ sélectionne tous les objets $ O_j $ de la base de données tels que $ d(O_j, Q) \le r(Q) $. L'algorithme \textbf{RangeSearch} part du nœud racine et parcourt récursivement tous les chemins qui ne peuvent pas être exclus pour aboutir aux objets qualifiants.
\begin{figure}[H]
	\centering
	\includegraphics[width=.9 \textwidth]{Figures/rangesearxh.png} % Include the image .png
\end{figure} 

Comme, lors de l'accès au nœud $ N $, la distance entre $ Q $ et $ O_p $, l'objet parent de $ N $, a été déjà calculée, il est possible de se débarrasser d'un sous-arbre sans devoir calculer aucune nouvelle distance. La condition appliquée pour l'élagage (action de débarrasser) est la suivante.
\begin{description}
	\item[Lemme 1 :] Si $ d(O_r, Q) > r(Q) + r(O_r) $, alors, pour chaque objet $ O_j  $ dans $ T (O_r) $, c'est $ d(O_j, Q) > r(Q) $. Ainsi, $ T (O_r) $ peut être éliminé de la recherche en toute sécurité.
\end{description}
En fait, puisque (par l'inégalité triangulaire)
\begin{displaymath}
    d(O_j, Q) \ge d(O_r, Q) - d(O_j, O_r) 
\end{displaymath}  
et (par déf. du rayon de couverture)
\begin{displaymath}
	d(O_j, O_r) \le r(O_r)
\end{displaymath}
il s'agit de 
\begin{displaymath}
	 d(O_j, Q) \ge d(O_r, Q)-r(O_r).
\end{displaymath} 
Puisque, par hypothèse,  
\begin{displaymath}
	 d(O_r, Q) - r(O_r) > r(Q)
\end{displaymath}
le résultat suit.\\
 
Pour appliquer le Lemme 1, il faut calculer la distance $ d(O_r, Q) $. Cela peut être évité en tirant parti du résultat suivant.
\begin{description}
	\item[Lemme 2 :]  Si $ | d(O_p, Q) - d(O_r, Op) |> r(Q) + r(O_r) $, alors $ d(O_r, Q) > r(Q) + r(O_r) $.
\end{description}

C'est une conséquence directe de l'inégalité triangulaire, qui garantit que 
\begin{displaymath}
	 d(O_r, Q) \ge d(O_p, Q)-d(O_r, O_p) ~~ et  ~~  d(O_r, Q) \ge d(O_r, O_p)-d(O_p, Q)
\end{displaymath}
  sont tous les deux vrais (voir figure 3.14 en dessous). Le même principe d'optimisation est appliqué aux nœuds feuilles. Les résultats expérimentaux montrent que cette technique permet d'économiser jusqu'à 40\% de calculs de distances [CiPaZe97]. Le seul cas où les distances doivent nécessairement être calculées est celui du nœud racine, pour lequel l'$ O_p $ est indéfini.

\begin{figure}[H]
	\centering
	\includegraphics[width=.6 \textwidth]{Figures/lemme.png} % Include the image .png
	\caption{La figure montre comment le Lemme 2 est utilisé pour éviter de calculer les distances. \\Cas a) : $ d(O_r, Q) \ge d(O_p, Q) - d(O_r, O_p) > r(Q) + r(O_r) $ ; \\
		Cas b) : $ d(O_r, Q) \ge d(O_r, O_p) - d(O_p, Q) > r(Q) + r(O_r) $.}
\end{figure} 

\subsubsection{Les requêtes du k plus proche voisins (k-NN)}

L'algorithme de recherche k-NN (K Nearest Neighbors) récupère les k plus proches voisins d'un objet requête $ Q $ - on suppose qu'au moins $ k $ objets sont indexés par l'arbre M-tree. Nous utilisons une technique de branchement et de liaison(branch-and-bound), assez similaire à celle conçue pour les arbres R-tree [RKV95], qui utilise deux structures globales : une file d'attente prioritaire, \textbf{PR}, et un tableau de k éléments, \textbf{NN}, qui, à la fin de l'exécution, contiendra le résultat.\\

$ PR $ est une file d'attente de pointeurs vers des sous-arbres actifs, c'est-à-dire des sous-arbres où l'on peut trouver des objets qualifiants. Avec le pointeur vers (la racine du) sous-arbre $ T (O_r) $, une limite inférieure, $ d_{min}(T (O_r)) $, sur la distance de tout objet dans $ T (O_r) $ par rapport à $ Q $ est également conservée. La limite inférieure est

\begin{equation}
    d_{\min}(T(O_r)) = \max \left\{d(O_r, Q), 0\right\}
\end{equation}
puisqu'aucun objet dans $ T(O_r) $ ne peut avoir une distance de $  Q  $ inférieure à $ d(O_r, Q)-r(O_r) $.
Ces limites sont utilisées par la fonction \textit{ChooseNode} pour extraire de $ PR $ le noeud suivant à examiner. Comme le critère d'élagage de la recherche k-NN est dynamique - le rayon de recherche est la distance entre $ Q $ et son k-ième voisin le plus proche actuel - l'ordre dans lequel les nœuds sont visités peut affecter les performances. Le critère heuristique mis en œuvre par la fonction \textit{ChooseNode} consiste à sélectionner le nœud pour lequel la limite inférieure de $ d_{min} $ est minimale. (D'après les observations expérimentales, d'autres critères ne conduisent pas à une meilleure performance [CiPaZe97].).
\begin{figure}[H]
	\centering
	\includegraphics[width=.9 \textwidth]{Figures/choosenode.png} % Include the image .png
\end{figure} 


À la fin de l'exécution, la i-ième entrée du tableau $ NN $ aura la valeur  \begin{displaymath}
	NN[i] = [oid(O_j),d(O_j, Q)] 
\end{displaymath}
$ O_j $ étant le i-ième voisin le plus proche de $ Q $. La valeur de la distance dans la i-ième entrée est désignée par $ d_i $, de sorte que $ d_k $ est la plus grande valeur de distance dans $ NN $. Il est clair que $ d_k $ joue le rôle d'un rayon de recherche dynamique, puisque tout sous-arbre pour lequel $ d_{min}(T(O_r)) > d_k $ peut être élagué en toute sécurité.\\

Les entrées du tableau $ NN $ sont initialement fixées à  \begin{displaymath}
	NN[i] = [null ,\infty] (i= 1,..., k)
\end{displaymath} c'est-à-dire que les $ oid $ sont indéfinis et $ d_i = \infty $. Lorsque la recherche commence et que l'on accède aux nœuds (internes), l'idée est de calculer, pour chaque sous-arbre $ T(O_r) $, une limite supérieure, $ d_{max}(T(O_r)) $, sur la distance de tout objet dans $ T(O_r) $ par rapport à $ Q $. La limite supérieure est fixée à

\begin{equation}
d_{\max}(T(O_r)) =  d(O_r, Q)+r(O_r)
\end{equation}

Considérons le cas le plus simple $ k = 1 $, deux sous-arbres, $ T(O_{r1}) $ et $ T(O_{r2}) $, et supposons que $ dmax(T(O_{r1})) = 5 $ et $ dmin(T(O_{r2})) = 7 $. Puisque $ d_{max}(T(O_{r1})) $ garantit qu'un objet dont la distance de $  Q $ est au plus égale à 5 existe dans $ T(O_{r1}) $, $ T(O_{r2}) $ peut être élagué de la recherche. Les limites $ d_{max} $ sont insérées à des positions appropriées dans le tableau $ NN $, laissant juste le champ $ oid $ non défini. L'algorithme de recherche $ k-NN $ est donné ci-dessous.
\begin{figure}[H]
	\centering
	\includegraphics[width=.9 \textwidth]{Figures/knnsearch.png} % Include the image .png
\end{figure} 

La méthode \textit{k-NN-NodeSearch} implémente la majorité de la logique de recherche. Sur un nœud interne, elle détermine d'abord les sous-arbres actifs et les insère dans la file d'attente $ PR $. Ensuite, si nécessaire, elle appelle la fonction \textit{NN-Update} pour effectuer une insertion ordonnée dans le tableau $ NN $ et reçoit en retour un (possiblement nouvelle) valeur de $ d_k $. Cette valeur est ensuite utilisée pour retirer de $ PR $ tous les sous-arbres pour lesquels la limite inférieure de $ d_{min} $ dépasse $ d_{k} $. Des actions similaires sont effectuées dans les nœuds feuilles. Dans les deux cas l'optimisation visant à réduire le nombre de calculs de distance, qui utilise les distances pré-calculées de l'objet parent, est appliquée.

\begin{figure}[H]
	\centering
	\includegraphics[width=.9 \textwidth]{Figures/knnnodesearch.png} % Include the image .png
\end{figure} 

\subsection{Insertion d'objets}
L'algorithme d'insertion descend récursivement l'arbre MT pour localiser le nœud feuille le plus approprié pour accueillir un nouvel objet, $ O_n $, déclenchant éventuellement une division si la feuille est pleine. Le raisonnement de base pour déterminer le nœud feuille "le plus approprié" est de descendre, à chaque niveau de l'arbre, le long d'un sous-arbre, $ T(O_r) $, pour lequel aucun élargissement du rayon de couverture n'est nécessaire, c'est-à-dire  \begin{displaymath}
	 d(O_r, O_n) \le r(O_r).
\end{displaymath}  S'il existe plusieurs sous-arbres ayant cette propriété, on choisit celui pour lequel l'objet $ O_n $ est le plus proche de $ O_r $. Ce critère heuristique tente d'obtenir des sous-arbres bien regroupés, ce qui a un effet bénéfique sur les performances.\\

Si aucun pivot pour lequel $ d(O_r, O_n) \le r(O_r) $ existe - donc un rayon de couverture doit être élargi - notre choix est de minimiser l'augmentation du rayon de couverture, c'est-à-dire $ d(O_r, O_n)-r(O_r) $. Ceci est étroitement lié au critère heuristique qui suggère de minimiser le "volume" global couvert par les pivots dans le nœud actuel. L'algorithme Insert résume les arguments ci-dessus.
\begin{figure}[H]
	\centering
	\includegraphics[width=.9 \textwidth]{Figures/insert.png} % Include the image .png
\end{figure} 


\subsection{Stratégies de fractionnement}
La stratégie de fractionnement "idéale" devrait sélectionner les deux objets à promouvoir, $ O_{p1} $ et $ O_{p2} $, et répartir les entrées de telle sorte que les deux régions ainsi obtenues aient un "volume" minimum et un "chevauchement" minimum. Ces deux critères visent à améliorer l'efficacité des algorithmes de recherche, car le fait d'avoir des régions de petite taille (faible volume) conduit à des arbres bien groupés et réduit la quantité d'espace inactif indexé - espace où aucun objet n'est présent - et le fait que le chevauchement entre les régions soit faible (voire nul) réduit le nombre de chemins à parcourir pour répondre à une requête.\\

Le critère du volume minimum conduit à concevoir des stratégies de fractionnement qui tentent de minimiser les valeurs des rayons de couverture, tandis que l'exigence de chevauchement minimum suggère que, pour des valeurs fixes des rayons de couverture, la distance entre les deux objets de référence choisis devrait être maximisée.

\subsubsection{Choix des pivots}
La méthode Promote détermine, à partir d'un ensemble d'entrées, $ Ne $ , deux objets à promouvoir et à stocker dans le nœud parent (voir section 3.5.2). Les algorithmes spécifiques que nous considérons peuvent d'abord être classés selon qu'ils "confirment" ou non l'objet parent dans son rôle.

\begin{description}
	\item[Définition : ]  Une stratégie de fractionnement confirmée choisit l'un des deux objets à promouvoir, par exemple l'$ O_{p1} $, pour être l'objet parent lui-même, $ O_p $, du nœud de fractionnement.
\end{description}

En d'autres termes, une stratégie de fractionnement confirmée ne fait que "extraire" une région, centrée sur le deuxième Pivot, l'$ O_{p2} $, de la région qui restera centrée sur l'$ O_p $. En général, cela simplifie l'exécution du fractionnement et réduit le nombre de calculs de distance.\\

Les alternatives que nous décrivons pour la mise en œuvre de Promote ne sont qu'un sous-ensemble sélectionné de l'ensemble que Ciaccia, Patella, Rabitti et Zezula ont évalué expérimentalement. Les autres stratégies sont décrit dans [CiPaZe97].

\begin{itemize}
	\item \textbf{m\_RAD :} L'algorithme du "minimum (sum of) RADius" est le plus complexe en termes de calcul de la distance. Il prend en compte toutes les paires d'objets possibles et, après avoir partitionné l'ensemble des entrées, promeut la paire d'objets pour laquelle la somme des rayons de couverture, $ r(O_{p1}) + r(O_{p2}) $, est minimale.
	\item \textbf{RANDOM :} Cette variante sélectionne simplement de manière aléatoire le ou les objets de référence.
	Bien que cette stratégie ne semble pas "intelligente", elle est rapide et ses performances peuvent servir de référence pour d'autres stratégies.
	\item \textbf{SAMPLING :} C'est la stratégie RANDOM, mais elle a été répétée sur un échantillon d'objets de taille $ s > 1 $. Pour chacune des paires d'objets $ s(s - 1)/2 $ de l'échantillon, les entrées sont réparties et des rayons de couverture potentiels sont établis. La paire pour laquelle la somme résultante des rayons de couverture, $ r(O_{p1})+r(O_{p2}) $, est minimale est alors sélectionnée.
	En cas de promotion confirmée, seules $ s $ différentes distributions sont essayées. Dans les expériences de [CiPaZe97], la taille de l'échantillon a été fixée à $ 1/10 $-ème de la capacité des nœuds.
	\item \textbf{M\_LB\_DIST :} L'acronyme signifie "Maximum Lower Bound on DISTance".
	Cette stratégie diffère des précédentes en ce sens qu'elle n'utilise que les distances stockées pré-calculées. Dans la version confirmée, où $ O_{p1} \equiv O_p $, l'algorithme détermine que l'$ O_{p2} $ est l'objet le plus éloigné de l'$ O_p $, c'est-à-dire:
	\begin{equation}
		 d(O_{p2}, O_p) = \max_j{d(O_j, O_p)}
	\end{equation}
	Lorsque $ O_{p1} = O_p $, les deux objets promus sont choisis de manière à ce que:
	\begin{equation}
	\makecell{d(O_{p1}, O_p) = \min_j{d(O_j, O_p)} \\
		 d(O_{p2}, O_p) = \max_j{d(O_j, O_p)}}
	\end{equation}
	La distance entre les deux pivots est alors garantie d'être au moins\\ $ d(O_{p2}, O_p)-d(O_{p1}, O_p) $, et aucun autre choix ne peut conduire à une limite supérieure.
\end{itemize}
Dans notre système nous avons choisi la dernière mèthode qui est M\_LB\_DIST.

\subsubsection{Répartition des entrées}
Étant donné un ensemble d'entrées $ Ne $ et les deux pivots $ O_{p1} $ et $ O_{p2} $, le problème est de savoir comment partitionner efficacement $ Ne $ en deux sous-ensembles, $  N_1 $ et $ N_2 $. À cette fin, nous envisageons deux stratégies de base. La première est basée sur l'idée de la décomposition généralisée de l'hyperplan [Uhl91] et conduit à des fractionnements déséquilibrés, alors que la second obtient une répartition équilibrée. Ils peuvent être brièvement décrits comme suit.
\begin{itemize}
	\item \textbf{Hyperplan généralisé (Generalized Hyperplane) :} Assignez chaque objet $ O_j \in Ne $ au  pivot le plus proche, c'est-à-dire si  \begin{displaymath}
		d(O_j, O_{p1}) \le d(O_j, O_{p2}) 
	\end{displaymath} alors assignez $ O_j $ à $ N_1 $, sinon assignez $ O_j $ à $ N_2 $.
	
	\item \textbf{Équilibré (Balanced) :} Calculer $ d(O_j, O_{p1}) $ et $ d(O_j, O_{p2}) $ pour tous les $ O_j \in Ne $ Répéter jusqu'à ce que $ Ne $ soit vide :\\
	 - Affecter à $ N_1 $ le plus proche voisin de l'$ O_{p1} $ dans $ Ne $ et le retirer de $ Ne $ ;\\
	 - Attribuer à $ N_2 $ le plus proche voisin de l'$ O_{p2} $ dans $ Ne $ et le retirer de $ Ne $.
\end{itemize}
En fonction de la distribution des données et de la manière dont les pivots sont choisis, une stratégie de fractionnement déséquilibrée peut conduire à un meilleur partitionnement des objets, en raison du degré de liberté supplémentaire que l'on obtient. En particulier, il faut remarquer que, si l'obtention d'un fractionnement équilibré avec des méthodes d'accès spatiale oblige à n'élargir les régions que selon les dimensions nécessaires, dans un espace métrique, l'augmentation conséquente du rayon de couverture se propagerait à toutes les "dimensions".\\

Un comportement intermédiaire peut être obtenu en combinant les deux algorithmes ci-dessus et en fixant un seuil minimum d'utilisation des nœuds. Si au moins $ m \le M/2 $ entrées par nœud sont nécessaires, la distribution équilibrée peut être appliquée aux 2 premiers $ m $ objets, après quoi l'hyperplan généralisé pourrait être utilisé. 

Pour en savoir plus sur l'évaluation des pérformances de la structures M-tree avec ses différentes types de requêtes et stratégies de fractionnement (partitionnement) le lecture peut consulter l'article [CiPaZe97].

\section{Conclusion}
Dans ce chapitres nous avons fait une études non exhaustive des méthodes d'indexations et leurs classifications. Pricipalement, nous avons présenté la structure d'accès métrique M-Tree et comment elle est construite, ainsi que ses algorithmes d'insertion et de recherche.

Généralement, la structure M-tree présente des avantages suivants: coût de recherche diminuer, dynamique, et résistance à l’augmentation de
la dimension. Mais, il peut présenter des chevauchement des nœuds [Hanif18].\\


Dans ce qui suit nous présenterons notre systèmes de recherche d'images par le contenu.

%
%
%
%La recherche par la requête intervalle R(q,r) d’un objet requête q et de rayon r dans M-tree exploite les distances pré-calculer pour éviter le parcours des nœuds inutiles. En effet, le processus de recherche commence par la racine, et pour chaque nœud visité: si la condition
%$ |d(q,p^p) — d(pp,^p)| — r^c > r $ est vérifiée alors on peut éviter le parcours du nœud sans faire des calculs supplémentaires, tel que $ d(q,p^p) $ est la distance entre la requête et l’objet père du nœud , $ d(p^p,p) $ est la distance entre l’objet p et l’objet père du nœud et $ r^c $ le rayon de couverture de nœud.
%
%La complexité de la construction de M-tree est d’ordre$  O(n.m^2.log_m n) $ telle que n est lenombre des distances stockées dans les feuilles et m représente la taille maximale des nœuds.


%P. Ciaccia, M. Patella, F. Rabitti, P. Zezula Indexing Metric Spaces with M-tree
% Chapter Template

\chapter{Implémentation et évaluation expérimentale} % Main chapter title

\label{Chapter4} % Change X to a consecutive number; for referencing this chapter elsewhere, use \ref{ChapterX}

%----------------------------------------------------------------------------------------
%	SECTION 1
%----------------------------------------------------------------------------------------

\section{Introduction}
Après avoir étudié le domaine de CBIR (les systèmes de recherche d’images par le contenu ) et leur principe de fonctionnement, l’implémentation d’une application ou d’un système de recherche d’images par le contenu devient une nécessité afin d’avoir une vue plus claire de ce que nous avons introduit dans les premiers chapitres. \\

Nous présenterons alors, dans ce chapitre, notre application,
les descripteurs utilisés et les bases d’images choisies afin de pouvoir évaluer le système, ainsi les différentes étapes par lesquelles nous sommes passés pour sa réalisation.

\section{L'outil d’implémentation}
Dans la conception de notre application, nous avons choisi Python
comme langage de programmation, ce choix est justifié par plusieurs
facteurs, parmi eux on cite:
\begin{itemize}
	\item Une librairie très riche, il est complété par de multiples boîtes à outils (le calcul numérique matriciel avec Numpy, vision par ordinateur avec OpenCV, ...etc).
	\item Une syntaxe simple permettant une souplesse durant l'implémentation.
	\item Possible d’exécuter le code en dehors du programme (Testes unitaires).
	\item Une aide très bien faite.
	\item Opensource contrairement à Matlab.
\end{itemize}

\begin{figure}[H]
	\centering
	\includegraphics[width=0.3\textwidth]{python}
	\caption{Exemple de librairies Python.}
\end{figure}

\section{Les descripteurs d’image utilisés}
La première étape de création d'un système CBIR est le choix d'un descripteur pour indexer les images. \\
Pour la création de notre vecteur descripteur (signature), nous avons utilisé des caractéristiques de bas niveau comme déjà spécifier dans le chapitre 2. L'extraction se fait selon l'attribut visuel choisi: couleur, texture ou forme.
\subsection{Les descripteurs de couleur}
\textbf{Les modèles de couleur: }
Dans notre système, nous avons intégré l'espace RGB et l'espace TSV (HSV):\\

\begin{itemize}
	\item Le modèle TSV (Teinte Saturation Valeur) : est une représentation physique de la couleur, cet espace présente l’avantage de simuler le comportement visuel humain.
	
	\item Le modèle RVB (Rouge, Vert, Bleu) : est l’espace de couleur le plus utilisé pour la représentation de la couleur. L’avantage d’utiliser ce modèle est que cette représentation est extrêmement basique, puisqu’aucun traitement n’est nécessaire.\\
	
\end{itemize}

Notre système présente à l'utilisateur les descripteurs suivants :
\begin{itemize}
	\item L'histogramme (RGB et HSV),
	\item Les moments statistiques.
\end{itemize}

\subsubsection{Histogramme}
\paragraph{RVB (RGB):}
Dans la création de l’application, nous avons choisi d’utiliser les histogrammes par bloc dans l’espace RVB comme une technique de base comme spécifier dans l'article [Abed15].
On a choisi de diviser l'histogramme en 17 blocs pour chaque composante; Rouge, Verte et Bleu. Le vecteur de caractéristiques $ V_{histRGB} $ est généralement de taille 17x17x17 = 4913.
\begin{table}[H]
	\centering
	\caption{Exemple d'histogramme par blocs}
	\begin{tabular}{|c|c|c|c|c|c|}
		\hline
		\textbf{Block} & \textbf{Fréquence} & \textbf{Block} & \textbf{Fréquence} & \textbf{Block} & \textbf{Fréquence}\\
		\hline
		
		\makecell{0-15 } & \makecell{454 } & \makecell{16-30 } & \makecell{2324 }   & \makecell{31-45 } & \makecell{345 }   \\
		\hline
		
		\makecell{46-60 } & \makecell{903 } & \makecell{61-75 } & \makecell{133 }   & \makecell{76-90 } & \makecell{563 }   \\
		\hline
		
		\makecell{91-105 } & \makecell{123} & \makecell{106-120 } & \makecell{67 }   & \makecell{121-135 } & \makecell{124 }   \\
		\hline
		
		\makecell{136-150 } & \makecell{856} & \makecell{151-165 } & \makecell{45 }   & \makecell{166-180 } & \makecell{454 }   \\
		\hline
		
		\makecell{181-195 } & \makecell{355} & \makecell{196-210} & \makecell{31}   & \makecell{211-215 } & \makecell{4546 }   \\
		\hline
		
		\makecell{216-230 } & \makecell{456} & \makecell{231-255} & \makecell{3456}   & \makecell{  } & \makecell{  }   \\
		\hline
	\end{tabular}
	
	
\end{table}
\paragraph{TSV (HSV):}
Pour notre moteur de recherche d'images, nous utiliserons un histogramme couleur 3D dans l'espace couleur HSV avec 8 blocs pour le canal teinte, 12 blocs pour le canal saturation et 3 blocs pour le canal valeur en divisant l'image en 5 régions, ce qui donne un vecteur de caractéristiques $ V_{histHSV} $ est de  dimension 8 x 12 x 3 x 5 = 288 x 5 = 1440.
\begin{figure}[H]
	\centering
	\includegraphics[width=0.3\textwidth]{5regions}
	\caption{Exemple de division d'une image en 5 régions différentes [Site01].}
\end{figure}
\subsubsection{Les moments de couleur}
Pour enrichir les index de la couleur, nous avons utilisé les moments de couleur au lieu de calculer la distribution complète comme pour les histogrammes. Dans cette étape, nous avons calculé les trois premiers moments de couleur (la moyenne, l’écart-type et le moment d'ordre 3) pour chaque canal (R,G,B) dans le but de garder seulement les neuf valeurs obtenues, ainsi le vecteur de caractéristiques $ V_{moments} $ est de tailles 9.

\subsection{Les descripteurs de texture}
Pour caractériser la texture, nous avons choisi l'un des descripteurs classiques à savoir les mesures de Haralick basé sur les matrices de co-occurrence. De plus, nous avons intégré les filtres de Gabor comme étant l'un des descripteurs fortement utilisés dans la littérature.
\subsubsection{Les mesures de Haralick}
Les mesures de Haralick sont calculées à partir de la matrice de co-occurrence de l'image. Haralick décrit 14 statistiques qui peuvent être calculées à partir de la matrice de co-occurrence dans le but de décrire la texture de l'image [Site02]. Nous adoptant l'implémentation de la librairie Mahotas [Site03] qui met en œuvre seulement les 13 premières mesures. La dernière (14ème) est normalement considérée comme instable. Alors, le vecteur de caractéristiques $ V_{Haralick} $ est de taille 13.

\subsubsection{Les filtres de Gabor}
Dans le but d'améliorer les performances lié à la recherche basé sur la texture, nous avons utilisé les filtres de Gabor vue leurs robustesse prouvé par de nombreuse travaux de recherche dans la littérature [ElHasnaoui17] et [ZZ18].\\

En traitement d'images, un filtre de Gabor, nommé d'après Dennis Gabor, est un filtre linéaire utilisé pour l'analyse de la texture:

\begin{equation}
g(x,y;\lambda,\theta,\psi,\sigma_x,\sigma_y) = \frac{1}{2\pi \sigma_x \sigma_y} \exp\left(\frac{-x'^2}{\sigma_x^2} + \frac{y'^2}{\sigma_y^2}\right)\exp\left(i\left(2\pi\frac{x'}{\lambda}+\psi\right)\right) 
\end{equation}
Où: $ x' = x \cos\theta + y \sin\theta\, $, $ y' = -x \sin\theta + y \cos\theta\, $, \\
$\psi$ : La phase, \\
$\theta$ : la direction, \\
et $ \lambda $: la fréquence.\\

Les filtres d'ondelettes de Gabor s'étendent sur 5 fréquences : 0.06, 0.09, 0.13, 0.18, 0.25 avec huit orientations $ \theta_0 = 0, \theta_n = \theta_{n-1} + \frac{\pi}{8} , n = 1,...,7$  et qui sont appliqués sur l'image. La moyenne et l'écart-type des coefficients d'ondelettes de Gabor des images résultante sont utilisés pour former un vecteur de caractéristiques $ V_{Gabor} $ de taille 5x8x2 = 240.

\subsection{Les descripteurs de forme}
Dans la partie basé sur la forme, nous avons intégré les moments de Hu comme descripteur classique et les moments de Zernike qui constitue une amélioration de performance comparant au moment de Hu.
\subsubsection{Moments de Hu}
Les moments de Hu sont basés sur les moments géométriques discutés dans le chapitre II section 2.2.3.\\
Si ƒ(x, y) est une image numérique, alors l'équation précédente devient:
\begin{equation}
\mu_{p,q} = \sum_{x=0}^{M-1}\sum_{y=0}^{N-1} (x- \bar{x})^p (y-\bar{y})^q f(x, y)
\end{equation}
D'où les moments géométriques (centraux):
\begin{figure}[H]
	\includegraphics[width=0.5\textwidth]{momentCent} \includegraphics[width=0.5\textwidth]{momentCent1}
\end{figure}
Où:
\begin{equation}
\begin{tabular}{cc}
$\bar{x} = \frac{M_{1,0}}{M_{0,0}}$ & $\bar{y} = \frac{M_{0,1}}{M_{0,0}}$
\end{tabular}
\end{equation}
Les moments $ \mu_{p,q} $ invariants en ce qui concerne à la fois la translation et l'échelle peuvent être construits à partir des moments géométriques en divisant par un moment central zéro-ième correctement mis à l'échelle :
\begin{equation}
\eta _{{ij}}={\frac  {\mu _{{ij}}}{\mu _{{00}}^{{\left(1+{\frac  {i+j}{2}}\right)}}}}\,\!
\end{equation}
où $ i + j \ge 2 $.
Comme le montre le travail de Hu, [Hu62] les invariants en matière de translation, d'échelle et de rotation peuvent être construits :
\begin{figure}[H]
	\centering
	\includegraphics[width=1\textwidth]{huMom}
	\caption{Les moments de Hu.}
\end{figure}

Le vecteur de caractéristiques $ V_{Hu} $ est alors de taille 7.
\subsubsection{Moments de Zernike}
En 1934, Zernike [Site05] [MYB07] a proposé un ensemble de polynômes orthogonaux définis sur le cercle unité, à savoir les polynômes orthogonaux de Zernike, leur formule de définition est la suivante :
\begin{equation}
V_{mn}(x, y)~=~V_{mn}(r,\theta)~=~R_{mn}(r)\exp(jn\theta)
\end{equation}
où:
\begin{displaymath}
R_{mn}(r) = \sum_{s=0}^{\frac{m-\mid n \mid}{2}}(-1)^{s}~~F(m,n,s,r)
\end{displaymath}
\begin{displaymath}
F(m,n,s,r) = \frac{(m-s)!}{s!(\frac{m+\mid n \mid}{2}~-s)!~(\frac{m-\mid n \mid}{2}~-s)! }~r^{m-2s}
\end{displaymath}
$ R_{mn}(r) $  est le polynôme radial orthogonal, $ V_{mn}(x, y) $ est le polynôme orthogonal de Zernike, c'est un ensemble de fonctions orthogonales à valeurs complexes avec complétude définies sur le disque unité $x^{2} + y^{2} \leq ~1$ , n et m sont les ordres des polynômes orthogonaux de Zernike, où n est un entier positif ou zéro, m est un entier positif ou négatif, ils sont soumis aux conditions 
\begin{equation}
m- \mid n \mid ~=~pair~~,~~\mid n \mid~\leq~m
\end{equation}

\paragraph{Définition des Moments de Zernike:}
En 1980, sur la base des polynômes orthogonaux de Zernike, Teague a proposé pour la première fois la définition des moments de Zernike d'une fonction d'image f( x, y) en deux dimensions
\begin{equation}
Z_{mn} = \frac{m+1}{\pi} \int_{x} \int_{y} f(x,y)[V_{mn}(x,y)]^{*} ~dx~dy~~~~~\mbox{where $x^{2} + y^{2} \leq ~1$}
\end{equation}
où $m = 0,1,2,...,\infty$ et définit l'ordre, $f(x,y)$ est la fonction décrite et $*$ désigne le conjugué complexe. Alors que $n$ est un nombre entier (qui peut être positif ou négatif) décrivant la dépendance angulaire, ou rotation, sous conditions (4.4). Pour les images numériques, les intégrales sont remplacées par des sommations, alors les moments de Zernike sont réécrits comme :

\begin{equation}
Z_{mn} = \frac{m+1}{\pi} \sum_{x} \sum_{y} P_{xy}[V_{mn}(x,y)]^{*} ~~~~~\mbox{where $x^{2} + y^{2} \leq ~1$}
\end{equation}
Dans la littérature on trouve d'autres variants ou améliorations des moments de Zernike. Pour nous, on a adopter l'implémentation de la librairie Mahotas pour nous simplifier la vie. On générale, il faut fixer un rayon centré sur le centre de masse de l'image (le rayon du cercle enveloppant minimal de la forme).
Le vecteur de caractéristiques $ V_{Zernike} $ et de taille 25.

\subsection{Combinaison des descripteurs}
Les attributs: couleur, texture, forme décrivent les images par leur contenu visuel. La combinaison de ces attributs peut caractériser mieux le contenu. Il est donc intéressant de
combiner ces différents attributs pour une recherche plus efficace et plus discriminante. Les problèmes qui se posent lors de la combinaison de ces différents attributs pour la recherche et l’indexation sont au moins de trois ordres :

\begin{itemize}
	\item  \textbf{L'espace de description} : Le choix de l'espace de description consiste à rechercher les attributs visuels significatifs de la base de données d'images, l'ensemble de ces
	attributs étant représenté par un nuage de points dans un espace dimensionnel haut, alors les vecteurs contiennent plusieurs attributs, un problème qui se pose est celui de la dimension de l'espace de description. Ce problème est connu dans la communauté
	des bases de données par la malédiction de la dimension, lorsque le nombre de dimensions augmente, le volume de l'espace croît rapidement si bien que les données se retrouvent isolées et deviennent éparses.
	
	\item \textbf{ La mesure de la similarité} : Il s'agit d'une étape essentielle dans tout système de
	recherche. Dans le cas où les images sont décrites par différents attributs, une solution
	classique pour mesurer la similarité est de calculer séparément les mesures de
	similarité pour chaque attribut et de déduire ensuite une mesure composite de la
	similarité globale entre les images. Cela suppose évidemment que les différents
	attributs sont indexés séparément (avec des structures d'index séparées). Dans la base
	de données, il y a peu de méthodes qui utilisent plusieurs index pour structurer les
	données. Une autre difficulté liée à la similitude est de déterminer comment combiner
	plusieurs mesures souvent définies sur des domaines différents, avec des dynamiques
	différentes, des degrés d'importance différents, surtout pour l'utilisateur, mais aussi de
	natures différentes.
	
	\item \textbf{Structuration} : la phase de construction d'une structure d'index est une étape utile dans
	le cas où les données sont volumineuses et appartiennent à un grand espace de
	description. Il s'agit de structurer les nuages de points relatifs aux descripteurs des
	images et de les stocker efficacement dans une machine. Cette tâche de structuration
	peut s'avérer difficile dans le cas où les données à structurer sont de nature hétérogène.
	La difficulté réside dans le choix de la distance à utiliser pour structurer (mise en place
	d'un index) et dans la standardisation des différents types de données.
\end{itemize}
\subsubsection{La couleur et la texture}
Pour diminuer la taille du vecteur descripteur (signature) autant que possible nous avons choisi d'utiliser les moments de couleur, qui produit une signature de taille 9, pour caractériser la couleur. En ce qui concerne la caractérisation de la texture on travaille avec les filtres de Gabor.
\begin{figure}[H]
	\centering
	\includegraphics[width=0.6\textwidth]{clrtxtr}
	\caption{Dérivation de signature couleur et texture.}
\end{figure}
Où: 
M: Moyenne, E: Ecart-type, T: moment d'ordre trois.
\begin{equation}
V_{final} = V_{moments} \bigcup V_{Gabor}
\end{equation}
\subsubsection{La couleur et la forme}
Nous utilisons les moments de couleur pour les mêmes raisons pour caractériser la couleur. En ce qui concerne la caractérisation de la forme on travaille avec les moments de Zernike car l'expérience preuve qu'il sont plus performant que les moments de Hu.
\begin{figure}[H]
	\centering
	\includegraphics[width=0.6\textwidth]{clrshp}
	\caption{Dérivation de signature couleur et texture.}
\end{figure}
Où: 
$ M_i $: Moment i.
\begin{equation}
V_{final} = V_{moments} \bigcup V_{Zernike}
\end{equation}

Pour fusionner les différents descripteurs, une méthode utilisée dans différents travaux [ElAsnaoui17] consiste à pondérer les distances de similarité. Ainsi les descripteurs pertinents pour la recherche se verront attribuer des poids les plus importants, et auront donc plus d’influence dans le résultat final. La combinaison des distances est donnée par la relation suivante :
\begin{equation}
	D_{global}(Q, I) = w_1 \times 	D_{1}(Q, I) + w_2 \times 	D_{2}(Q, I)
\end{equation}
où $ w_i $ est un poids qui prend une valeur entre 0 et 1 avec $ w_1+w_2= 1 $.
\section{Mesures de similarité}
La deuxième étape dans les systèmes de recherche d'image par le contenu est le choix d'une mesure de similarité pour effectuer la recherche.\\

Pour rechercher les images les plus similaires à une image requête, il faut pouvoir mesurer la similarité entre les images. Lorsqu’un utilisateur lance une recherche, le système effectue une mesure entre le signature de la requête et les signatures des images de la base dans l’espace des attributs (signatures).\\

D'une façon générale, nous avons choisi d’utiliser les distances dérivées de la famille de distances de Minkowski; la distance de Manhatan et la distance Euclidienne.\\

Pour les distributions statistiques à savoir les histogrammes, nous avons ajouter les distances:  $\chi^2$ (CHI-square) et Bhattacharyya  .

\begin{table}[H]
	\centering
	\caption{Les distances de Minkowski}
	\begin{tabular}{|c|c|c|}
		\hline
		\textbf{Distance} & \textbf{Formule}\\
		\hline
		\makecell{Manhatttan } & \makecell{\\
			$  d_1(I_1, I_2) = \sum_{i=1}^{N} \left|{I}_{1}(i)-{I}_{2}(i)\right|  $ }   \\
		\hline
		
		\makecell{Euclidienne} & \makecell{\\ $ d_2(I_1, I_2) =  \sqrt{\sum_{i=1}^{N} \left|{I}_{1}(i)-{I}_{2}(i)\right|^2} $}   \\
		\hline
		
		\makecell{$\chi^2$ (CHI-square)  } & 
			\makecell{\\
				$d(I_1, I_2)=\sum_{i=1}^{N} \frac{(I_1(i)-I_2(i))^2}{I_1(i)}$
			} \\   
		\hline
		
		\makecell{Bhattacharyya } & 
		\makecell{\\
			$d(I_1, I_2) ~=~ \sqrt{ 1 - \frac{1}{ \sqrt{\bar{I_1}\bar{I_2} N^2} }  \sum_{i=1}^{N} \sqrt{I_1(i)\times I_2(i)}}$
		} \\   
		\hline
		
		
	\end{tabular}
	
\end{table}
Ou :\\
$ I_1, I_2 $ sont les deux vecteurs de caractéristiques.\\
$ N $ est la dimension de vecteur.\\
$ i $ indice.
$ bar{I_i} $ la moyenne.

\section{Les bases d’mages utilisées }
\textbf{Corel-1000 - Wang:}
Cette base d’images contient 1000 images en couleurs. Ces images ont
été divisées en 10 classes de 100 images. Les thèmes représentés par les 10 classes sont : monuments, plage, autobus, dinosaures, aliments, éléphants, fleurs, chevaux, montagnes et l’Afrique.

\begin{figure}[H]
	\centering
	\includegraphics[width=0.6\textwidth]{wang} 
	\caption{Quelques images de la base d’image WANG.}
\end{figure}

\textbf{Columbia Object Image Library (COIL-100):}
Cette base d’images est très connue pour la reconnaissance des objets. Il y a deux bases d’images COIL : COIL-20 qui contient des images en niveaux de gris prises à partir de 20 objets différents et COIL-100 qui contient des images en couleurs prises à partir de 100 objets différents. Les deux bases d’images consistent en des images prises à partir des objets 3D avec des positions différentes. La base COIL-100 a 7200 images en couleurs (100 objets x72 images/objet). Chaque image a une taille de 128x128 pixels.

\begin{figure}[H]
	\centering
	\includegraphics[width=0.6\textwidth]{coil} 
	\caption{Objets utilisés dans COIL-100.}
\end{figure}

\textbf{Columbia-Utrecht Reflectance and Texture Database (CuRRET):}
Des chercheurs des universités de Columbia et d'Utrecht ont collaboré dans une étude approfondie de l'aspect visuel des surfaces du monde réel. Cet effort conjoint, parrainé en partie par REALISE de la Commission européenne, la National Science Foundation et par le DARPA/ONR dans le cadre de la subvention MURI n° N00014-95-1-0601, a donné lieu à 3 bases de données [Site04]. Nous avons choisi 33 échantillons parmi 61 échantillons avec 50 images par échantillon.
\begin{figure}[H]
	\centering
	\includegraphics[width=0.5\textwidth]{curet} 
	\caption{Quelques images de la base d’image CurRRET.}
\end{figure}

%\textbf{MIT1995 - VisTex:}
%La base de données VisTex est une collection d'images de texture. La base de données a été créée dans le but de fournir un large ensemble de textures de haute qualité pour les applications de vision par ordinateur. En particulier, l'ensemble a été conçu comme une alternative à la bibliothèque de textures Brodatz, qui n'est pas disponible gratuitement pour la recherche [Site01]. L'objectif de VisTex est de fournir des images de texture qui sont représentatives des conditions du monde réel. Si VisTex peut remplacer les collections de textures traditionnelles, il comprend des exemples de nombreuses textures non traditionnelles. La base de données comporte plus de 100 images.
%
%\begin{figure}[h]
%	\centering
%	\includegraphics[width=0.6\textwidth]{VisTex} 
%	\caption{Quelques images de la base d’image VisTex.}
%\end{figure}
%
%\textbf{MIT1995 - TRUNK12:}
%Comme il n'existe pas d'ensemble de données d'images d'écorces d'arbres connu et accessible au public, un nouvel ensemble de données accessible au public a été créé dans le cadre de la thèse de Bsc [Site02]. Il contient environ 360 images d'écorces de 12 arbres différents qui se trouvent en Slovénie. Chaque classe d'arbres comprend environ 30 images au format JPEG, avec une résolution de 3000 x 4000 pixels. Toutes les images ont été prises avec l'appareil photo Nikon COOLPIX S3000 dans les mêmes conditions (distance de 20 cm, plusieurs arbres par classe, en évitant le bruit comme la mousse, mêmes conditions de lumière, prises en position verticale).
%
%\begin{figure}[H]
%	\centering
%	\includegraphics[width=0.6\textwidth]{trunk} 
%	\caption{Quelques images de la base d’image TRUNK12.}
%\end{figure}

\section{Mesure de la qualité des réponses}
Dans les systèmes de recherche d’images, une image est pertinente pour
une requête si les deux images sont dans la même classe.
La précision (ou valeur prédictive positive) est la proportion des items pertinents parmi l'ensemble des items proposés; le rappel (ou sensibilité) est la proportion des items pertinents proposés parmi l'ensemble des items pertinents. Ces deux notions correspondent ainsi à une conception et à une mesure de la pertinence.\\

Lorsqu'un moteur de recherche, par exemple, retourne 30 pages web dont seulement 20 sont pertinentes (les vrais positifs) et 10 ne le sont pas (les faux positifs), mais qu'il omet 40 autres pages pertinentes (les faux négatifs), sa précision est de 20/30 = 2/3 et son rappel vaut 20/(20+40) = 1/3.\\

La précision peut ainsi être comprise comme une mesure de l'exactitude ou de la qualité, tandis que le rappel est une mesure de l'exhaustivité ou de la quantité.\\


\begin{figure}[H]
	\includegraphics[width=0.5\textwidth]{recpre} 
	\includegraphics[width=0.5\textwidth]{recallprecision} 
	\caption{Rappel et précision comme illustré par Wikipédia.}
\end{figure}
Le rappel et la précision sont deux mesures très utilisées pour l’évaluation des performances d’un système CBIR. 

\begin{equation}
R = \frac{\text{VP}}{\text{FN et VP}} = \frac{\text{Nombre d'images pértinentes retrouvées} }{\text{Nombre total d'images pértinentes}}
\end{equation}
\begin{equation}
P = \frac{\text{VP}}{\text{VP et FP}} = \frac{\text{Nombre d'images pértinentes retrouvées}}{\text{Nombre total d'images retrouvées}}
\end{equation}
Où: \\
VP: Vrais Positifs\\
VN: Vrais Négatifs\\
FP: Faux Positifs\\
FN: Faux Négatifs.
 
En statistique, la précision est appelée valeur prédictive positive et  le rappel est appelé sensibilité.\\

D'après ses deux mesures on peut déduire un troisième qu'on appelle F-mesure ou F-score. Une mesure qui combine la précision et le rappel est leur moyenne harmonique:
\begin{equation}
F = \frac{2\times R\times P}{R + P} 
\end{equation}
La F-mesure correspond à un compromis de la précision et du rappel donnant la performance du système.\\

Il faut noter que nous avons ajouter la possibilité d'afficher la matrice de confusion: une matrice qui mesure la qualité d'un système de classification. Chaque ligne correspond à une classe réelle, chaque colonne correspond à une classe estimée. 
\begin{figure}[H]
	\centering
	\includegraphics[width=0.7\textwidth]{Figures/cm.png} 
	\caption{Exemple d'une matrice de confusion (source: Wikipédia).}
\end{figure}
Le système contient une fenètre pour calculer la pértinance de chaque requête:
\begin{figure}[H]
	\centering
	\includegraphics[width=0.6\textwidth]{mesures} 
	\caption{Exemple d'une matrice de confusion calculé par le système.}
\end{figure}
\section{Architecture de l’application}
Dans un système d’extraction d’images par le contenu il existe deux types de traitements :
\begin{itemize}
	\item \textbf{Traitement offline :}
	Ce type de traitement représente la phase de la construction de la base des signatures (d’attributs/vecteurs descripteurs).
	Cette opération est réalisée durant la construction du système.
	\item \textbf{Traitement online :}
	Ce type de traitement est effectué lors de l’introduction de la requête de l’utilisateur.
	Les signatures sont extraits de l’image requête puis comparés à ceux de la base de signatures qui a été déjà construite au préalable.
\end{itemize}
\begin{figure}[H]
	\centering
	\includegraphics[width=0.9\textwidth]{architectureBDMM} 
	\caption{Architecture de l’application.}
\end{figure}

\begin{figure}[H]
	\centering
	\includegraphics[width=0.9\textwidth]{architecture} 
	\caption{Architecture de l’application avec utiles utilisés.}
\end{figure}
\paragraph{La base de signatures}
La base d’attributs ou signatures est constituée d’un ensemble de fichiers CSV (un fichier par image). Chaque fichier est constituée de $ n+1 $ champs :
\begin{itemize}
	\item 1 champ pour le chemin absolu de l'image (oid pour M-Tree),
	\item n champs de vecteurs descripteurs calculer par le descripteur.
\end{itemize}
\section{L’interface utilisateur }
Généralement, l'interface utilisateur de notre système permet à l'utilisateur des tâches comme:
 \begin{itemize}
 	\item Rechercher des images similaires à l'image requête (en ligne),
 	\item Indexer une bases d'images (hors ligne),
 	\item Calculer la qualité de chaque réponse ou requête,
 \end{itemize}

L'interface principale permet à l’utilisateur (Figure 4.8): 
\begin{itemize}
	\item d'indexer une bases d'images (phase hors ligne) ou una bases de signatures,
	\item d’introduire son image requête,
	\item de choisir le descripteur à utiliser avec le type d'images en cas d'indexation d'une base d'images,
	\item de choisir la mesure de similarité à utiliser,
	\item de rechercher des images similaires à l'image requête (en ligne),
	\item de visualiser les images indexer pour en sélectionner la requête,
	\item de calculer la qualité de chaque réponse ou requête,
	\item de consulte la fenêtre aide pour savoir comment utiliser le système...etc.
\end{itemize}

\begin{figure}[H]
	\centering
	\includegraphics[width=0.7\textwidth]{gui} 
	\caption{L'interface d’utilisateur.}
\end{figure}

La section des options de l'indexation et la recherche change selon les attributs visuels choisis.

\begin{figure}[H]
	\centering
	\includegraphics[width=0.35\textwidth]{baseC} \space
	\includegraphics[width=0.35\textwidth]{baseT} 
	\caption{L'interface d’utilisateur, section requête: \\
	Gauche: basé couler, Droite: Basé texture.}
\end{figure}
\begin{figure}[H]
	\centering
	\includegraphics[width=0.35\textwidth]{baseF} \space
	\includegraphics[width=0.35\textwidth]{baseCTS} 
	\caption{L'interface d’utilisateur, section requête: \\
		Gauche: basé forme, Droite: Basé fusion}
\end{figure}


Après avoir choisi la base à indexer et les autres paramètres relatives à l'étapes d'indexations (descripteur, type d'images et mesure de similarité) l’utilisateur indexe cette base en appuyant sur la bouton 'Indexer'. Ainsi, l'indexation commence. \\

\begin{figure}[H]
	\centering
	\includegraphics[width=0.7\textwidth]{browse folder} 
	\caption{Choix d'une base à indexer.}
\end{figure}

Une fois l'indexation est faite, l’utilisateur peut lancer une recherche après avoir choisi l'mage requête en appuyant sur la bouton 'Rechercher'. Pendant cette étape la signature de l’image est calculé pour l'utiliser dans l'un des requêtes de rechrche de la structure d'indexation M-Tree (Requête intervalle ou k-NN).\\

\begin{figure}[H]
	\centering
	\includegraphics[width=0.7\textwidth]{browse query} 
	\caption{Choix de l'mage requête.}
\end{figure}
Résultas:
\begin{figure}[H]
	\centering
	\includegraphics[width=0.7\textwidth]{query result} 
	\caption{Résultats de la recherche.}
\end{figure}

Le système affiche une liste d'images ordonnées par le degré de similarité.

Pour mesurer la qualité de la réponse on va vers le menu \textbf{fenêtre->mesures}:
\begin{figure}[H]
	\centering
	\includegraphics[width=0.7\textwidth]{mesuresQu} 
	\caption{La fenêtre des mesures de la pertinence.}
\end{figure}
On calcule:
\begin{figure}[H]
	\centering
	\includegraphics[width=0.7\textwidth]{QUresult} 
	\caption{Résultatsdes mesures de la pertinence.}
\end{figure}

\section{Évaluation de système}
Nous avons testé la performance de chaque descripteurs utilisé et dans cette section nous présenterons les différents résultats.

\paragraph{Processus suivi:}
Dans cette partie nous avons évalué l’application avec le calcule du rappel, la précision et F-mesure pour chaque image requête. Le principe de fonctionnement est simple; les images requêtes sont sélectionnées de manière aléatoires (sans répétition de la même image requête) et on calcul les mesures de pertinence. 
\subsection{Descripteurs de la couleur}
Pour cette section, nous avons effectué un ensemble de tests sur 50 images requêtes (5 images pour chaque classe) de la base Wang. Les mesures de chaque classe sont calculées par la moyenne de ses images. Puis à la phase finale, on peut déduire la précision globale de système qui représente la moyenne de toutes les classes. On travailler avec les mêmes requêtes pour les deux descripteurs.

\begin{figure}[H]
	\centering
	\includegraphics[width=.7\textwidth]{MomentsColPert} 
	\caption{Pertinence des moments de couleur.}
\end{figure}

\begin{figure}[H]
	\centering
	\includegraphics[width=.7\textwidth]{histRGBPert} 
	\caption{Pertinence de l'histogramme RGB.}
\end{figure}
On remarque, d'après les deux figures 4.22 et 4.23, que les résultats sont proches avec les deux descripteurs sauf que l'histogramme présente presque 0,70 en précision contre 0,63 pour les moments de couleur.
\subsection{Descripteurs de la texture}
Pour cette section, nous avons effectué un ensemble de tests sur 25 images requêtes de la base CuRRET avec la distance de Manhatan. On travaille avec les mêmes requêtes pour les deux descripteurs. Généralement, on effectue une segmentation des images avant d'extraire les informations.\\
On fixe les options comme suite:
\begin{figure}[H]
	\centering
	\includegraphics[width=.45\textwidth]{Haralick options} \space
	\includegraphics[width=.45\textwidth]{GaborOptions} 
	\caption{Options: des mesures de Haralick (Gauche),  des filtres de Gabor (Droite).}
\end{figure}
On obtient:
\begin{figure}[H]
	\centering
	\includegraphics[width=.7\textwidth]{HaralickPert} 
	\caption{Pertinence des mesures de Haralick.}
\end{figure}



\begin{figure}[H]
	\centering
	\includegraphics[width=.7\textwidth]{GaborPert} 
	\caption{Pertinence des filtres de Gabor.}
\end{figure}
Comme les résultats le montre, les filtres de Gabor présente des réponse de bonne qualité comparant aux mesures de Haralick (Comme le montre les figure 4.25 et 4.26, 75\% en précision pour Gabor contre 43\% pour Haralick). 
\subsection{Descripteurs de la forme}
Pour évaluer les descripteurs de la forme, nous avons effectué un ensemble de testes sur 25 images requêtes de la base COIL-100 avec la distance de Manhatan. On a travaillé avec les mêmes requêtes pour les deux descripteurs.\\

On fixe les options comme suite:
\begin{figure}[H]
	\centering
	\includegraphics[width=.45\textwidth]{HuOptions} \space
	\includegraphics[width=.45\textwidth]{ZernikeOptions} 
	\caption{Options: des mesures de Hu (Gauche), des mesures de Zernike (Droite).}
\end{figure}
On obtient:
\begin{figure}[H]
	\centering
	\includegraphics[width=.7\textwidth]{HuPert} 
	\caption{Pertinence des moments de Hu.}
\end{figure}

\begin{figure}[H]
	\centering
	\includegraphics[width=.7\textwidth]{ZernikePert} 
	\caption{Pertinence des moments de Zernike.}
\end{figure}
On remarque que les moments de Zernike sont plus performant que les moments de Hu. Comme le montre les figures 4.28 et 4.29, les moments de Zernike donnent une précision de 55\% alors que ceux de Hu donnent seulement 37\%.
\subsection{Descripteurs combinés}
Dans cette partie nous travaillons avec les mêmes bases de données et mêmes images requêtes utilisés dans les deux parties précédentes; CuRRET pour la texture et COIL-100 pour la forme. Nous avons fixer $  0,5 $ pour les deux poids dans un premier temps.
\begin{figure}[H]
	\centering
	\includegraphics[width=.45\textwidth]{ColTxtrOptions2} \space
	\includegraphics[width=.45\textwidth]{ColShpOptions} 
	\caption{Options des moments de couleur combiné avec les filtres de Gabor (Gauche), et des moments de couleur combiné avec les moments de Zernike (Droite).}
\end{figure}
On obtient:
\begin{figure}[H]
	\centering
	\includegraphics[width=.7\textwidth]{colTxtrPert} 
	\caption{Pertinence des moments de couleur combiné avec les filtres de Gabor.}
\end{figure}

\begin{figure}[H]
	\centering
	\includegraphics[width=.7\textwidth]{colShpPert} 
	\caption{Pertinence des moments de couleur combiné avec les moments de Zernike.}
\end{figure}
Les résultats donner dans la figure 4.31 ne présent pas une précision acceptable par rapport à celui donner dans la figure 4.24. Alors, nous avons changer les poids dans le but de l'améliorer comme suite:
\begin{figure}[H]
	\centering
	\includegraphics[width=.5\textwidth]{ColTxtrOptions}
	\caption{Options des moments de couleur combiné avec les filtres de Gabor (Deuxième expérience):
		20\% Couleur avec 80\% Texture.}
\end{figure}
On obtient:
\begin{figure}[H]
	\centering
	\includegraphics[width=.7\textwidth]{colTxtrPer2} 
	\caption{Pertinence des moments de couleur combiné avec les filtres de Gabor (Deuxième expérience).}
\end{figure}

En générale, on remarque que la combinaison des descripteurs améliore la précisions des résultats obtenus:
\begin{itemize}
	\item Pour la couleur et la texture: comme le montre le figures 4.34, on obtient une précision de 70\% contre  75\% (Figure 4.26) avec la texture seul (le choix des poids compte dans la qualité ici).
	\item Pour la couleur et la forme: comme le montre les figures 4.32, on obtient une précision de 92\% contre  55\% (Figure 4.29) avec la forme seul.
\end{itemize}
\section{Problèmes rencontrés}
\begin{itemize}
	\item La première problématique qui s’est imposée durant la réalisation du notre
	application est le choix des descripteurs discriminants d’image pour s’assurer de
	l’obtention des meilleurs résultats et pour une évaluation significative.
	\item Aussi le choix du modèle de couleur et la mesure de distance à utiliser.
	
	\item Et aussi les extensions des images. Dans un premier lieu nous avons travaillé
	qu’avec les images .jpg après, nous avons pu utiliser des images .png après une
	petite adaptation de code (il est possible d'ajouter d'autres format en modifiant une seul ligne de code).
	
	\item Dans la phase d'indexation de quelques descripteur à savoir les filtres de Gabor, l'opération prend des dizaines de minutes, une qui nous a compliqué les testes d'avancement.
	\item Pour les descripteur de forme nous avons hésiter avec la méthode de segmentation à utiliser et nous avons adobter un simple seuillage (thesholding) pour transformer les images en noir et blanc.
	\item ...etc.
\end{itemize}
\section{Perspectives}
Comme perspectives, Nous proposons :
\begin{itemize}
	\item D’utiliser les descripteurs de l'apprentissage automatique (Machine Learning).
	
	\item De Travailler avec tous les format des images existant.
	\item D’essayer autres techniques de mesures de similarité entre les descripteurs
	d’images.
    \item Tester notre application avec d’autres bases de données.
\end{itemize}
\section{Conclusion}
Au cours de ce dernier chapitre, nous avons présenté notre système de recherche d’images par le contenu où nous avons essayé d’utiliser les descripteurs connues et les mesures de similarité les plus recommandés dans la littérature. Les tests ont été effectués avec des bases d’images universellement connues. La comparaison de nos résultats avec ceux de la littérature permet de valider notre travail. 
\chapter*{Conclusion générale}
Le développement des utiles technologiques traitants le contenu multimédia a conduit à une augmentation du nombre d’images sur le Web et a rendu le développement au niveau des outils qui organisent et gèrent ces données une exigence. Le but principal de cette mémoire était de traiter le problème de la recherche d’images par le contenu. Un domaine très riche qui présente une variance dans les techniques utilisées qu’on ne peut pas limiter à une approche ou une méthode particulière.\\

Les moteurs de recherches d’images sont classés selon le niveau d'interprétation en deux catégories, ceux qui extraient le contenu bas niveau et ceux qui extraient le contenu sémantique. Dans notre travail nous somme
intéressé par la première catégorie avec la caractérisation des attributs visuels; couleur, texture et forme. La recherche dans ces systèmes se base sur quatre étapes fondamentaux, le choix de descripteur à utiliser pour l'extraction des attributs visuels, le choix d'une mesure de similarité entre ses attributs, l’indexation des images et le calcul de similarité ou la recherche.\\

Il n’existe pas une règle qui impose le choix d’une technique spécifique pour créer un système pertinent. Pour cela, nous avons présenté le principe de fonctionnement d’un système de recherche d’images par le contenu en premier lieu. Ensuite, nous avons étudié les différentes méthodes utilisées
pour l’extraction des attributs visuels avant de détailler l’approche de la mesure de similarité. Ainsi, nous avons présenté une méthode d'index M-tree que nous avons utilisé pour accélérer la recherche. A la fin de ce travail, Nous avons essayé d’appliquer nos connaissances pour développer un système qui permet à l’utilisateur de proposer facilement
une requête et visualiser les résultats. Nous avons aussi testé les différentes approches pour mieux cerner les difficultés et valider les résultats.\\
En résumé, on espère qu’on a atteint notre but d’introduire le domaine de la recherche d’images par le contenu (CBIR), ce domaine qui reste toujours ouverte à de nouvelles pistes de recherches.

%----------------------------------------------------------------------------------------
%	THESIS CONTENT - APPENDICES
%----------------------------------------------------------------------------------------

%\appendix % Cue to tell LaTeX that the following "chapters" are Appendices
%\include{Appendices/AppendixA}
%\include{Appendices/AppendixB}
%\include{Appendices/AppendixC}

%----------------------------------------------------------------------------------------
%	BIBLIOGRAPHY
%----------------------------------------------------------------------------------------

%\printbibliography
%----------------------------------------------------------------------------------------
%	BIBLIOGRAPHY
%----------------------------------------------------------------------------------------

%\bibliographystyle{apalike}
%\bibliography{example.bib} 

{

\hypersetup{
	colorlinks=false,       % false: boxed links; true: colored links
}

}
\begin{thebibliography}{widest entry}
	\bibitem[Abed15]{Abed15}  ABED, M.H.   and AL-FARTTOOSI, D.S.J. (2015) Content    based image retrieval based on histogram. International Journal of Computer Applications. ReserchGate. 
	
	\bibitem[Andaloussi10]{Andaloussi10}  Andaloussi, SJ. “Indexation de l’information médicale. Application à la recherche d’images et de vidéos par le contenu ”, Thèse de Doctorat, l’Université européenne de Bretagne Télécom Bretagne En habilitation conjointe avec l’Université de Rennes 1 Co-tutelle avec Faculté des sciences Dhar El Mehraz Fès, Université Sidi Mohamed Ben Abdellah.
	2010.
	
	\bibitem[Bimbo99]{Bimbo99} A. Del Bimbo, Visual Information Retrieval. Morgan Kaufmann Publishers, 1999.
	
	\bibitem[Chang97]{Chang97} Chang, S. F., Chen, W., Meng, H. J., Sundaram, H., \& Zhong, D.,1997,. "VideoQ : An automated content based video search system using visual cues". In The Fifth ACM International Multimedia Conference, pages 313-324, 1997.
	
	\bibitem[Chávez01]{Chávez01} Chávez, Edgar, Navarro, Gonzalo, Baeza-Yates, Ricardo, \& Marroquín, José Luis. "Searching in metric spaces. ACM Computing Surveys (CSUR)", 33(3), 273-321. 2001
	
	\bibitem[Chawki16]{Chawki16} Chawki, Y. “Estimation en Analyse Spectrale Haute Résolution Multidimensionnelle : Application en Indexation et au Filtrage des Images Sismiques”, Thèse de Doctorat, Université Moulay Ismail, Faculté des Sciences et Techniques d’Errachidia. 2016.
	
	
	\bibitem[CiPaZe97]{CiPaZe97}  Ciaccia, Paolo, Patella, Marco, \& Zezula, Pavel. "M-tree: An Efficient Access Method for Similarity Search in Metric Spaces". Paper presented at the Proceedings of the 23rd International Conference on Very Large Data Bases. 1997.
	
	
	\bibitem[Deza09]{Deza09 } Deza, Michel Marie, \& Deza, Elena. Encyclopedia of distances: Springer. 2009.
	
	\bibitem[ElAsnaoui17]{ElAsnaoui17} .Khalid EL ASNAOUI. “Indexation et recherche d’images par le contenu: Algorithmes et application au Lifelogging”, Thèse
	de Doctorat, Université Moulay Ismail, Faculté des Sciences et Techniques d’Errachidia. 2017.
	
	
	\bibitem[Fauqueurs03]{Fauqueurs03}  Julien Fauqueur. "Contributions pour la Recherche d’Images par Composantes Visuelles. Interface homme-machine [cs.HC]". Université de Versailles-Saint Quentin en Yvelines, 2003. Français. fftel-00007090ff.
	
	
	\bibitem[Flickher95]{Flickher95}  Flickner, M., Sawhney, H., Niblack, W., Ashley, J., Huang, Q., Dom, B.,. \& Steele, D., "Query by image and video content: The QBIC system". IEEE Computer, pages 23–32, September 1995.
	
	\bibitem[Gabor46]{Gabor46} Gabor, D., “Theory of communication”, Electrical Engineers-Part III: Radio and Communication Engineering, Journal of the Institution. Vol. 93, n° 26, pages: 429–441. 1946.
	
	\bibitem[Ganesan07]{Ganesan07} G.W. Jiji et L. Ganesan, "Comparative analysis of colour models for colour textures based on feature extraction". International Journal of Soft Computing, p 361–366, 2007.
	
	\bibitem[Gonz02]{Gonz02} Gonzales, R. C., et Woods, R. E. “Digital image processing. Prentice-Hall”, New Jersey, 8, 14, 27, 28. 2002.
	
	\bibitem[Gueg07]{Gueg07} Gueguen, L. “Extraction d’information et compression conjointes des séries temporelles d’images satellitaires”, Thèse de Doctorat, Ecole Nationale Supérieure des Télécommunications Paris. 2007.
	
	\bibitem[Hafner95]{Hafner95} J. Hafner, H. Sawhnay, et al. "Efficient color histogram indexing for quadratic form distance functions". IEEE Transactions on Pattern Analysis and Machine Intelligence (PAMI), 17(7) :729–736, July 1995.
	
	\bibitem[Hamming50]{Hamming50} Hamming, Richard W. "Error detecting and error correcting codes". Bell System technical journal, 29(2), 147-160, 1950.
	
	\bibitem[Hanif18]{Hanif18}  Youssef HANYF, "L'indexation et la recherche par similarité dans les bases de données complexes : une approche métrique.", THESE DE DOCTORAT NATIONAL, Université Chouaib Doukkali, Faculté des Sciences– El Jadida, 2018.
	
	\bibitem[Haralick73]{Haralick73} Haralick, R.M., Shanmugam, K., et Dinstein, I. "Textural Features for Image Classification", IEEE Transactions on systems, man, and cybernetics. N°6, pages: 610–621. 1973.
	
	\bibitem[Houari10]{Houari10} Kamel Houari. "Recherche d’images par le contenu". Thèse de doctorat. Université de Constantine. 2010.
	
	\bibitem[Hu62]{Hu62} Hu, M. K., "Visual pattern recognition by moment invariants, computer methods in image analysis". Transactions on Information Theory, 8(2) :179-187, 1962.
	
	\bibitem[Hubbard95]{Hubbard95} B. Burke Hubbard, "Ondes et ondelettes". Sciences d'Avenir. Belin, 1995.
	
	
	\bibitem[Imane12]{Imane12} Imane Nedjar. "CMBIR (content medical based image retrieval)
	développement d’outil logiciel d’annotation d’images médicales, utilisant les
	méthodes d’indexation par descripteurs invariants de contenus". Mémoire de
	magister. Université de Tlemcen. 2012.
	
	\bibitem[Jean01]{Jean01}  Jeannin, S. “MPEG-7 Visual part of experimentation model, Version 9.0”, ISO/IEC JTC1/SC29/WG11 N, 3914. 2001.
	
	\bibitem[Jiang91]{Jiang91}  Jiang, X. Y.,  Bunke, H., "Simple and fast computation of moments. Pattern recognition", 24(8) :801-806, 1991.
	
	\bibitem[Jerome05]{Jerome05} Jérôme Landré. "Analyse multirésolution pour la recherche et
	l’indexation d’images par le contenu dans les bases de données images -
	Application à la base d’images paléontologique trans "tyfipal" ". Thèse de
	Doctorat. Université de bourgogne. 2005.
	
	
	\bibitem[Kender98]{Kender98} J. Kender and B. Yeo. "Video scene segmentation via continuous video coherence". In IEEE Computer Vision and Pattern Recognition, pages 367373,1998.
	
	
	\bibitem[Khouloud09]{Khouloud09} Khouloud Meskaldji, "Extraction et traitement de l’information : Un prototype d’un système de recherche d’images couleurs par le contenu magistère", Université Mentouri de Constantine, 2009.
	
	\bibitem[Kim99]{Kim99}. Kim, W. Y., Kim, Y. S., et Kim, Y. S. “A new region-based shape descriptor”, ISO/IEC MPEG99 M, 5472, 1999.
	
	\bibitem Lance67]{Lance67} Lance, G. N., \& Williams, W. T. "A general theory of classificatory sorting strategies: 1. Hierarchical systems". The computer journal, 9(4), 373-380, (1967).
	
	\bibitem[Levenstein65]{Levenstein65} Levenstein, V. "Binary codes capable of correcting spurious insertions and deletions of ones". Problems of Information Transmission, 1(1), 8-17. (1965).
	
	\bibitem[Linda01]{Linda01} Linda G. Shapiro and George C. Stockman, Computer Vision, Upper Saddle River: Prentice-Hall, 2001.
	
	\bibitem[Lohse93]{Lohse93} A. R. Rao et G. L. Lohse, "Identifying high level features of texture perception. Computer Vision, Graphics and Image Processing , Graphic Models and Image Processing", Vol. 55, pp. 218-233, 1993.
	
	\bibitem[Lowe99]{Lowe99}  Lowe, D. G., Object recognition from local scale-invariant features. In Computer vision, 1999. The proceedings of the seventh IEEE international conference on, volume 2, pages 1150–1157, 1999.
	
	\bibitem[Mallat89]{Mallat89} S.G.Mallat, A theory for multiresolution signal decomposition: the wavelet representation, IEEE Trans. Pattern Analysis and Machine Intelligence, Vol. 11, pp. 674-693, 1989.
	
	\bibitem[Manjunathi96]{Manjunathi96} Manjunathi, B. S., Ma, W. Y., Texture features for browsing and retrieval of image data, IEEE Transactions on pattern analysis and machine intelligence. Vol. 18, n° 8, pages: 837–842, 1996.
	
	\bibitem[Marcelaje80]{Marcelaje80} Marcelaje, S., Mathematical description of the response of simple cortical cells, Journal of Optical Society of America. Vol 70, n° 11, pages: 1297-1300, (1980).
	
	
	\bibitem[Mercier01]{Mercier01} Mercier, D., et Séguier, R. “Utilisation des STANN en audio : illustration en reconnaissance de chiffre ”, Journée Valgo. 2001.
	
	\bibitem[MM11]{MM11} Merabet Nabila et Mahlia Meriem. Recherche d’images par le contenu. Mémoire en master. Université de Tlemcen. 2011.
	
	\bibitem[Mouragnon06]{Mouragnon06} Mouragnon, E., Lhuillier, M., Dhome, M., Dekeyser, F., and Sayd, P., Real time localization and 3d reconstruction. In Computer Vision and Pattern Recognition, 2006 IEEE Computer Society Conference on, volume 1, pages 363–370. IEEE, 2006.
	
	\bibitem[MYB07]{MYB07} Maofu Liu , Yanxiang He \& Bin Ye (2007) "Image Zernike moments shape feature evaluation based on image reconstruction", Geo-spatial Information Science, 10:3, 191-195, DOI: 10.1007/s11806-007-0060-x
	
	\bibitem[Negrel14]{Negrel14} Negrel, R., Picard, D., and Gosselin, P.-H. (2014). Evaluation of second-order visual features for land-use classification. In Content-Based Multimedia Indexing (CBMI), 2014 12th International Workshop on, pages 1–5.
	
	\bibitem[Nguy09]{Nguy09} Nguyen, T. O. “Localisation de symboles dans les documents graphiques”, Thèse de Doctorat, Université Nancy 2. 2009.
	
	\bibitem[Nister04]{Nister04}  Nistér, D., Naroditsky, O., and Bergen, J., Visual odometry. In Computer Vision and Pattern Recognition, 2004. CVPR 2004. Proceedings of the 2004 IEEE Computer Society Conference on, volume 1, pages I–652, (2004).
	
	
	\bibitem[Ouhda19]{Ouhda19}  Mohamed Ouhda. “Apport de la Classification et de la Segmentation dans la Recherche d’Image par le Contenu : Application aux
	Images Médicales”, Thèse de Doctorat, Université Moulay Ismail, Faculté des Sciences et Techniques d’Errachidia. 2019.
	
	\bibitem[Partio02]{Partio02} Partio, M Content-based image retrieval using shape and texture attributes, Thèse de Doctorat, Tampere University of Technology, Department of Electrical Engineering, Institute of Signal Processing. 2002.
	
	\bibitem[Pass97]{Pass97}  Pass, G., Zabih, R., \& Miller, J., (1997),.Comparing images using color coherence vetors. In Fourth ACM Conference on Multimedia, pages 65-73, 1997.
	
	\bibitem[Perronnin10]{Perronnin10}  Perronnin, F., Sánchez, J., and Mensink, T., (2010b),. Improving the fisher kernel for large-scale image classification. In Computer Vision–ECCV 2010, pages 143–156. Springer.
	
	\bibitem[Prokop92]{Prokop92}  Prokop R., Reeves A., A survey of moment based techniques for unoccluded object representation and recognition, CVGIP : Graphical Models and Image Processing, vol. 54, no 5, pp. 438-460, 1992.
	
	\bibitem[Policarpo98]{Policarpo98} Fabio Policarpo, The Computer Image, ACM Press. Pages 298-308. 1998.
	
	\bibitem[Rao93]{Rao93}  A. R. Rao et G. L. Lohse: Towards a texture naming system , Identifying relevant dimension of texture. IBM Research report, RC 19140 (83352), pp.29 , 1993.
	
	\bibitem[RKV95]{RKV95} N. Roussopoulos, S. Kelley, and F. Vincent. Nearest neighbor queries. In Proceedings of the 1995 ACM SIGMOD International Conference on Management of Data, pages 71–79, San Jose, CA, May 1995. 
	
	\bibitem[Rui98]{Rui98} Rui Y., S He A., Huang T., 1998, A modified Fourier descriptor for shape matching in
	MARS, Workshop on Image Databases and Multi Media Search, vol. 8, pp. 165-180.
	
	
	\bibitem[Sonka99]{Sonka99} . Sonka, M., Hlavac, V., et Boyle, R. “Image Processing, Analysis and Machine Vision”, PWS Publishing, 2nd edition. 1999.
	
	\bibitem[Stricker95]{Stricker95}  M.A. Stricker et M. Orengo. Similarity of color images. In SPIE, Storage
	and Retrieval for image Video Databases, pages 381-392. pàà 1995.
	
	
	\bibitem[Stricker94]{Stricker94}  Stricker, M.,  Swain, M., 1994,. The capacity of color histogram indexing. In IEEE, Conference on Computer Vision and Pattern Recongnition, pages 704-708.
	
	\bibitem[Swain91]{Swain91 }  Swain, M.J., Ballard, D.H., Color indexing, International Journal of Computer
	Vision. Vol 7, n° 1, pages: 11-32, 1991.
	
	\bibitem[Tabb06]{Tabb06} Tabbone, S., Wendling, L., et Salmon, J.-P. “A new shape descriptor defined on the Radon transforms”, Computer Vision and Image Understanding. Vol 102, n° 1, pages: 42-51. 2006.
	
	\bibitem[Taubin92]{Taubin92} Taubin, G.,  Cooper, D. B., 1992., Recognition and positioning of rigid objects using algebraic and moment invariants. PhD thesis, Providence, RI, USA.
	
	\bibitem[Teague80]{Teague80} Teague, MR. (1980). Image Analysis via the General Theory of Moments.  J.
	Opt. Soc. Am. 70(8):920-930.
	
	\bibitem[Tuce98]{Tuce98 }  Tuceryan, M., Anil, K. Jain. “Texture Analysis, the Handbook of Pattern Recognition and Computer Vision”, 2nd Edition, World Scientific Publishing Co, River Edge NJ
	USA. Vol 55, n° 2-3, pages: 207-248. 1998.
	
	\bibitem[Uhl91]{Uhl91} J.K. Uhlmann. Satisfying general proximity/similarity queries with metric
	trees. Information Processing Letters, 40(4):175–179, November 1991.
	
	
	\bibitem[Vailaya01]{Vailaya01} Aditya Vailaya, Mário Figueiredo, Anil Jain et HongJiang Zhang. A: Bayesian Framework for Semantic Classification of Outdoor Vacation Images, IEEE Trans. Image Processing, Vol. 10, No. 1, pp. 157-172, 2001.
	
	
	\bibitem[Zavi01]{Zavi01}. Zavidovique, B., Seetharamann, G. “Trees for peano raster based Image processing”, CA CS Technical Report, and Working paper, University of Louisiana. 2001.
	
	
	
	\bibitem[Zezula06]{Zezula06}  Zezula, Pavel, Amato, Giuseppe, Dohnal, Vlastislav, \& Batko, Michal. Similarity search: the metric space approach (Vol. 32): Springer Science \& Business Media. (2006).
	
	\bibitem[Zhan02]{Zhan02} Zhang, R., Zhongfei, Z. “A Clustering Based Approach to Efficient Image Retrieval”, Proceedings of the 14th IEEE International Conference on Tools with Artificial Intelligence (ICTAI’02), pages: 339-346. 2002.
	
	\bibitem[Zhang05]{Zhang05} Zhang D., LU G., 2005, Study and evaluation of different Fourier methods for image retrieval, Image and Vision Computing, vol. 23, pp. 33-49.
	
	%----------------------------------------------------------------------------------------
	\bibitem[ZZ18]{ZZ18}  ZERROUKI Houcemeddin et ZENNAKI Abderrahmane. Développement d’un système de recherche d’images par le contenu. Mémoire en master. Université de Tlemcen. 2018.
	
	
	--------------------------------- Sitographie ---------------------------------
	\bibitem[Site01]{Site01} \url{https://www.pyimagesearch.com/2014/12/01/complete-guide-building-image-search-engine-python-opencv/}
	\bibitem[Site02]{Site02} \url{http://murphylab.web.cmu.edu/publications/boland/boland_node26.html}
	\bibitem[Site03]{Site03} \url{https://mahotas.readthedocs.io/en/latest/features.html#global-features}
	\bibitem[Site04]{Site04} \url{http://www.robots.ox.ac.uk/~vgg/research/texclass/index.html}
	
	\bibitem[Site05]{Site05} \url{https://homepages.inf.ed.ac.uk/rbf/CVonline/LOCAL_COPIES/SHUTLER3/node11.html}
	
\end{thebibliography}


\end{document}  
