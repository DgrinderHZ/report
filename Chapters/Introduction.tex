

\chapter*{Introduction générale}
La recherche d'information peut être considérer parmi les activités quotidiennes de tout être humain d'aujourd'hui. Cette recherche n'a pas commencé avec le développement de l’Internet mais plutôt avant. Avec la révolution numérique de ces dernières décennies et l’avancement continu des capacités de calcul et de stockage des machines, l’information numérique est devenue le cœur de tous les secteurs d’activités. Ces progrès se sont accompagnés d’une expansion rapide des bases d'images, ce qui a compliqué les méthodes de recherche traditionnelle.\\

L'image numérique devient alors une nécessité grâce à son pouvoir expressif et la facilité de son transfert, de ce fait, la quantité d'images sur le web augmente rapidement, ce qui donne naissance à un nouveau besoin que l’on ne connaissait pas auparavant à savoir des outils capables d’aider l’utilisateur dans l’indexation et la recherche d’images.\\

Les premières solutions utilisées pour la recherche d’image sont basées sur la recherche textuelle par des annotations. Cette technique consiste à ajouter à chaque image un ensemble de mots-clés qui vont décrire son contenu visuel. De là, un système de gestion de base sera capable de trouver les images similaires.\\

Aujourd’hui, les moteurs de recherche d’images avancés sont classés en deux catégories selon leur principe de fonctionnement; la première catégorie exploite les caractéristiques visuelles des images et la deuxième catégorie utilise les concepts sémantiques associés avec les images.\\

Nous nous intéressons dans notre travail à la première classe, connue sous le nom de la recherche d’images par le contenu « Content Based Image Retrieval ou CBIR».
Dans ce mode, l’image est représentée par un ensemble de descripteurs
numériques dans une base d'index qui essayent de décrire des attributs visuels différents, généralement de bas niveaux comme la couleur, la forme et la texture. L’utilisateur est appelé à donner une image exemple ou requête qui ressemble à ce qu’il cherche. La recherche consiste ici de mesurer la similarité de manière automatique entre la requête et les images de la base d'index selon certains critères préétablis.\\

Cette mémoire se compose de quatre chapitres qui nous permettront de
présenter les différents aspects de notre travail. Le premier chapitre présente une études bibliographique des généralités sur le traitement d’image numérique et le principe d'un systèmes de recherche d’image par contenu (CBIR). Le deuxième chapitre donne une vue plus détaillée sur les systèmes de recherche d’images par le contenu surtout la partie de descripteur et celle de mesure de similarité. Le troisième chapitre a pour but d’exposer la partie d'accélération de recherche à l'aide de la structure d'index M-Tree. Le dernier chapitre est dédié pour la partie expérimentations et les résultats obtenus par notre système.