
%----------------------------------------------------------------------------------------
%	BIBLIOGRAPHY
%----------------------------------------------------------------------------------------

\bibliographystyle{apalike}
\bibliography{} 
$\left[\textbf{Ouhda19} \right]$ . Mohamed OUHDA. “Apport de la Classification et de la Segmentation dans la
Recherche d’Image par le Contenu : Application aux
Images Médicales”, Thèse
de Doctorat, Université Moulay Ismail, Faculté des Sciences et Techniques d’Errachidia. 2019.

\

$\left[\textbf{Elasna17} \right]$ .Khalid EL ASNAOUI. “Indexation et recherche d’images par le contenu:
Algorithmes et application au Lifelogging”, Thèse
de Doctorat, Université Moulay Ismail, Faculté des Sciences et Techniques d’Errachidia. 207.

\

$\left[\textbf{Nguy09} \right]$ . Nguyen, T. O. “Localisation de symboles dans les documents graphiques”, Thèse
de Doctorat, Université Nancy 2. 2009.

\

$\left[\textbf{Zavi01}\right]$. Zavidovique, B., Seetharamann, G. “Trees for peano raster based Image processing”, CA CS Technical Report, and Working paper, University of Louisiana. 2001.

\

$\left[\textbf{Gonz02}\right]$. Gonzales, R. C., et Woods, R. E. “Digital image processing. Prentice-Hall”, New
Jersey, 8, 14, 27, 28. 2002.

[Haralick73]. Haralick, R.M., Shanmugam, K., et Dinstein, I. Textural Features for Image
Classification, IEEE Transactions on systems, man, and cybernetics. N°6, pages: 610–621.
1973.

\

$\left[\textbf{Gueg07}\right]$. Gueguen, L. “Extraction d’information et compression conjointes des séries temporelles d’images satellitaires”, Thèse de Doctorat, Ecole Nationale Supérieure des Télécommunications Paris. 2007.

[Sonka99]. Sonka, M., Hlavac, V., et Boyle, R. “Image Processing, Analysis and Machine
Vision”, PWS Publishing, 2nd edition. 1999.
\

$\left[\textbf{Tuce98}\right]$. Tuceryan, M., Anil, K. Jain. “Texture Analysis, the Handbook of Pattern Recognition and Computer Vision”, 2nd Edition, World Scientific Publishing Co, River Edge NJ
USA. Vol 55, n° 2-3, pages: 207-248. 1998.


\

$\left[\textbf{Zhan02}\right]$. Zhang, R., Zhongfei, Z. “A Clustering Based Approach to Efficient Image Retrieval”, Proceedings of the 14th IEEE International Conference on Tools with Artificial Intelligence (ICTAI’02), pages: 339-346. 2002.
%----------------------------------------------------------------------------------------

\


$\left[\textbf{Rui 98}\right]$ Rui Y., S He A., Huang T., 1998, A modified Fourier descriptor for shape matching in
MARS, Workshop on Image Databases and Multi Media Search, vol. 8, pp. 165-180.


\

$\left[\textbf{Zhang 05}\right]$ Zhang D., LU G., 2005, Study and evaluation of different Fourier methods for image
retrieval, Image and Vision Computing, vol. 23, pp. 33-49.

\

$\left[\textbf{Prokop 92}\right]$ Prokop R., Reeves A., 1992, A survey of moment based techniques for unoccluded
object representation and recognition, CVGIP : Graphical Models and Image Processing,
vol. 54, no 5, pp. 438-460.

\

$\left[\textbf{Hu 62}\right]$ Hu, M. K., 1962, Visual pattern recognition by moment invariants, computer methods in
image analysis. Transactions on Information Theory, 8(2) :179-187

\

$\left[\textbf{Jiang 91}\right]$ Jiang, X. Y.,  Bunke, H., 1991, Simple and fast computation of moments. Pattern
recognition, 24(8) :801-806.

\

$\left[\textbf{Taubin 92}\right]$ Taubin, G.,  Cooper, D. B., 1992., Recognition and positioning of rigid objects using algebraic and moment invariants. PhD thesis, Providence, RI, USA.

\

$\left[\textbf{Stricker94}\right]$ Stricker, M.,  Swain, M., 1994,. The capacity of color histogram indexing. In IEEE, Conference on Computer Vision and Pattern Recongnition, pages 704-708.

[Flickher 95] Flickner, M., Sawhney, H., Niblack, W., Ashley, J., Huang, Q., Dom, B.,. \& Steele,
D., 1 995, Query by image and video content: The QBIC system. IEEE Computer, pages 23–
32, September 1995.

[Swain91]. Swain, M.J., Ballard, D.H., 1991, Color indexing, International Journal of Computer
Vision. Vol 7, n° 1, pages: 11-32,

[Kender 98] J. Kender and B. Yeo. Video scene segmentation via continuous video coherence. In
IEEE Computer Vision and Pattern Recognition, pages 367373,1998.

[Chang 97] Chang, S. F., Chen, W., Meng, H. J., Sundaram, H., \& Zhong, D.,1997,. VideoQ : An
automated content based video search system using visual cues. In The Fifth ACM
International Multimedia Conference, pages 313-324, 1997

[Pass 97] Pass, G., Zabih, R., \& Miller, J., (1997),.Comparing images using color coherence vetors.
In Fourth ACM Conference on Multimedia, pages 65-73, 1997

\

$\left[\textbf{Mouragnon 06}\right]$ Mouragnon, E., Lhuillier, M., Dhome, M., Dekeyser, F., and Sayd, P., 2006, Real
time localization and 3d reconstruction. In Computer Vision and Pattern Recognition, 2006
IEEE Computer Society Conference on, volume 1, pages 363–370. IEEE.

\

$\left[\textbf{Nister 04}\right]$ Nistér, D., Naroditsky, O., and Bergen, J., (2004), Visual odometry. In Computer Vision and Pattern Recognition, 2004. CVPR 2004. Proceedings of the 2004 IEEE Computer
Society Conference on, volume 1, pages I–652.

\

$\left[\textbf{Mouragnon 06}\right]$ Mouragnon, E., Lhuillier, M., Dhome, M., Dekeyser, F., and Sayd, P., 2006, Real
time localization and 3d reconstruction. In Computer Vision and Pattern Recognition, 2006
IEEE Computer Society Conference on, volume 1, pages 363–370. IEEE.

\

$\left[\textbf{Lowe 99}\right]$ Lowe, D. G., 1999, Object recognition from local scale-invariant features. In Computer
vision, 1999. The proceedings of the seventh IEEE international conference on, volume 2,
pages 1150–1157.

\

$\left[\textbf{Negrel 14}\right]$ Negrel, R., Picard, D., and Gosselin, P.-H. (2014b). Evaluation of second-order visual features for land-use classification. In Content-Based Multimedia Indexing (CBMI), 2014
12th International Workshop on, pages 1–5.

\

$\left[\textbf{Perronnin 10}\right]$ Perronnin, F., Sánchez, J., and Mensink, T., (2010b),. Improving the fisher kernel for large-scale image classification. In Computer Vision–ECCV 2010, pages 143–156. Springer.

$\left[\textbf{Stricker95}\right]$ M.A. Stricker et M. Orengo. Similarity of color images. In SPIE, Storage
and Retrieval for image Video Databases, pages 381-392. pàà 1995.

[Bimbo99] A. Del Bimbo, Visual Information Retrieval. Morgan Kaufmann Publishers, 1999.

[Policarpo98] Fabio Policarpo, The Computer Image, ACM Press. Pages 298-308. 1998.
\\
[Linda01] Linda G. Shapiro and George C. Stockman, Computer Vision, Upper Saddle River:
Prentice-Hall, 2001.

[Rao93] A. R. Rao et G. L. Lohse: Towards a texture naming system , Identifying relevant
dimension of texture. IBM Research report, RC 19140 (83352), pp.29 , 1993.

[Lohse93] A. R. Rao et G. L. Lohse, Identifying high level features of texture perception.
Computer Vision, Graphics and Image Processing , Graphic Models and Image Processing,
Vol. 55, pp. 218-233, 1993

[Ganesan07] G.W. Jiji et L. Ganesan, Comparative analysis of colour models for colour textures
based on feature extraction. International Journal of Soft Computing, p 361–366, 2007.

[Fauqueurs03] Julien Fauqueur. Contributions pour la Recherche d’Images par Composantes Visuelles. Interface
homme-machine [cs.HC]. Université de Versailles-Saint Quentin en Yvelines, 2003. Français. fftel-
00007090ff

[Jerome05] Jérôme Landré. Analyse multirésolution pour la recherche et
l’indexation d’images par le contenu dans les bases de données images -
Application à la base d’images paléontologique trans "tyfipal". Thèse de
Doctorat. Université de bourgogne. 2005.

[Mallat89] S.G.Mallat, A theory for multiresolution signal decomposition: the wavelet
representation, IEEE Trans. Pattern Analysis and Machine Intelligence, Vol. 11, pp. 674-693,
1989.

[Vailaya01] Aditya Vailaya, Mário Figueiredo, Anil Jain et HongJiang Zhang. A: Bayesian
Framework for Semantic Classification of Outdoor Vacation Images, IEEE Trans. Image
Processing, Vol. 10, No. 1, pp. 157-172, 2001.

[Hubbard95] B. Burke Hubbard, Ondes et ondelettes. Sciences d'Avenir.Belin, 1995

M. K. Hu, Visual pattern recognition by moments invariants. Computer methods in
image analysis. Transactions on Information Theory, 8, 1962.

[Houari10] Kamel Houari. Recherche d’images par le contenu. Thèse de doctorat.
Université de Constantine. 2010.

[Khouloud09] Khouloud Meskaldji, Extraction et traitement de l’information : Un prototype d’un
système de recherche d’images couleurs par le contenu magistère, Université Mentouri de
Constantine, 2009.
[Imane12] Imane Nedjar. CMBIR (content medical based image retrieval)
développement d’outil logiciel d’annotation d’images médicales, utilisant les
méthodes d’indexation par descripteurs invariants de contenus. Mémoire de
magister. Université de Tlemcen. 2012.

Julien Fauqueur. Contributions pour la Recherche d’Images par Composantes Visuelles. Interface
homme-machine [cs.HC]. Université de Versailles-Saint Quentin en Yvelines, 2003. Français. fftel00007090f

[Hafner95] J. Hafner, H. Sawhnay, et al. Efficient color histogram indexing for quadratic
form distance functions. IEEE Transactions on Pattern Analysis and Machine
Intelligence (PAMI), 17(7) :729–736, July 1995.

[MM11] Merabet Nabila et Mahlia Meriem. Recherche d’images par le contenu.
Mémoire en master. Université de Tlemcen. 2011.

[ZZ18]  ZERROUKI Houcemeddin et ZENNAKI Abderrahmane. Développement d’un système de recherche
d’images par le contenu. Mémoire en master. Université de Tlemcen. 2018.

[Gabor46]. Gabor, D., “Theory of communication”, Electrical Engineers-Part III: Radio and
Communication Engineering, Journal of the Institution. Vol. 93, n° 26, pages: 429–441.
1946.

[Chawki16]. Chawki, Y. “ Estimation en Analyse Spectrale Haute Résolution Multidimensionnelle : Application en Indexation et au Filtrage des Images Sismiques”, Thèse de Doctorat,
Université Moulay Ismail, Faculté des Sciences et Techniques d’Errachidia. 2016.

[Hanif18] Youssef HANYF, "L'indexation et la recherche par similarité dans les
bases de données complexes : une approche
métrique.", THESE DE DOCTORAT NATIONAL, U n i v e r s i t é C h o u a i b D o u k k a l i
Faculté des Sciences– El Jadida, 2018

[RKV95] N. Roussopoulos, S. Kelley, and F. Vincent. Nearest neighbor queries. In
Proceedings of the 1995 ACM SIGMOD International Conference on Management of Data, pages 71–79, San Jose, CA, May 1995. 

[CiPaZe97] Ciaccia, Paolo, Patella, Marco, \& Zezula, Pavel. (1997). M-tree: An Efficient Access Method for
Similarity Search in Metric Spaces. Paper presented at the Proceedings of the 23rd
International Conference on Very Large Data Bases.

[Chávez01] Chávez, Edgar, Navarro, Gonzalo, Baeza-Yates, Ricardo, \& Marroquín, José Luis. (2001). Searching in
metric spaces. ACM Computing Surveys (CSUR), 33(3), 273-321

[Zezula06]  Zezula, Pavel, Amato, Giuseppe, Dohnal, Vlastislav, \& Batko, Michal. (2006). Similarity search: the
metric space approach (Vol. 32): Springer Science \& Business Media

[Levenstein65] Levenstein, V. (1965). Binary codes capable of correcting spurious insertions and deletions of ones.
Problems of Information Transmission, 1(1), 8-17.

[Deza09] Deza, Michel Marie, \& Deza, Elena. (2009). Encyclopedia of distances: Springer.

[Hamming50] Hamming, Richard W. (1950). Error detecting and error correcting codes. Bell System technical
journal, 29(2), 147-160.

[Lance67] Lance, G. N., \& Williams, W. T. (1967). A general theory of classificatory sorting
strategies: 1. Hierarchical systems. The computer journal, 9(4), 373-380.

[Andaloussi10]. Andaloussi, SJ. “Indexation de l’information médicale. Application à la recherche
d’images et de vidéos par le contenu ”, Thèse de Doctorat, l’Université européenne de Bretagne Télécom Bretagne En habilitation conjointe avec l’Université de Rennes 1 Co-tutelle
avec Faculté des sciences Dhar El Mehraz Fès, Université Sidi Mohamed Ben Abdellah.
2010.

[ElAsnaoui17] El Asnaoui K. "Indexation et recherche d’images par le contenu: Algorithmes et application au Lifelogging", Thèse de Doctorat,
Université Moulay Ismail, Faculté des Sciences et Techniques d’Errachidia. 2017.

[Partio02]. Partio, M Content-based image retrieval using shape and texture attributes, Thèse de
Doctorat, Tampere University of Technology, Department of Electrical Engineering, Institute
of Signal Processing. 2002.

[Marcelaje 80]. Marcelaje, S., (1980), Mathematical description of the response of simple cortical
cells, Journal of Optical Society of America. Vol 70, n° 11, pages: 1297-1300.

[Manjunathi 96]. Manjunathi, B. S., Ma, W. Y., 1996, Texture features for browsing and retrieval of
image data, IEEE Transactions on pattern analysis and machine intelligence. Vol. 18, n° 8,
pages: 837–842.

[Kim99]. Kim, W. Y., Kim, Y. S., et Kim, Y. S. “A new region-based shape descriptor”,
ISO/IEC MPEG99 M, 5472.1999.

[Jean01]. Jeannin, S. “MPEG-7 Visual part of experimentation model, Version 9.0”, ISO/IEC
JTC1/SC29/WG11 N, 3914. 2001.

[Mercier01]. Mercier, D., et Séguier, R. “Utilisation des STANN en audio : illustration en reconnaissance de chiffre ”, Journée Valgo. 2001.

[Tabb06]. Tabbone, S., Wendling, L., et Salmon, J.-P. “A new shape descriptor defined on the
Radon transforms”, Computer Vision and Image Understanding. Vol 102, n° 1, pages: 42-
51. 2006.

[Hu62]. Hu, M.K. “Visual Pattern Recognition by moment invariants”, IRE Transaction on
Information Theory. Vol 8, n°2, pages: 179–187. 1962.\\


Sitographie
[site 01] http://fr.wikipedia.org/wiki/Luminance. Visité le 20/02/2018
[site 02] https://www.supinfo.com/articles/single/5679-traitement-imagedetection-contours-filtre Visité le 28/02/2018
[site 03] https://fr.wikipedia.org/wiki/Traitement_d%27images Visité le
08/03/2018