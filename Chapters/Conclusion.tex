\chapter*{Conclusion générale}
Le développement des utiles technologiques traitants le contenu multimédia a conduit à une augmentation du nombre d’images sur le Web et a rendu le développement au niveau des outils qui organisent et gèrent ces données une exigence. Le but principal de cette mémoire était de traiter le problème de la recherche d’images par le contenu. Un domaine très riche qui présente une variance dans les techniques utilisées qu’on ne peut pas limiter à une approche ou une méthode particulière.\\

Les moteurs de recherches d’images sont classés selon le niveau d'interprétation en deux catégories, ceux qui extraient le contenu bas niveau et ceux qui extraient le contenu sémantique. Dans notre travail nous somme
intéressé par la première catégorie avec la caractérisation des attributs visuels; couleur, texture et forme. La recherche dans ces systèmes se base sur quatre étapes fondamentaux, le choix de descripteur à utiliser pour l'extraction des attributs visuels, le choix d'une mesure de similarité entre ses attributs, l’indexation des images et le calcul de similarité ou la recherche.\\

Il n’existe pas une règle qui impose le choix d’une technique spécifique pour créer un système pertinent. Pour cela, nous avons présenté le principe de fonctionnement d’un système de recherche d’images par le contenu en premier lieu. Ensuite, nous avons étudié les différentes méthodes utilisées
pour l’extraction des attributs visuels avant de détailler l’approche de la mesure de similarité. Ainsi, nous avons présenté une méthode d'index M-tree que nous avons utilisé pour accélérer la recherche. A la fin de ce travail, Nous avons essayé d’appliquer nos connaissances pour développer un système qui permet à l’utilisateur de proposer facilement
une requête et visualiser les résultats. Nous avons aussi testé les différentes approches pour mieux cerner les difficultés et valider les résultats.\\
En résumé, on espère qu’on a atteint notre but d’introduire le domaine de la recherche d’images par le contenu (CBIR), ce domaine qui reste toujours ouverte à de nouvelles pistes de recherches.