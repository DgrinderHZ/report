\begin{center}
	\addsec{Conclusion générale}
\end{center}
L’augmentation du nombre d’images sur le Web a rendu le développement
des outils qui organisent ces données une exigence. Le but principal de ce
mémoire était de traiter le problème de la recherche d’images par le contenu.
Un domaine très large et présente une variance dans les techniques utilisées
qu’on ne peut limiter à une approche ou une méthode particulière.
Les moteurs de recherches d’images sont classés selon le niveau
d'interprétation en deux catégories, ceux qui extraient le contenu bas niveau
et ceux qui extraient le contenu sémantique. Dans notre travail nous somme
intéressé par la première catégorie. La recherche dans ces systèmes base sur
deux étapes fondamentaux, l’indexation des images et le calcul de similarité.
Il n’existe pas une règle qui impose le choix d’une technique spécifique pour
créer un système Irréprochable. Pour cela, nous avons présenté le principe
de fonctionnement d’un système de recherche d’images par le contenu en
premier lieu. Ensuite, nous avons étudié les différentes méthodes utilisées
pour l’extraction des attributs visuels avant de détailler l’approche de la
mesure de similarité.
A la fin de ce travail, Nous avons essayé d’appliquer nos connaissances pour
développer une interface qui permet à l’utilisateur de proposer facilement
une requête et visualiser les résultats. Nous avons aussi testé les différentes
approches pour mieux cerner les difficultés et valider les résultats.
En résumé, on espère qu’on a atteint notre but d’introduire le domaine de la
recherche d’images, ce domaine qui reste toujours ouverte à de nouvelles
pistes de recherches.